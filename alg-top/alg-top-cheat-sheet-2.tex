Define chain homotopy and state same std. facts.

Let \( A_\bullet, B_\bullet \) be chain complexes and \( f,g : A_\bullet \to B_\bullet \) chain maps.

A chain homotopy between \( f \) and \( g \) is a sequence of group homs. \( P_n : A_n \to B_{n+1} \)
(or graded homs. \( P : A_\bullet \to B_\bullet \) of degree 1),
such that \(f - g = d \circ P + P \circ d\), i.e. 
\(f_n - g_n = d^B_{n+1} \circ P_n + P_{n-1} \circ d^A_n \quad \text{for every } n \in \mathbb{Z}.\)

If there is a chain homotopy \(P\), the chain maps \(f\) and \(g\) are said to be chain-homotopic and we write \(f \simeq g\).

We have the following propositions:

The chain map \(f\) is a chain homotopy equivalence if there is another chain map \(h: B_\bullet \to A_\bullet\) such that \(h \circ f \simeq \text{id}_{A_\bullet}\), 
and \(f \circ h \simeq \text{id}_{B_\bullet}\). Such an \(h\) is called a chain homotopy inverse.
Proof. 
Let \(P\) be a chain-homotopy between \(f\) and \(g\), so that \(f = g + d \circ P + P \circ d\). From the explicit description of \(f_*\) and \(g_*\), it follows

\(f_*([a]) = [f(a)] = [g(a) + d(P(a)) + P(d(a))] = [g(a)] = g_*([a])\)
for every \([a] \in H_n(A_\bullet)\). Here we have used that \(d(a) = 0\) (since \(a \in Z_n(A_\bullet) = \ker d_n\)) and that \(d(P(a)) \in B_n(B_\bullet)\), hence this summand does not change the homology class.

If \( f, g : A_\bullet \to B_\bullet \) are chain homotopic, then \( f_{\ast} = g_{\ast} : H_n(A_\bullet) \to H_n(B_{\bullet})\)
for every \( n \in \mathbb{Z} \).

Let \( h : B_\bullet \to A_\bullet \) be a chain hom. inv. of \( f_{\bullet} \). Then \( h \circ f \simeq \text{id}_{A_{\bullet}} \), \( f \circ h \simeq \text{id}_{B_{\bullet}} \) and the previous we have that \( (h_\ast \circ f_{\ast} = \text{id}_{H_n(A)} \), \( (f_\ast \circ h_{\ast} = \text{id}_{H_n(B)} \). The result follows.


Define a Homotopy, give some examples and essential properties

Two continuous maps \( f, g : X \to Y \) are homotopic if there is a continuous map
\( H : X \times [0, 1] \to Y \) such that \( H(x, 0) = f(x) \) and \(H(x, 1) = g(x)\) for every \( x \in X \). 
In this case we write \( f \simeq g \).

Similarly as for chain homotopies, \( f : X \to Y \) is a homotopy equivalence if there is some cont. \( g : Y \to X \)
called homotopy inverse with \( g \circ f \simeq \text{id}_X \) and \( f \circ g \simeq \text{id}_{Y} \).
We write \( X \simeq Y \) and say that \( X \) and \( Y \). Homotopy equivalent or have the
same homotopy type.

If \( X, Y \) are homeomorphic, then they have the same homotopy type, but the converse does not necessarily hold.

\( X \) is path-connected if and only if \( Y \) is.
\( X \) is simply connected if and only if \( Y \) is.

Let \( C \subseteq \mathbb{R}^n \) be convex and \( f, g : [0, 1] \to C \) paths with the same endpoints, then the
linear homotopy is given by
\( H(s, t) = (1-t)f(s) + tg(s) \)

\( R^n \) or more generally any space homeomorphic to a ball, has the same homotopy type as a point \( p \) inside it, since
\( H(t, x) = tp + (1-t)\text{id}_{\mathbb{R}^n}(x) \)

Prove that homotopic maps induce the same chain map between homologies

We have the powerful theorem \( 3.2.1 \), which asserts that if \( f,g : X \to Y \) are homotopic,
then \( f_{\ast} = g_{\ast} : H_n(X) \to H_n(Y) \) for all \( n \in \mathbb{N} \).

The core idea in the proof is to show the existence of chain homotopy between \( f_{\sharp},g_{\sharp} : C_\bullet(X) \to C_\bullet(Y) \).
To this end, let \( \Delta^n \)be standard \( n \)-simplex and consider the prism \(\Delta^n \times [0, 1] \subseteq \mathbb{R}^{n+2} \)
with new vertices due to the copies of the original simplex at the "bottom" \( v_i \coloneqq  (e_i, 0) \) and \( w_i \coloneqq  (e_i, 1) \) at the "top"
(See the attached image). The prism can then be decomposed into the union of \( (n+1) \)-simplices \( \Delta_i = \langle v_0, \dots, v_i, w_i, \dots, w_n \rangle \).

We can "sweep" through each of these \( \Delta_i \) by moving \( w_i \) down to \( v_i \) and get an \( n \)-simplex
\(\langle (e_i, t), w_1, \dots, w_n \rangle \) for each \( t \in [0, 1] \).

Let now \( H : X \times I \to Y\) be a homotopy from \( f \) to \( g \) and \( \sigma : \Delta^n \to X \in C_n(X) \).
Extending it by the identity \( \tilde{\sigma} \coloneqq \sigma \times \text{id}_{[0, 1]} : \Delta^n \times I \to X \times I \)
and similarly \( \tilde{H} = H \circ \tilde{\sigma} : \Delta^n \times I \to Y \) has the desired side effect of satisfying
\( \tilde{H}|_{\langle v_0, \dots, v_n\rangle} = \tilde{H}|_{\Delta^n \times \{0\}} = f \circ \sigma \) and 
\( \tilde{H}|_{\langle w_0, \dots, w_n\rangle} = \tilde{H}|_{\Delta^n \times \{1\}} = g \circ \sigma \),
so its a homotopy between \( f \circ \sigma, g \circ \sigma \). 
We then define the prism operator \( P_n : C_n(X) \to C_{n+1}(Y) \)
\( P(\sigma) \coloneqq \sum_{i=0}^n (-1)^i \tilde{H}|_{\Delta_i} = \sum_{i=0}^n (-1)^i \tilde{H}|_{\langle w_0, \dots, v_i, w_i, w_n\rangle}\)

With some additional effort it can then be shown that this indeed a chain homotopy.

Corollary
If \(f: X \to Y\) is a homotopy equivalence, then \(f_*: H_n(X) \to H_n(Y)\) is an isomorphism for every \(n \in \mathbb{N}\).
Proof. Let \(g: Y \to X\) be a homotopy inverse. From \(g \circ f \simeq \text{id}_X\) and Theorem 3.2.1 it follows
\(g_* \circ f_* = (g \circ f)_* = (\text{id}_X)_* = \text{id}_{H_n(X)},\)
and analogously \(f_* \circ g_* = \text{id}_{H_n(Y)}\). Thus \(f_*\) is a group isomorphism, with \((f_*)^{-1} = g_*\).

If \(X\) is a contractible space, i.e. \(\text{id}_X\) is homotopic to a constant map (e.g. \(X \subseteq \mathbb{R}^k\) is an open or closed ball), then \(H_0(X) \cong \mathbb{Z}\) and \(H_n(X) = 0\) for every \(n > 0\).

If \(X\) is homotopic to a constant map, then \(X\) is homotopy equivalent to a point \(\{p\}\) whose Homology group is the stated one and since the homotopy
from the contraction is invertible, \( H_n(X) \sigmeq H_n(\{p\})\), by the previous.

One a more abstract note, the theorem asserts that the homology \( H_n \) is a functor between the homotopy category \( \textbf{HTop} \) 
(topological spaces with homotopy classes of cont. maps as morphisms) and \( \text{AbGrp} \).

A consequence

Define the different kinds of retracts and their relation to homology groups

Let \(X\) be a topological space and \(A \subseteq X\) a subset, and denote by \(\iota : A \to X\) the inclusion map. 

1. \(A\) is a retract of \(X\), if there is a retraction map, i.e. a continuous map \(r: X \to A\) such that \(r|_A = r \circ \iota = \text{id}_A\).
More explicitely for all \( x \in X, a \in A\) we have \( r(x) \in A \) and \( r(a) = a \) and continuous \( r \), which is usually the hard part.
2. \(A\) is a deformation retract if there is a retraction \(r: X \to A\) such that \(\iota \circ r \simeq \text{id}_X\).
More explicitely for every \( x \in X \) and \( a \in A \) there is some \( H : X \times I \to A \) with \( H(x, 0) = x \) and \( H(x, 1) = r(x) \).
3. \(A\) is a strong deformation retract, if its is a deformation retract and also satisfying \( H(t, a) = a \) for all \( t \in [0, 1] \).

Every non-empty space is trivially retractable to some point in it.
If \( X \) is Hausdorff then \( A \) must be closed.

A standard example of a strong deformation retract are \( S^n \) and \( \mathbb{R}^{n+1}\setminus \{0\} \),
and the homotopy is
\( H(x, t) = (1 - t)x + t\frac{x}{\|x\|} \)

The situation where a deformation retract is not strong is a little bit more elaborate.

Fact:
If \(A \subseteq X\) is a deformation retract, then \(\iota_* : H_n(A) \cong H_n(X)\) are isomorphisms for every \(n \in \mathbb{N}\).
(To see this note that deformation retracts are a homotopy equivalence between \( A \) and \( X \)).

For the relative homology of \( (X, A) \) we have that the LSE simplifies to an SES 

\( 0 \to H_n(A) \to H_n(X) \to H_n(X, A) \to 0 \)
where the middle two morphisms are \( \iota_{\ast} \) and the canonical \( \pi \).

Moreover it holds the splitting lemma since \( r_\ast \iota_\ast = \text{id}_{H_n(A)} \), so we can write
\( H_n(X) \simeq H_n(A) \oplus H_n(X, A) \).

Define the fundamental group

Given a topological space \(X\) and a base point \(x_0 \in X\), the \textbf{fundamental group} of \(X\) at \(x_0\), denoted by \(\pi_1(X, x_0)\), is the set of homotopy classes of loops based at \(x_0\).

Recall the definitions
Loop: A loop based at \(x_0\) is a continuous map \(f: [0, 1] \to X\) such that \(f(0) = f(1) = x_0\). 
Homotopy class: The homotopy class of a loop \(f\), denoted by \([f]\), is the set of all loops based at \(x_0\) that are homotopic to \(f\).
The group structure is created by the following components:
1. Identity: The constant loop at \(x_0\) represents the identity element.
2. Multiplication: The product of two homotopy classes \([f]\) and \([g]\) is defined as \([f] \cdot [g] = [f * g]\), where \(f * g\) is the loop obtained by 
first traversing \(f\) and then traversing \(g\).
More explicitely:
\[
    (\gamma \cdot \gamma')(t) = \begin{cases}
    \gamma(2t) & \text{if } t \in [0, \frac{1}{2}], \\
    \gamma'(2t - 1) & \text{if } t \in [\frac{1}{2}, 1].
    \end{cases}
\]
3. Inverse: The inverse of a homotopy class \([f]\) is the homotopy class of the loop obtained by traversing \(f\) in the reverse direction, denoted by \([f]^{-1} = [f(1-t)]\).

Example

* The fundamental group of a simply connected space (like a disk or a sphere) is the trivial group (containing only the identity element), as all loops can be continuously shrunk to a point.
* The fundamental group of a circle is the group of integers \(\mathbb{Z}\), as loops can wind around the circle multiple times in either direction. 

State the relation between the fundamental group and first homology group.

Recall: 
The commutator \( [a, b] \coloneqq  ghg^{-1}h^{-1} = (gh)(hg)^{-1}\) 
The commutator subgroup,
\( G' \coloneqq \{ [a, b] | a, b \in G \}\}\)
The abelianization of \( G \)
\( G^{\text{ab}} \coloneqq G/G' \)
with the universal property that for each \( f : G \to A \), where \( A \) is abelian we have a unique \( g : G^{\text{ab}} \to A \),
s.t. \( f = g \circ \pi \), where \( \pi : G \to G^{\text{ab}} \) is canonical.

Theorem 3.3.2. Let \(X\) be any topological space and \(x_0 \in X\) any point. Then the map
\(h: \pi_1(X, x_0) \to H_1(X), \quad [\gamma] \mapsto [\gamma]\)
is a well-defined, also called the Hurewicz homomorphism. 
If moreover \(X\) is path-connected, then \(h\) is surjective and \(\ker h\) is the commutator subgroup of \(\pi_1(X)\), i.e. \(H_1(X) \simeq \pi_1(X)^{\text{ab}}\).
The proof consists of the following steps:
Prove that if \( \gamma, \gamma' \) are homotopic they're also homologous, this is done in a similar manner as for the prism operator,
by defining the homotopy \( H : [0, 1] \times [0, 1] \to X \) with \( H(s, 0) = \gamma(s) \), \( H(s, 1) = \gamma'(s) \)
and \( H(0, t) = H(1, t) = x_0 \). Split the square domain of \( H \) via the diagonal \( (0, 0), (1, 1) \) into 2 triangles \( T_1, T_{2}\) and consider the restrictions
\( \sigma_i \coloneqq H|_{T_{i}} \), then they're simplices on \( X \) and \( \sigma \coloneqq \sigma_{1} - \sigma_{2} \in C_{2}(X)\) , with boundary
\( H|_{\langle p_0, p_1 \rangle} - H|_{\langle p_3, p_2 \rangle} = \gamma - \gamma'\).

Then define the triangle \( \Delta \) with vertices \( p_0 = (0, 0),\ p_{1} = (1, \frac{1}{2}), p_2 = (0, 1) \) and the singular simplex \( \sigma : \Delta \to X\)
given by \( \sigma(s, t) \coloneqq (\gamma \cdot \gamma')(t) \) then \( \Partial \sigma = \sigma|_{\langle p_1, p_2 \rangle} - \sigma|_{\langle p_0, p_2 \rangle} + \sigma|_{\langle p_0, p_1 \rangle} = \gamma' - (\gamma \cdot \gamma') + \gamma = 0\), so \( \gamma \cdot \gamma' \sim \gamma + \gamma' \).
And we have indeed a well-defined homomorphism.
Surjectivity is not too suprising due to path connectedness, since every \( \gamma : \Delta^1 \to X \) is already a cycle and moreover
for every cycle \( \sigma \in Z_1(X) \) not starting/ending in \( x_0 \) we can find a path \( \alpha(0) = x_0,\ \alpha(1) = x_1 \),
and then \( \alpha^{-1} \cdot \gamma \cdot \alpha \) is a loop starting/ending at \( x_0 \), which is homologous to \( \sigma \).

The last part is rather involved, although one direction \( \pi_1(X)^{\text{ab}} \subseteq \text{ker} h\) is trivial.


Define affine (singular) simplices, the cone operator and the barycentry subdivision operator
with std. facts.

Remember that affine maps are invariant under convexity, meaning, if \(f : M \to M'\) (both domain/codomain convex)
it holds \( f(\sum_{i=0}^k t_i p_i  = \sum_{i=0}^k t_if(p_i) \) for any convex set of points \( \{p_i\}_{i \in [k]} \).

Let \( M \) be convex and \( \Delta^n = \langle e_0, \dots, e_n \rangle\) the standard \( n \)-simplex.
Given n+1 points \(p_i \in M\) let \( [p_0, \dots, p_n] \) decode the unique affine mapping \( f \) with \( f(p_i) = e_i \),
which we also refer to as affine singular simplex.
We can then consider the subgroups \( A_n(M) \) of the free group on the (usual) singular simplexes \( C_n(M) \), which only contains
linear combinations of such affine maps and obtain the complex of of affine singular chains.

The main benefit of switching the definition is to not require our points to be linearly independent, hence the singular simplices may have a lower dimensional image,
so its more a technicality. Hatcher completely omits this detail for example.

Let \( b \in M \) any point
and define the cone operator by \( \mathcal{C}_b(\sigma) = [b, p_0, \dots, p_n] \) with \( \sigma = [p_0, \dots, p_n] \) (an affine singular simplex)
and extending it linearly for arbitrary chains.
We implicitely assume to work over the augmented chain complex so we must consider the degenerate case \( \mathcal{C}_b([\emptyset]) = [b] \).

We then have 
\( \partial \circ \mathcal{C}_b + \mathcal{C}_b \circ \partial = \text{id} \)
for all \( \sigma \in A_n(M) \), so \( \mathcal{C}_b \) is a homotopy between the identity and the zero maps of \( A_\bullet(M) \).


The barcycenter of an affine singular simplex \( \sigma = [p_0, \dots, p_n] \) in \( M \) is \( b(\sigma) \coloneqq \frac{\sum_{i=0} p_i}{n+1} \),
this inspires the barycentric subdivision operators \( \mathcal{B}_n : A_n(M) \to A_n(M) \) which are inductively defined via \( \mathcal{B}_n = \text{id} \)
for \( n \leq 0 \) (augmented case included) and otherwise \( \mathcal{B}_n(\sigma) \coloneqq \mathcal{C}_{b(\sigma)}(\mathcal{B}_{n-1}(\partial \sigma)) \)
and extending it linearly for arbitrary chains.

The geometric interpretation of it goes like this: the barycentric subdivision of a
simplex is obtained by first subdividing all its facets, and then taking the cone over each part
with vertex at the barycenter (with suitable signs). The basic case is the barycentric subdivision
of points (0-simplices), which are the points themselves becasue \( \mathcal{B}_0 = \text{id} \).

(Insert here image from Hatcher page 120 Chapter 2).

Let \( \sigma \) be an affine singular \( n \)-simplex in \( \mathbb{R}^m \). Then each affine singular simplex
\( \sigma' \in \mathcal{B}_n(\sigma) \) has diameter \( d(\sigma') \leq \frac{n}{n+1}d(\sigma) \),
and for the diameter it holds \( d(\sigma) = \sup_{0 \leq i < j < n} \|p_i - p_j\| \).

The affine barycentric subdivision \( \mathcal{B} : A_\bullet(M) \to A_\bullet(M) \) is a chain map
\( \partial_n \circ \mathcal{B}_n = \mathcal{B}_{n-1} \circ \partial_n \) for every \( n \in \mathbb{Z} \).

Define the group hom. \( \mathcal{T}_n : A_n(M) \to A_{n+1}(M) \) by 
\( \mathcal{T}_n = 0 \) for \( n \leq 0 \)
and otherwise \( \mathcal{T}_{n}(\sigma) \coloneqq \mathcal{C}_{b(\sigma}(\sigma -  \mathcal{T}_{n-1}(\partial \sigma)) \) extend it linearly for arbitrary chains.
Then \( \mathcal{T} \) is a chain homotopy between \( \text{id} \) and \( \mathcal{B} \) so \( \partial \circ \mathcal{T} + \mathcal{T} \circ \partial = \text{id} - \mathcal{B} \).

Build up the prerequisites for the theorem of small chains and state it.

We first generalize the barycentric subdivision operators from affine (singular) simplexes to standard singular simplexes,
Define \( \mathcal{S}_n : C_n(X) \to C_n(X) \) by \( \mathcal{S}_n(\sigma) := \sigma_{\sharp}\mathcal{B}_n(\text{id}_{\Delta^n})\)
and also \( h_n(\sigma) \coloneqq \sigma_{\sharp}\mathcal{T}(\text{id}_{\Delta^n}) \) where \( \text{id}_{\Delta^n} \)
is the identity map of the standard simplex (which is of course also an affine (singular) simplex).
Then \( \mathcal{S}_n, h_n \) inherit all the properties we previously determined.
Moreover \( \mathcal{S}^m \) is chain homotopic to the identity and a chain homotopy is given by
\( h^{(m)} \coloneqq h \circ (\text{id} + \mathcal{S} + \dots + \mathcal{S}^{m-1} \).

Let \( \mathcal{U} = \{U_i\}_{i \in I} \) be an open cover of \( X \), then a singular simplex \( \sigma : \Delta^n \to X \)
is \( \mathcal{U} \)-small if its image is contained in one of the \( U_i \in \mathcal{U} \).
A singular n-chain \( \sum_j a_j \sigma_j \) is \( \mathcal{U} \)-small if each \( \sigma_j \) is.

We denote by \( C^{\mathcal{U}}_n(X) \subseteq C_n(X) \) the subgroup of \( \mathcal{U} \)-small \( n \)-chainsA, which is freely
generated by the \( \mathcal{U} \)-small \( n \)-simplices. This creates a (\( \mathcal{U} \)-small) subcomplex since the boundary
of each singular simplex must also be \( \mathcal{U} \)-small.

The theorem of small chains then states:
The inclusion \(\iota: C^{U}_\bullet(X) \to C_\bullet(X)\) is a chain homotopy equivalence. 
More precisely, there is another chain map \(p: C_\bullet(X) \to C^{U}_\bullet(X)\) such that \(p \circ \iota = \text{id}_{C^{U}_\bullet(X)}\) 
and \(\iota \circ p \simeq \text{id}_{C_\bullet(X)}\).

In particular, \(\iota_*: H_n(C^{U}_\bullet(X)) \to H_n(X)\) is an isomorphism.
The proof goes along the following lines:
Since \( U_i \) cover \( X \) \( \sigma^{-1}(U_i) \) is a cover of \( \Delta^n \) for each \( \sigma \),
by a standard theorem of analysis there is the Lebesgue number \( \lambda \) s.t. for each \( p \in \Delta^n \)
the open ball \( B_\lambda(p) \subset U_i \) for some \( i \). Thus there is some \( m \) such that
the iterated barycentric subdivision \( \mathcal{S}^m(\sigma) = \sigma_{\sharp}(\mathcal{B}^m(\Delta^n)) \)is \( \mathcal{U} \)-small.
Define then \( D_n(\sigma) \coloneqq h^{(m(\sigma))} \), then with a lot of effort it can be shown that \( \rho \coloneqq \text{id} - \partial \circ D - D \circ \partial \)
is the desired chain map and \( D_n \) the chain homotopy.



Derive how a short exact sequence of complexes can be turned into a long exact sequence.

Suppose we have the (not necessarily and most of time not short) sequence
\( C_n(A) \to C_n(B) \to C_n(C) \)
where \( A_\bullet, B_\bullet, C_\bullet \) are chain complexes
with chain maps \( f_n, g_n \) respectively
Our goal is to extend this to a long chain
\( \dots H_{n+1}(C) \to H_n(A) \to H_n(B) \to H_n(C) \to H_{n-1}(A) \to \dots \).
To this end consider the commutative diagram
(insert diagram from page 33. after Example 5.1.4)

Suppose that \([c] \in H_n(C_\bullet)\) is represented by a cycle \(c \in Z_n(C_\bullet) \subset C_n\), i.e. \(d(c) = 0\). 

Since \(g\) is surjective, there is \(b \in B_n\) such that \(g(b) = c\). Since \(g\) is a chain map, we have

\[g(d(b)) = d(g(b)) = d(c) = 0.\] 

Thus \(d(b) \in \ker g = \text{im } f\) by exactness.

Since \(f\) is injective, there is a unique \(a \in A_{n-1}\) such that \(f(a) = d(b)\). Moreover \(f(d(a)) = d(f(a)) = d(d(b)) = 0\). The injectivity of \(f\) gives \(d(a) = 0\), thus \(a \in Z_{n-1}(A_\bullet)\) defines a homology class \([a] \in H_{n-1}(A_\bullet)\).

This allows us to fix the connecting homomorphism 

\(\Phi_n([c]) := [a]. \qquad (5.4)\)

We then need to verify that the class \([a] \in H_{n-1}(A_\bullet)\) is independent of the choices of \(c\) and \(b\).

Suppose \(c, c' \in Z_n(C_\bullet)\) are two homologous \(n\)-cycles, i.e. \([c] = [c'] \in H_n(C_\bullet)\), and choose \(b, b' \in B_n\) such that \(c = g(b)\) and \(c' = g(b')\). Denote \(a, a' \in A_{n-1}\) the unique elements such that 
\(f(a) = d(b) \text{ and } f(a') = d(b').\)

Since \(c \sim c'\), there is \(\bar{c} \in C_{n+1}\) such that \(d(\bar{c}) = c' - c\). Since \(g\) is surjective, we can find \(\bar{b} \in B_{n+1}\) such that \(g(\bar{b}) = \bar{c}\). 
Since \(g\) is a chain map, we have \(g(d(\bar{b})) = d(g(\bar{b})) = d(\bar{c}) = c' - c = g(b') - g(b)\). 
Thus \(b' - b - d(\bar{b}) \in \ker g = \text{im } f\), so there is a unique \(\bar{a} \in A_n\) such that \(b' - b - d(\bar{b}) = f(\bar{a})\). 
Applying \(d\), and since \(f\) is a chain map, we obtain \(f(d(\bar{a})) = d(f(\bar{a})) = d(b' - b) - d^2(\bar{b}) = d(b' - b) = f(a') - f(a)\). 
The injectivity of \(f\) gives \(d(\bar{a}) = a' - a\), hence \([a'] = [a]\) as wanted. 

To show that \(\phi_n\) is a group homomorphism, let \(c, c' \in Z_n(C_\bullet)\) represent two homology classes \([c], [c'] \in H_n(C_\bullet)\), 
choose \(b, b' \in B_n\) such that \(g(b) = c\) and \(g(b') = c'\), and denote by \(a, a' \in A_{n-1}\) the unique elements such that \(f(a) = d(b)\) and \(f(a') = d(b')\), 
so that \(\phi_n([c]) = [a]\) and \(\phi_n([c']) = [a']\). 

To compute \(\phi_n([c] + [c']) = \phi_n([c + c'])\) we can take \(b + b'\) (since \(g(b + b') = g(b) + g(b') = c + c'\)). Then \(d(b + b') = d(b) + d(b') = f(a) + f(a') = f(a + a')\), so that \(\phi_n([c + c']) = [a + a'] = [a] + [a'] = \phi_n([c]) + \phi_n([c'])\), as wanted.

We conclude \( \phi_n : H_n(C_\bullet) \to H_{n-1}(A_\bullet) \) is a well-defined group homomorphism.
