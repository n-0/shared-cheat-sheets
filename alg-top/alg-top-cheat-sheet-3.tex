Prove the existence of the Mayer-Vietories Sequence

We now come back to the discussion at the beginning of the section. Let \(\mathcal{U} = \{U, V\}\) be a covering of \(X\), i.e. the subsets \(U, V \subset X\) satisfy 
\(\text{int}\ U \cup \text{int}\ V = X\). Consider the inclusions \(i_U : U \cap V \to U\), \(i_V : U \cap V \to V\), \(j_U : U \to X\) and \(j_V : V \to X\), 
and denote \(C^{U}_\bullet(U + V) = C^{U}_\bullet(X)\) the complex of \(U\)-small singular chains. We define then the chain morphisms
\(f = (i_U, -i_V) : C_\bullet(U \cap V) \to C_\bullet(U) \oplus C_\bullet(V), \quad c \mapsto (c, -c)\)
and
\(g = (j_U + j_V) : C_\bullet(U) \oplus C_\bullet(V) \to C_\bullet(U + V), \quad (c_U, c_V) \mapsto c_U + c_V.\)

The inclusions are just there to make the proof trivial that this is indeed a chain map.
Theorem 5.2.1. With the above notations, the following hold:

1. \(f\) and \(g\) give a short exact sequence of complexes
\(0 \to C_\bullet(U \cap V) \xrightarrow{f_\sharp} C_\bullet(U) \oplus C_\bullet(V) \xrightarrow{g_\sharp} C_\bullet(U + V) \to 0.\)

2. There is a long exact sequence of homology groups: the Mayer-Vietoris sequence in homology
\(\dots \xrightarrow{\delta_n} H_n(U \cap V) \xrightarrow{(i_{U*}, -i_{V*})} H_n(U) \oplus H_n(V) \xrightarrow{j_{U*} + j_{V*}} H_n(X) \xrightarrow{\delta_n} H_{n-1}(U \cap V)\)

Proof:
By the definition of small chains, each the image of each singular simplex is either contained in \( U \) or \( V \), making \( g \) surjective.
It is trivial to assert that \( f \) is injective since \( (0, 0) = f(a) \) implies already that \( a = 0 \).
It should be straighforward to see that \( \text{ker}\ g = \text{im} f \).
Thus we can use the existence of the connecting homomorphism \( \phi_n \) together with the theorem of small chains, to assert
\( H_n(C_{\bullet}(U + V)) = H_n(C^{\mathcal{U}}_{\bullet}(X)) \simeq H_n(X) \)


The connecting homomorphism can be a little simplified here,
let \( [c] \in H_n(X) \simeq H_n(U + V) \) then by the Theorem of small chains we have cycles \( c = c_{U} + c_{V} = g(c_{U}, c_{V}) \)
Moreoso there is a unique \( a \in C_{n-1}(U \cap V) \) with \( f(a) = \partial(c_U, c_V) \), but then \( a = \partial c_U = -\partial c_V  \)
so \( \phi_n([c_U + c_V]) = [\partial c_U] = - [\partial c_V] \).

Mayer-Vietoris also works in reduced homology, where we instead use the augmented chain complexes 
\( \tilde{C}_\bullet(U), \tilde{C}_\bullet(V), \tilde{C}_\bullet(U \cap V) \),
one needs to be careful however should \( U \cap V = \emptyset \) then we still have \( H_{-1}(\emptyset) \simeq \mathbb{Z} \).


Compute the (reduced) Homology of a sphere with the help of Mayer-Vietoris

Example 5.3.1. We can now compute the (reduced) homology of a sphere \(S^n\). For \(n = 0\), \(S^0 \subset \mathbb{R}^1\) consists of two points \(\{1, -1\}\). Thus 
\[H_k(S^0) = \begin{cases}
\mathbb{Z}^2 & \text{for } k = 0, \\
0 & \text{for } k \neq 0,
\end{cases} \quad \text{and} \quad \tilde{H}_k(S^0) = \begin{cases}
\mathbb{Z} & \text{for } k = 0, \\
0 & \text{for } k \neq 0.
\end{cases}\]

We can now inductively write \(S^n\) as the union of the (slightly enlarged) hemispheres

\(U := \{(x_0, \dots, x_n) \in S^n : x_n > -\epsilon\}\)
and 
\(V := \{(x_0, \dots, x_n) \in S^n : x_n < \epsilon\},\)

which are both homeomorphic to a ball and intersect in a small neighbourhood of the equator
\(U \cap V = \{(x_0, \dots, x_n) \in S^n : -\epsilon < x_n < \epsilon\}.\)

In particular \(U, V \simeq \{p\}\) and \(U \cap V \simeq S^{n-1}\) (making \(\epsilon \to 0\)), and the Mayer-Vietoris long exact sequence gives
\(0 = \tilde{H}_k(U) \oplus \tilde{H}_k(V) \to \tilde{H}_k(S^n) \to \tilde{H}_{k-1}(S^{n-1}) \to \tilde{H}_{k-1}(U) \oplus \tilde{H}_{k-1}(V) = 0.\)
I.e. the connecting homomorphisms are isomorphisms, and by induction we have

\[\tilde{H}_k(S^n) = \begin{cases}
\mathbb{Z} & \text{for } k = n, \\
0 & \text{for } k \neq n,
\end{cases} \quad \text{and thus} \quad H_k(S^n) = \begin{cases}
\mathbb{Z} & \text{for } k = 0 \text{ or } k = n, \\
0 & \text{otherwise.}
\end{cases}\]

After this computation, the implication of homeomorphic spaces having isomorphic homologies assert that 
two spheres \(S^n\) and \(S^m\) of different dimensions \(n \neq m\) are not homeomorphic.

Also
For any \(n \geq 1\) there is no retraction from the disk \(D^n\) to its boundary, the sphere \(S^{n-1}\).
Proof: 

Let \(\iota : S^{n-1} \to D^n\) be the inclusion of \(S^{n-1}\) as the boundary of \(D^n\), 
and suppose there is a retraction \(r : D^n \to S^{n-1}\), i.e. \(r|_{S^{n-1}} = r \circ \iota = \text{id}_{S^{n-1}}\). 
Then the composition
\(r_* \circ \iota_* : H_{n-1}(S^{n-1}) \cong \mathbb{Z} \to H_{n-1}(D^n) = 0 \to H_{n-1}(S^{n-1}) \cong \mathbb{Z}\)
should be the identity, which is clearly impossible.


State and prove the Brower fix point Theorem
For any \( n \geq 1 \), any continuous map \( f : D^n \to D^n \)
has a fixed point \( p \in D^n \), (i.e. \(f(p) = p\)).
Proof: 
Suppose on the contrary that \(f(x) \neq x\) for all \(x \in D^N\). 
Then we can define a map \(r: D^n \to S^{n-1}\) by letting \(r(x)\) be the point of \(S^{n-1}\) where the ray in \(\mathbb{R}^n\) starting at \(f(x)\) and passing through 
\(x\) leaves \(D^n\). Continuity of \(r\) is intuitively clear since small changes from \( h(x) \) change \( x \) also a little since \( f(x) \) is cont.. 
However we proved already that no such retraction can exist, reductio ad absurdum.
The situation is pictorially represented below (here \( f = h \)).


Define the degree of maps of the sphere and state some properties

Let \(f: S^n\to S^n\) be a continuous map. Then \(f\) induces a homomorphism \(f_*\colon H_n\left(S^n\right) \to H_n\left(S^n\right)\), 
Considering the fact that \(H_n\left(S^n\right)\cong\mathbb{Z}\), we see that \(f_*\) must be of the form \(f_*\colon x\mapsto\alpha x\) 
for some fixed \(\alpha\in\mathbb{Z}\). This \(\alpha\) is then called the degree of \(f\).
Note that \( \text{deg} f \) is independent of the choice of generator, suppose \( \sigma \in H_n(S^n) \) is a generator,
then the only other is \( - \sigma \) and \( f_{\ast}(-\sigma) = -f_{\ast}(\sigma) = - (\text{deg}\ f)\sigma = (\text{deg}\ f)(- \alpha) \).

Proposition 5.3.6. 
Let \(f, g: S^n \to S^n\) be two continuous maps. Then it holds:
1. \(\deg \text{id} = 1\), and \(\deg (f \circ g) = \deg f \cdot \deg g\).
2. \(\deg f = 0\) if \(f\) is not surjective.
3. If \(f \simeq g\) are homotopic, then \(\deg f = \deg g\).
4. If \(f\) is a reflection with respect to a hyperplane (through the origin), then \(\deg f = -1\).
5. The antipodal map \(-\text{id}_{S^n}: S^n \to S^n\), \(x \mapsto -x\), has degree \((-1)^{n+1}\).
6. If \(f\) has no fixed points, then \(\deg f = (-1)^{n+1}\).

If \( G \) is a non-trivial group acting freely on \( S^n \) with even \( n \), then \( G \simeq \mathbb{Z}/2\mathbb{Z} \)
Proof:
Since homeomorphisms have degree \(\pm 1\), an action of a group \(G\) on \(S^n\) determines a degree function \(d: G \to \{\pm 1\}\). 
This is a homomorphism since \(\deg (f \circ g) = \deg f \deg g\). 
If the action is free, \(d\) only \( e \in G \) has fixed points (in fact it is \( \text{id}_{S^n} \)) 
and all nontrivial element of \(G\) must then have \((-1)^{n+1}\) by the last property from above. 
Thus when \(n\) is even, \(d\) has trivial kernel (elements sent to \( 1 \)), so \(G \subset \mathbb{Z}_2\).


State and prove the Hairy Ball Theorem

Theorem: The sphere \(S^n\) admits a tangent nowhere-vanishing vector field if and only if \(n\) is odd.
Assume there exists a non-vanishing vector field \( v : S^n \to \mathbb{R}^n \) for \( n \) even.
Define the homotopy \( H(x, t) \coloneqq \cost(t\pi)x + \sin(t\pi)v(x)(\frac{1}{\|v(x)\|}) \) 
(here we used the non vanishing part) between the identity map \( \text{id} \) 
and the antipodal map \( - \text{id} \). The points for each \( t \), \( H(x, t) \)
lies in on the unit circle contained in the plane spanned by \( x, v(x) \). 
Since homotopic maps must have the same degree, so we should have 
\( \text{deg}(-\text{id}) = (-1)^{n+1} = 1 = \text{deg}(\text{id}) \), which is impossible, since \( n+1 \) is odd.


It is left to show that there always exist some nowhere-vanishing vector field for \( n = 2k - 1 \) odd.
We can define \(v(x_1, x_2, \dots, x_{2k-1}, x_{2k}) = (-x_2, x_1, \dots, -x_{2k}, x_{2k-1})\) (which we can do since there is 
Then \(v(x)\) is orthogonal to \(x\), so \(v\) is a tangent vector field on \(S^n\), and \(|v(x)| = 1\) for all \(x \in S^n\). 


Moreover \( S^n \) has a nowhere vanishing vector field iff. \( n \) is odd.

State and prove the Jordan-Curve Theorem

Some useful Lemmas are:
Let \( 0 \leq r \leq n \) and \( D \subseteq S^n \) a subset homeomorphic to a closed ball \( D^r \subseteq \mathbb{R}^r \).
Then \( \tilde{H}_k(S^n\setminus D) = 0 \) for all \( k \in \mathbb{Z} \).

From this one can deduce
Let \( 0 \leq r < n \) and \( S \subseteq S^n \) a subset homeomorphic to the \( r \)-dimensional sphere \( S^r \)
then \( \tilde{H}_k(S^n \ S) = \mathbb{Z} \) for \( k = n - r - 1 \) and \( 0 \) otherwise.

Finally:
Let \( n \geq 2 \) and \( S \subseteq \mathbb{R}^n \) a subset homeomorphic to \( S^{n-1} \).
Then \( \mathbb{R}^n \setminus S \) has two connected components. A bounded one and the other unbounded.
Proof. 
Think of \(\mathbb{R}^n \subset S^n\) by adding one point \(p\) at infinity. By the previous Lemma, \(\tilde{H}_0(S^n) \cong \mathbb{Z}\) 
and so \(H_0(S^n) \cong \mathbb{Z}^2\), implying two path-connected components in \(S^n \setminus S\).
Since \(S^n \setminus S\) is open (\(S\) is compact in the Hausdorff space \(S^n\), and thus closed), 
it is also locally path-connected, and thus the path-connected components coincide with the connected components.

The point \(p\) lies in exactly one of them, and removing \(p\) does not disconnect it (because \(n \geq 2\)), 
so \(\mathbb{R}^n \setminus S\) still has two connected components, and the unbounded one is the one where \(p\) was removed.

Define the relative homology of a pair \( (X, A) \)

Let \((X, A)\) be a pair, i.e. \(X\) is a topological space and \(A \subset X\). 
The inclusion \(\iota: A \to X\) gives a chain map \(\iota_\sharp: C_\bullet(A) \to C_\bullet(X)\). 
This chain map is injective, i.e. \(C_\bullet(A)\) is a chain subcomplex of \(C_\bullet(X)\), 
because \(\iota_\sharp\) identifies \(C_n(A)\) with the subgroup of \(C_n(X)\) generated by simplices \(\Delta^n \to X\) with image in \(A\). 
Since this inclusion is canonical, we will omit \(\iota_\sharp\) and simply write \(C_\bullet(A) \subset C_\bullet(X)\). 

Definition: 
The relative (singular) chain complex of the pair \((X, A)\) is \(C_\bullet(X, A) := C_\bullet(X) / C_\bullet(A)\).
The relative singular homology groups of \((X, A)\) are just the homology groups of the relative complex: \(H_n(X, A) := H_n(C_\bullet(X, A))\).

In particular, for \(A = \emptyset\) we obviously have \(C_\bullet(X, \emptyset) = C_\bullet(X)\) and \(H_n(X, \emptyset) = H_n(X)\).
Note that the groups of relative chains \(C_n(X, A) = C_n(X) / C_n(A)\) are generated 
by the singular simplices \(\Delta^n \to X\) whose image is not contained in \(X\). 
The morphisms \(\pi_n: C_n(X) \to C_n(X, A)\) consist simply of "forgetting" the simplices with image in \(A\).


This means for the boundary operator \( \partial \): that if \(c \in C_n(X, A)\), we first interpret \(c\) as a chain in \(X\), 
then compute its usual boundary in \(C_{n-1}(X)\), 
and finally discard all simplices in the boundary with image contained in \(A\). Hence the name "boundary relative to \(A\)".

In particular \(\partial c = 0 \in C_{n-1}(X, A)\) if all components of the boundary of \(c\) have image contained in \(A\), 
although some simplices of \(c\) may not be contained in \(A\). 
For example: a path joining two points of \(A\) but not entirely contained in \(A\) would be a relative 1-cycle. 
And the end of a path starting in \(A\) would be a relative 0-boundary.

To be more precise, denote by 
\(Z_n(X, A) = \{c \in C_n(X) \mid dc \in C_{n-1}(A)\} \subseteq C_n(X),\)
the relative \(n\)-cycles, and by
\(B_n(X, A) = \{c \in C_n(X) \mid \exists c' \in C_n(A) \text{ such that } c - c' \in B_n(X)\} \subseteq C_n(X)\)
the relative \(n\)-boundaries.

Then \( H_n(X, A) = \frac{Z_n(X, A)/C_n(A)}{B_n(X, A)/C_n(A)} \simeq \frac{Z_n(X, A)}{B_n(X, A)} \)

Define a map of pairs and state some essential properties

Let \((X, A)\) and \((Y, B)\) be pairs.
A map of pairs \(f: (X, A) \to (Y, B)\) is a continuous map \(f: X \to Y\) with \(f(A) \subseteq B\).

Lemma 6.1.3. 
Let \(f : (X, A) \to (Y, B)\) be a map of pairs. Then
1. There are natural group homomorphisms \(f_*: H_n(X, A) \to H_n(Y, B)\), defined as \(f_*([c]) = [f_\sharp(c)]\) for any relative \(n\)-cycle \(c\).
2. If \(g: (Y, B) \to (Z, C)\) is another map of pairs, then \((g \circ f)_* = g_* \circ f_*: H_n(X, A) \to H_n(Z, C)\) for all \(n \in \mathbb{Z}\).

Definition 6.1.4. 
Two maps of pairs \(f, g: (X, A) \to (Y, B)\) are homotopic (as maps of pairs) (also written \(f \simeq g\)) 
if there is a homotopy \(H: X \times [0, 1] \to Y\) between \(f\) and \(g\), such that \(H(a, t) \in B\), 
for every \(a \in A\), \(t \in [0, 1]\), i.e. \(H\) is a map of pairs \((X \times [0, 1], A \times [0, 1]) \to (Y, B)\).

If \( f \simeq g : (X, A) \to (Y, B) \) then \( f_{\ast} = g_{\ast} : H_n(X, A) \to H_n(Y, B)\) for all \( n \in \mathbb{Z} \).
\( f : (X, A) \to (Y, B) \)
The proof is based on the same insight as the original one for regular chain maps via the prism operator. 

Define the long exact sequence of relative homology

Theorem 6.1.8:
Let \((X, A)\) be a pair. Then there are natural connecting homomorphisms \(\Phi_n: H_n(X, A) \to H_{n-1}(A)\) fitting into a long exact sequence
\(\dots \longrightarrow H_n(A) \xrightarrow{\iota_*} H_n(X) \xrightarrow{\pi_*} H_n(X, A) \xrightarrow{\partial_n} H_{n-1}(A) \longrightarrow \dots\)
the long exact sequence of relative homology.
Proof:
This is a direct consequence of the SES
\( 0 \longrightarrow C_n(A) \longrightarrow C_n(X) \longrightarrow C_n(X, A) \to 0\)
and the naturality of the connecting homomorphisms.

The morphisms \(\iota_*\) and \(\pi_*\) in (6.2) are respectively induced by the inclusion \(\iota: A \to X\) and 
discarding the simplices with image in \(A\). The connecting homomorphisms \(\phi_n: H_n(X, A) \to H_{n-1}(A)\) are determined as follows: 
a relative homology class \([c] \in H_n(X, A)\) is represented by a relative cycle, i.e. \(c \in C_n(X)\) with \(dc \in C_{n-1}(A)\). 
From the general construction of connecting homomorphisms we deduce \(\phi_n([c]) = [dc] \in H_{n-1}(A)\). 
For this reason these connecting homomorphisms are sometimes also written \(\partial\), as the boundary maps.

Suppose \( A \subseteq X\) is a deformation retract of \( X \). Then \( H_n(X, A) = 0, \forall x \in \mathbb{Z}\).
Proof: \( \iota : A \to X\) induces an isomorphism \( H_n(A) \simeq H_n(X) \), the LES of rel. homol., \( H_n(X, A) = 0 \).

Note that the reduced relative homology bears no benefit since it creates a chain which is equivalent to the following
\( \dots \tilde{H}_n(A) \longrightarrow \tilde{H}_n(X) \longrightarrow H_n(X, A) \longrightarrow \tilde{H}_{n-1}(A) \longrightarrow \dots \),
so we would obtain no new information. Moreover \( \tilde{H}_n(X) \simeq H_n(X, \{p\}) \).


State and prove the excision theorem

Theorem 6.2.1 (Excision theorem). Let \(X\) be a topological space and \(Z \subset U \subset X\) subspaces such that \(\overline{Z} \subset int U\). 
Then the inclusion of pairs \((X \setminus Z, U \setminus Z) \to (X, U)\) induces isomorphisms 
\(H_n(X \setminus Z, U \setminus Z) \cong H_n(X, U)\)

for all \(n \in \mathbb{Z}\). \qquad (6.5)

Proof. Set \(V = X \setminus Z\), so that (6.5) becomes
\(H_n(V, U \cap V) \cong H_n(X, U).\)


Since \(int V = X \setminus \overline{Z}\), the condition \(\overline{Z} \subset int U\) is 
equivalent to \(int V \cup int U = X\), hence \(\mathcal{U} := \{U, V\}\) is a cover of \(X\). 
Denote \(C_\bullet(U + V) := C^{U}_\bullet(X)\) 
(as we did in Section 5.2 when constructing the Mayer-Vietoris sequence) 
and \(\iota: C_\bullet(U + V) \to C_\bullet(X)\) the natural inclusion. 
The Theorem of small chains (Theorem 4.2.5) gives a chain map \(p: C_\bullet(X) \to C_\bullet(U + V)\) 
such that \(p \circ \iota = \text{id}_{C_\bullet(U + V)}\) and \(\iota \circ p \simeq \text{id}_{C_\bullet(X)}\). 
In particular \(p\) restricts to the identity on the subcomplex \(C_\bullet(U) \subset C_\bullet(U + V) \subset C_\bullet(X)\). 
This means that we can complete the following commutative diagram 

(Insert diagram from page 44 Theorem 6.2.1)

Moreover, since the chain homotopy \(D: C_\bullet(X) \to C_\bullet(X)\) between \(\iota \circ p\) and \(\text{id}_{C_\bullet(X)}\) 
can be chosen to be zero on \(C_\bullet(U)\) due to the way the Theorem of small chains handles \( \mathcal{U} \)-small simplexes. 
Then \(D\) induces also a chain homotopy 
\(\overline{D}: C_\bullet(X, U) \to C_\bullet(X, U)\) between \(\overline{\iota \circ p}\) and \(\text{id}_{C_\bullet(X, U)}\), 
and thus \(\iota\) induces isomorphisms

\(H_n(C_\bullet(U + V) / C_\bullet(U)) \cong H_n(X, U). \qquad (6.6)\)

Note that the complex \(C_\bullet(U + V) / C_\bullet(U)\) consists of the free groups generated by the 
simplices with image in either \(U\) or \(V\) but not in \(U\), 
i.e. the inclusion \(C_\bullet(V) \hookrightarrow C_\bullet(U + V)\) induces an isomorphism of chain complexes

\(C_\bullet(V, U \cap V) = C_\bullet(V) / C_\bullet(U \cap V) \cong C_\bullet(U + V) / C_\bullet(U)\)
which in turns gives isomorphisms of homology groups
\(H_n(V, U \cap V) \cong H_n(C_\bullet(U + V) / C_\bullet(U)). \qquad (6.7)\)

The result then follows combining (6.6) and (6.7).


Define good pairs and state the isomorphism in homology for good pairs

Definition: 
A pair \((X, A)\) is a good pair if \(A \neq \emptyset\) is closed and a strong deformation retract of a neighbourhood, 
i.e. there is \(A \subseteq int U \subseteq U \subseteq X\) and a continuous map \(r: U \to A\) such that 
\(U \xrightarrow{r} A \hookrightarrow U\) is homotopic to \(\text{id}_U\) with an homotopy fixing every point of \(A\).

Theorem 6.2.4. 
Let \((X, A)\) be a good pair. The quotient map \(q: X \to X/A\) is a map of pairs \((X, A) \to (X/A, A/A)\) and induces isomorphisms 
\(H_n(X, A) \xrightarrow{q_*} H_n(X/A, A/A) \cong \tilde{H}_n(X/A).\)

The second isomorphism follows from \( A/A \) being just like a point in \( X / A \).

For the first isomorphism, suppose \(A\) is a strong deformation retract of a neighbourhood \(A \subset U \subset X\). 
In particular \(H_n(U, A) = 0\) for every \(n \in \mathbb{Z}\) (Corollary 6.1.10). 
The long exact sequence of relative homology of the pair \((X, U, A)\) gives then isomorphisms
\(\dots \longrightarrow H_n(U, A) = 0 \longrightarrow H_n(X, A) \xrightarrow{\hat{\imath}_*} H_n(X, U) \longrightarrow H_{n-1}(U, A) = 0 \longrightarrow \dots.\)
Also the excision theorem gives the isomorphism \( H_n(X\setminusA, U\setminusA) \simeq H_n(X, U)\)
Consider then the diagram

(insert diagram from page 45)

The first row are isomorphisms by the previous considerations, the lower ones are as well by noting that \( (X / A, A / A) \)
is a good pair and repeating the argument of the LES. Lastly we can readily see that \( q^\ast \) is an isomorphism for the last column
and thus by commutativity all the others are as well.


Give some examples of good pairs

Example 6.2.5. 
Consider the good pair \((D^n, S^{n-1})\) of a closed disk in \(\mathbb{R}^n\) and its boundary. 
Since \(D^n\) is contractible, the long exact sequence of relative (reduced) homology gives 

\(\tilde{H}_k(D^n, S^{n-1}) \cong \tilde{H}_{k-1}(S^{n-1}),\)
which is \(\mathbb{Z}\) for \(k = n\) and 0 otherwise.

On the other hand, if we collapse the whole boundary of \(D^n\) into a point, we obtain the \(n\)-dimensional sphere \(S^n\), 
whose reduced homology groups \(\tilde{H}_k(S^n)\) are also \(\mathbb{Z}\) for \(k = n\) and 0 otherwise.

Let \((X_i, p_i)_{i \in I}\) be a family of pointed spaces and assume that each \((X_i, p_i)\) is a good pair. Then the 
inclusions \(\iota_i: X_i \hookrightarrow \bigvee_{i \in I} X_i\) give a natural isomorphism

\(\bigoplus_{i \in I} \tilde{H}_n(X_i) \xrightarrow{\hat{\oplus} \iota_{i*}} \tilde{H}_n \left( \bigvee_{i \in I} X_i \right).\)

Proof. 
Set \(X = \bigsqcup_{i \in I} X_i\) and \(A = \{p_i\}_{i \in I} \subset X\), so that by definition \(\bigvee_{i \in I} X_i = X/A\).
If each \(p_i\) is a strong deformation retract of a neighbourhood \(p_i \in U_i \subset X_i\), 
then \(A \subset \bigsqcup_{i \in I} U_i \subset X\) is also a strong deformation retract, hence \((X, A)\) is a good pair.

Note that in general for path disconnected componentes \( \cup X_i = X \), \( C\bullet(X) = \bigoplus_{i \in I} C_\bullet(X_i) \). 
If furthermore \( A \subseteq X \) then \( C_\bullet(A) = \bigoplus_{i \in I} C_\bullet(A \cap X_i) \). 
Thus \( C_\bullet(X, A) = \bigoplus_{i \in I} C_\bullet(X_i, A \cap X_i) \) and 
\( H_n(X, A) = \bigoplus_{i \in I} H_n(X_i, A \cap X_i) \text{ for all } n \in \mathbb{Z} \).
Applying this to the previous then gives \( \tilde{H}_n(\bigvee_{i \in I} X_i) = H_n(X, A) = \bigoplus_{i \in I} H_n(X_i, p_i) \).



Define local homology and state a fundamental result for manifolds

Let \( X \) be a top. space and \( p \in X \) a (closed) point. The local homology groups are \( H_n(X, X \setminus \{p\}\).
By the excision theorem \( H_n(X, X\setminus \{p\}) \simeq H_n(U, U \setminus \{p\}) \) for any neighborhood of \( p \).

Theorem 6.3.3. 
Let \(M\) be a topological manifold (possibly with boundary) of dimension \(n\). Then 
1. For any interior point \(p \in M\) there are isomorphisms \(H_n(M, M \setminus \{p\}) \cong \mathbb{Z}\) and \(H_k(M, M \setminus \{p\}) = 0\) for every \(k \neq n\).
2. For any boundary point \(p \in M\) we have \(H_k(M, M \setminus \{p\}) = 0\) for every \(k \in \mathbb{Z}\).

In particular, if \(U \subset \mathbb{R}^n\) and \(V \subset \mathbb{R}^m\) are homeomorphic non empty open subsets, then \(n = m\).
