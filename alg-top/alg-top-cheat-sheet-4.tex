Define adjunctions and cells

Let \( X, Y \) be top. spaces and \( A \subseteq X \) a subset with cont. \( \phi : A \to Y \).
The adjunction space \( X \cup_{\phi} Y\) is the quotient space of \( X \sqcup Y \) by the equiv. relation \( a \sim \phi(a) \)
for all \( a \in A \).

The cases for identified points \( x, x' \in X \), \( y, y' \in Y \)
are
\( [y] = [y'] \) iff. \( y = y' \)
\( [x] = [y] \) iff. \( \phi(x) = y \), \( x \in A \)
\( [x] = [x'] \) iff. either \( x = x' \) or \( \phi(x) = \phi(x') \).

One calls \( \phi \) the gluing or attaching map and topology of \( X \cup_{\phi} Y \) is the usual quotient topology,
that is \( U \subseteq X \cup_{\phi} Y \) is open iff. \( \phi^{-1}(U) \) is (\( \phi \) becomes an open map).

The particular case of \( X = D^n \subset \mathbb{R}^n \) and \( A = S^{n-1} \), we refer to as the adjunction of a \( n \)-cell,
where an \( n \)-cell is meant to be \( D^n \setminus S^{n-1} \).


State some properties of the adjunction space and describe the construction to prove Hausdorfness

Let \( Y_\phi = D^n \cup_\phi Y \) then

1. \( \iota : Y \to Y_\phi \) is a homeomorphism onto its image \( \iota(Y) \) .
2. The restriction \( \Phi : D^n \to Y_{\phi} \) to the \( n \)-cell is a homeomorphism onto \( Y_\phi \setminus \iota(Y) \).
3. If \( Y \) is Hausdorff, so is \( Y_\phi \).
4. If \( Y \) is compact so is \( Y_\phi \).

For 3. one first needs to construct open neighborhoods of points in \( Y_\phi \).
If \( p \in e^n \subseteq Y_\phi \), a standard open ball \( B_\varepsilon(p) \subset \mathbb{R}^n \) suffices.
If \( p \in \iota(Y) \) this is more tricky, take some open subset \( V \subset Y \), then \( \iota(V) \) is open,
while \( \Phi^{-1}(V) = \phi^{-1}(V) \subseteq S^{n-1} \) is not open in \( D^n \) (only in the subspace of its boundary).
To solve this, enlarge \( V \) by

\( U_{\varepsilon} = V \cup \Phi((1 - \varepsilon, 1) \times \phi^{-1}(V)) \subseteq \Phi(e^n) \)
and note that \( \Phi(e^n) = ((1 - \varepsilon, 1] \times \phi^{-1}(V)) \cup_\phi V\),
where we used polar coordinates \( (r, \theta) \) to parameterize \( D^n \setminus \{0\} \).
Pictorially the situation is described below.

(Insert hand drawn picture here)


Then it still holds \( \iota^{-1}(U_{\varepsilon}) = V \) and also \( \Phi^{-1}(U_{\varepsilon}) \simeq (1 - \varepsilon, 1] \times \phi^{-1}(V) \subset D^n \)
is open.

Hausdorfness for \( x, y \in e^n \), or \( x \in e^n, y \in Y \) is straightforward, for \( x, y \in Y \) however we now can leverage the fact that \( Y \)
is Hausdorff so there are disjoint open neighborhoods \( V, W \) and their corresponding \( V_\varepsilon, W_\varepsilon \) are still disjoint and open in \( Y_\varphi \).


Describe the most basic case of the homology of a CW-complex aka attaching a cell to some top. space

Let \( Y_\phi = D^n \cup_\phi Y \) then
1. \( (Y_\varphi, Y) \) is a good pair, i.e. \( Y \subseteq Y_\varphi \) is a strong deformation retract of some neighborhood \( Y \subseteq U \subseteq Y_\varphi \).
2. It holds \( \tilde{H}_k(Y_\varphi) = \tilde{H}_k(Y) \) for all \( k \neq n-1, n \) while for \( k = n-1, n \) there is an exact sequence

\[0 \rightarrow \tilde{H}_n(Y) \xrightarrow{i_*} \tilde{H}_n(Y_{\varphi}) \xrightarrow{\partial_*} \tilde{H}_{n-1}(S^{n-1}) \xrightarrow{j_*} \tilde{H}_{n-1}(Y) \xrightarrow{i_*} \tilde{H}_{n-1}(Y_{\varphi}) \rightarrow 0. \hspace{1cm} (7.2)\]

Proof. 1. For any \(\varepsilon \in (0,1]\), the subset \(U_\varepsilon = ((1-\varepsilon, 1] \times S^{n-1}) \cup_{\varphi} Y\), constructed 
in the proof for Hausdorfness of adjunction spaces, is an open neighborhood of \(Y \subset Y_{\varphi}\) (in particular it is the "thickening" of \( \phi^{-1}(Y) \)).

Consider now the homotopy \(H: U \times [0,1] \rightarrow U\) defined by

\[
H([x],t) = \begin{cases}
[x] & \text{if } x \in Y \subset Y_{\varphi}, \\
[x + t(\frac{x}{|x|} - x)] = [(s + t(1-s), \theta)] & \text{if } x = (s, \theta) \in (1-\varepsilon, 1] \times S^{n-1}.
\end{cases} \hspace{1cm} (7.3)
\]

Pictorially this is just retracting \( U_\varepsilon \) along the radii of the origin back unto the boundary of \( D^n \).

Note that it is well-defined (and hence continuous), since for \([x] \in \Phi((1-\varepsilon, 1] \times S^{n-1}) \cap Y\) 
it holds \(|x| = 1\) (thought as a point in \(D^n\), and thus \( H([x], t) = [x] \) for every \( t \in [0, 1] \).

It is immediate to check that \( H([x], 1) \in Y \) for every \( x \in U \), thus \( r([x]) \coloneqq H([x], 1) \) defines a map \( r: U \rightarrow Y \). 
Since \( H([x], 0) = [x] \) for all \( [x] \), \( H \) is a homotopy between \( \text{id}_U \) and \( r \). 
Moreover, since \( H([x], t) = [x] \) for all \( [x] \in Y \) and \( t \in [0, 1] \), and in particular \( r(x) = x \), \(r\) is a 
strong deformation retract from \( U \) to \( Y \), as wanted.

2. The second can either be shown via Mayer-Vietoris with the choice \( U = U_\varepsilon, V = e^n \) or the long exact sequence 
of relative homology for the pair \( (Y, Y\varphi) \).

This allows to derive the Homology of the \( k \)-th skeleton of a CW-complex, 
if the homology of the \( k-1 \)-th is already known, by setting \( Y = X^(k-1) \) and iterating the theorem by first attaching
a cell computing the homology and then doing this again until all cells have been attached.


Define a CW-complex

Assume that a topological space \( X \) is a disjoint union of cells:
\[ X = \bigcup_{e \in E} e. \] 
For each \( k \geq 0 \) , the \(k\)-skeleton \(X^{(k)}\) of \(X\) is defined by 
\[ X^{(k)} = \bigcup \{e \in E: \dim(e) \leq k\}. \]
Of course, \(X^{(0)} \subset X^{(1)} \subset X^{(2)} \subset \dotsb\) and \(X = \bigcup_{k \geq 0} X^{(k)}\).

A CW complex/cell complex is an ordered triple \( (X, E, \Phi) \), where \( X \) is a Hausdorff space, \( E \) is a family of cells in \( X \), 
and \( \Phi = \{\Phi_e : e \in E\} \) is a family of maps, such that 

(1) \( X = \bigcup_{e \in E} e \) (disjoint union); 
(2) for each \( k \)-cell \( e \in E \), the map \( \Phi_e : (D^k, S^{k-1}) \to (e \cup X^{(k-1)}, X^{(k-1)}) \) is a relative homeomorphism; 
(3) if \( e \in E \), then its closure \( \overline{e} \) is contained in a finite union of cells in \( E \); 
(4) \( X \) has the weak topology determined by \( \{\overline{e} : e \in E\} \). 

3. is also called closure finiteness, which combined with 4 represents the CW (closure-weak) prefix.
If \( (X, E, \Phi) \) is a CW complex, then \( X \) is called a CW space, \( (E, \Phi) \) is called a CW decomposition of \( X \) , 
and \( \Phi_e \) is called the characteristic map of \( e \).
This definition has the benefit of skipping a few proofs, necessary in the lecture and is actually more general,
as it allows to call spaces, which we just found to be homeomorphic to CW complexes to be called CW complexes themselves.
It reduces to the lecture definition under a certain choice for the \( \Phi \), namely

(i) \( X^0 \) is discrete (aka points); 
(ii) for each \(n > 0\), there is a (possibly empty) index set \(A_n\), and a family of continuous functions 
\(\{f_{\alpha}^{n-1} : S^{n-1} \to X^{n-1} \mid \alpha \in A_n\}\) so that
\[
X^n = \left( \coprod_{\alpha} D^n \right) \coprod_{\varphi} X^{n-1}, 
\]
where \(\varphi = \coprod \varphi_{\alpha}^{n-1}\);
If \(\Phi_\alpha\) denotes the (usual) composite

\( D^n \xrightarrow{\cong} D^n_\alpha \hookrightarrow \left( \coprod_\alpha D^n_\alpha \right) \coprod X^{n-1} \xrightarrow{} \left( \coprod_\alpha D^n_\alpha \right) \coprod_\varphi X^{n-1}, \)

then \((X, E, \Phi)\) is a CW complex, where
\[ E = X^0 \cup \bigcup_{n \geq 1} \{ \Phi_\alpha (D^n - S^{n-1}) : \alpha \in A_n \} \]
and
\[ \Phi = \{ \text{constant maps to } X^0 \} \cup \bigcup_{n \geq 1} \{ \Phi_\alpha : \alpha \in A_n \}. \]

Further terminology:
Closed cells refer to \( \Phi_\alpha(D^n_\alpha) \subseteq X \).
Let \( X \) be a cell complex, a subspace \( A \subseteq X \) is a subcomplex if it is the union of cells of \( X \) and 
the closure of each cell is also contained in \( A \) (can also be put as \( \text{cl}(e) = \text{im} \phi_e \subset A \).
The union and intersection of subcomplexes is again a subcomplex.
The cell complex is finite if \( E \) is. The cell of largest dimension in \( E \) is the dimension of \( X \).

State some properties openess/closedness 

Elaborate on the topology of the different parts of a CW complex

Let \( X, E \) be a CW complex, then

\( X \) is Hausdorff (and even normal)
If \( X \) is finite then \( X \) is compact.

For some fixed \( n > 0 \), let \( E' \) be a family of \( n \)-cells in \( E \)
and let \( |E'| \) be the corresponding subcomplex.

1. \( X' = |E'| \cup X^{n-1} \) is closed in \( X \)
2. every \( n \)-skeleton \( X^{(n)} \) is closed in \( X \)
3. every \( n \)-cell \( e^n \) is open in \( X^{(n)} \)
4. \( X^{(n)} \setminus X^{(n-1)} \) is an open subset of \( X^{(n)} \).

Let \( X \) be a cell complex \( K \subseteq X \) a compact subset. 
Then \( K \) is contained in a finite subcomplex \( A \subseteq X \).
In particular a closed cell of \( X \) intersects finitely many cells.

Describe roughly the idea to verify that a CW pair is a good pair.

One again needs to construct open subsets in \( X \), this in so far not trivial
since this means that for all \( X^{(k)} \) the set needs to be open.
Take an open subset \(U^n\) of some \(X^n\): 
for each \(\alpha \in \bigcup_{k \geq n+1} I_k\) fix an \(\varepsilon_\alpha \in (0,1)\) and define inductively the open subset

\[ U_\varepsilon^k = \left( \bigcup_{\alpha \in I_k} (1-\varepsilon_\alpha, 1] \times \varphi_\alpha^{-1} (U_\varepsilon^{k-1}) \right) \cup_{(\varphi_\alpha)} U_\varepsilon^{k-1} \subseteq X^k \subseteq X, \hspace{1cm} (7.4) \]

(where again we are describing the points in each ball \(D^k_\alpha\) with polar coordinates). 
Then \(U_\varepsilon = \bigcup_{k \geq n+1} U_\varepsilon^k \subseteq X\) is an open subset, since \(U_\varepsilon \cap X^k = U_\epsilon^k \subseteq X^k\) is open for 
each \(k > n\).

To include \( A \) in our open set, we start inductively with \( U^{(0)} = A^{(0)} \) and then \( U^{k} = A^k \cup U_\varepsilon^k \) (of course
\(U_\varepsilon^{k-1}\) in the expression of \( U_\varepsilon^k \) needs to be replaced by \( U^{(k-1)} \).

We can then verify for each \( k \) that we can find some strong deformation retract \( r_k : U^k \to A^k \) with homotopy \( H_k : U^k \times [0, 1] \to U^k \)
to show that \( (X, A) \) is a good pair. To this end set \( r_0 = \text{id}_{U^0} \) and \( H_0(x, t) = x \). 
Inductively we set
\( r_k(\Phi_\alpha(s, \theta)) = r_{k-1}(\Phi_\alpha(1, \theta)) = r_{k-1}(\phi_\alpha(\theta)) \)
for \( (s, \theta) \in (1 - \varepsilon, 1] \times \varphi_\alpha^{-1}(U^{k-1}) \) and \( r(x) = x \) if \( x \in A^k \)
The homotopy is a bit more complicated mostly but the basic gist is \( H_k(x, t) = x \) for \( x \in A^k \) and \( H_{k-1}(r_k(x), t) \) for \( t > c(k) \)
and join for \( 0 < t \leq c(k) \) with an interpolation via \( \Phi \).
So if \( x \in U^k \setminus A^k \) we first project along the radii onto \( U^{k-1} \) and then retract via \( r_k \).

State and prove the relative homology of a cell complex \( X^{(n)} \) with respect to its sceleton \( X^{(n-1)} \).

\[
H_k(X^n, X^{n-1}) \cong \bigoplus_{\alpha \in I_n} H_k(D^n_\alpha, \partial D^n_\alpha) 
\cong \bigoplus_{\alpha \in I_n} \tilde{H}_{k-1} (\partial D^n_\alpha) 
\cong 
\begin{cases}
\bigoplus_{\alpha \in I_n} \mathbb{Z} e_\alpha & \text{if } k = n, \\
0 & \text{otherwise.}
\end{cases}
\quad (7.5)
\]

Proof. 
Each characteristic map \(\Phi_\alpha : (D^n_\alpha, \partial D^n_\alpha) \to (X^n, X^{n-1})\) is a map of pairs, 
hence there are induced group homomorphisms \((\Phi_\alpha)_* : H_k(D^n_\alpha, \partial D^n_\alpha) \to H_k(X^n, X^{n-1})\) which can be combined into

\[
\Psi = \bigoplus (\Phi_\alpha)_* : \bigoplus_{\alpha \in I_n} H_k(D^n_\alpha, \partial D^n_\alpha) \longrightarrow H_k(X^n, X^{n-1}). 
\quad (7.6)
\]

Further we know that \( X^n, X^{n-1} \) are good pairs, so we have \( H_k(X^n, X^{n-1}) \simeq \tilde{H}_k(X^n / X^{n-1} \).
Of course \((D^n_\alpha, \partial D^n_\alpha)\) are as well with \( H_k(D^n_\alpha, \partial D^n_\alpha) \simeq \tilde{H}_k(S^n_\alpha) \).
Since the attached \( \{e^n_\alpha\} \) are unaffected by taking the quotient except that they're all glued now just to a point \( p = X^{n-1}/X^{n-1} \),
we can express \( X^n/X^{n-1} = \bigvee_\alpha S^n_\alpha \), thus
\( H_k(X^n, X^{n-1}) \simeq \tilde{H}_k(\bigvee_\alpha S^n_\alpha) = \oplus_\alpha \tilde{H}_k(S^n_\alpha) \simeq \oplus_\alpha H_k(D^n_\alpha, \partial D^n_\alpha) \).


State and prove the expression of the homology of CW complex for higher dimensions than its own and how subcomplexes 
determine the Homology below.

1. \(H_k(X^n) = 0\) for any \(k > n\).
2. The map \(H_k(X^n) \to H_k(X)\) induced by the inclusion \(X^n \hookrightarrow X\) is an isomorphism for \(k < n\) and surjective for \(k = n\). 

Proof. 
From the long exact sequence of relative homology of the pair \((X^n, X^{n-1})\) we obtain exact sequences 
\[
H_{k+1}(X^n, X^{n-1}) \to H_k(X^{n-1}) \to H_k(X^n) \to H_k(X^n, X^{n-1})
\]

Since we determined already \(H_{k}(X^n, X^{n-1})\) for all \( k \), (\( 0 \) except for \( n = k \) then \( \oplus_{e_\alpha} \mathbb{Z}e_\alpha \)), so
we have that \(H_k(X^{n-1}) \rightarrow H_k(X^n)\) is surjective for \(k \neq n\), and injective for \(k \neq n-1\), thus in particular an isomorphism for 
\(k \neq n, n-1\). So there is a sequence of homomorphisms, which are almost all isomorphisms

\( H_k(X^0) \xrightarrow{\cong} H_k(X^1) \xrightarrow{\cong} \dotsb \xrightarrow{\cong} H_k(X^{k-1}) \hookrightarrow H_k(X^k) \twoheadrightarrow H_k(X^{k+1}) \xrightarrow{\cong} \dotsb \hspace{1cm} (7.7) \)
Since \(H_k(X^0) = 0\) for all \(k > 0\), \(H_k(X^n) = 0\) for all \(n < k\), as wanted.

The second part then follows directly for \( \text{dim }X < \infty \) the special case of infinity is a more intricate. 


Define the cellular chain complex

The cellular chain complex of the cell complex \(X\) is defined by 
* the abelian groups \(C^{cell}_n(X) := H_n(X^n, X^{n-1}) (\cong \bigoplus_{\alpha \in I_n} \mathbb{Z} e_\alpha)\).
* and the differentials \(d_n \colon C^{cell}_n(X) \to C^{cell}_{n-1}(X)\) obtained as the composition
\[ d_n \colon H_n(X^n, X^{n-1}) \xrightarrow{\partial_n} H_{n-1}(X^{n-1}) \xrightarrow{\pi_{n-1}} H_{n-1}(X^{n-1}, X^{n-2}), \]
where \(\partial_n\) is the connecting homomorphism in relative homology of the pair \((X^n, X^{n-1})\), and \(\pi_{n-1}\) 
is the map obtained from the pair \((X^{n-1}, X^{n-2})\).

To check that \(C^{cell}_\bullet(X)\) is an actual complex, we compute \(d_{n-1} \circ d_n = \pi_{n-2} \circ \partial_{n-1} \circ \pi_{n-1} \circ \partial_n = 0\), 
where we have used \(\partial_{n-1} \circ \pi_{n-1} = 0\) by exactness of the long exact sequence of the pair \((X^{n-1}, X^{n-2})\). 


Prove the equivalence of cellular and singular homology.

The homology groups of \((C^{cell}_\bullet(X), d)\) are isomorphic to the singular homology groups of \(X: H_n(C^{cell}_\bullet(X)) \cong H_n(X)\).

Proof. 
To compute the homology of \(C^{cell}_\bullet(X)\) consider the following commutative diagram, where the rows and columns are exact 
because they are part of the corresponding long exact sequences of relative homology combined with \( H_n(X^{n-1}) = H_{n-1}(X^{n-2}) = H_n(X^{n+1}, X^n) = 0 \)
and \( H_n(X^{n+1}) \simeq H_n(X) \) by a previous result.

(insert image from page 59. Th. 7.3.4)

There is an isomorphism \(H_n(X) \cong H_n(X^n) / \text{im } \partial_{n+1}\). 
Note that \(\pi_n\) is injective and maps \(H_n(X^n)\) to \(\ker d_n\), and \(\text{im } \partial_{n+1}\) to \(\text{im } d_{n+1}\). 
Thus \(H_n(X) \cong \ker d_n / \text{im } d_{n+1}\). But the injectivity of \(\pi_{n-1}\) gives \(\ker d_n = \ker (\pi_{n-1} \circ \partial_n) = \ker \partial_n\). 
Hence \(H_n(X) \cong \ker d_n / \text{im } d_{n+1} = H_n(C^{cell}_\bullet (X))\) as wanted.


Give an explicit description of the boundary map for cellular homology and explain how 
this can be used to compute Homologies.

We already know that the generators of \(C^{cell}_n(X)\) are the \( n \)-cells of \( X \).
Let \( \phi^\alpha_n : \partial e^\alpha_n \simeq S^{n-1} \to X^{n-1} \) be the attaching map.
Consider the composition
\( \phi^{\alpha \beta}_n : S^{n-1} \to \partial e^\alpha_n \to X^{n-1} \to X^{n-1} / (X^{n-1} \setminus e^\beta_{n-1}) \to S^{n-1} \)
where first map is an isomorphism due to the characteristic map \( \Phi^\alpha_n \), the second just the regular attaching map \( \phi^\alpha_n \)
and the third the usual projection into the quotient \( X^{n-1} / (X^{n-1} \setminus e^\beta_{n-1}) \), note that this wraps \( e^\beta_{n-1} \)
into a sphere as all of its attachments are fused into a single point, which explains the last map as an isomorphism (given by \( \Phi^\beta_{n-1} \)

The boundary map \( \partial_n : C^{cell}_n(X) \to C^{cell}_{n-1}(X)\) 
is then
\( \partial_n(e^\alpha_n) = \sum_\beta (\text{deg }) \phi_n^{\alpha\beta} e^\beta_{n-1} \)
where \( \text{deg } \) is the usual degree of sphere mappings \( S^{n-1} \to S^{n-1} \) and we sum
over all generators of \( C^{cell}_{n-1} \).

Example:

The torus \( T^2 \) has a cellular decomposition into one \( 0 \)-cell call it \( x \), two \( 1 \)-cells \( a,b \) and one \( 2 \)-cell \( \sigma \).
Since \( \partial_1 : S^{0} \to S^0 \) is just the constant map its degree must be \( 0 \). 
The attaching of \( \sigma \) is done according to \( ab(ba)^{-1} \) hence \( \partial_2(\sigma) = (1 - 1)a + (1 - 1)b = 0 \) and trivially
\( \partial_3 = 0 \) since there are no \( 3 \)-cells. Thus we have the chain complex

\( 0 \to^{\partial_3} \mathbb{Z} \to^{\partial_2} \mathbb{Z} \oplus \mathbb{Z} \to^{\partial_1} \mathbb{Z} \to 0\)
and the homology group must be
\[
H_k(T) = \begin{cases} 
\mathbb{Z} & k = 0,2 \\ \mathbb{Z}^2 & k = 1 \\ \{0\} & \text{otherwise.} 
\end{cases}
\]

The same procedure can be applied to a general genus \( g \)-surface, just that now we have \( 2g \) \(1\)-cells and the attaching is \( abc\dots(abc\dots)^{-1} \)
so the homology group for \( k=1 \) reads instead \( \mathbb{Z}^{2g} \).
