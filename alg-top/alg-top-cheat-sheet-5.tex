State the Eilenberg-Steenrod Axioms

Eilenberg-Steenrod Axioms for homology theories
A homology theory is a sequence of functors \(H_n \colon \textbf{Top}^{(2)} \to \textbf{AbGrp}\) 
(where \(\textbf{Top}^{(2)}\) is the category of pairs of topological spaces with pairs of continuous functions as morphisms)
with the following properties (in order to lighten notation, we write \(H_n(X, \emptyset) =: H_n(X)\)): 

1. Homotopic invariance: 
If \(f, g \colon (X, A) \to (Y, B)\) are homotopic continuous maps of pairs, 
then \(H_n(f) = H_n(g) \colon H_n(X, A) \to H_n(Y, B)\) for every \(n \in \mathbb{Z}\).

2. Long exact sequence: 
The inclusions 
\[
(A, \emptyset) \hookrightarrow (X, \emptyset) \hookrightarrow (X, A)
\] 
induce a long exact sequence
\[
... \longrightarrow H_n(A) \longrightarrow H_n(X) \longrightarrow H_n(X, A) \xrightarrow{\partial_n} H_{n-1}(A) \longrightarrow ... ,
\]
where the connecting homomorphisms \(\partial_n\) are also functorial, 
in the sense that for any continuous map of pairs \(f \colon (X, A) \to (Y, B)\) 
the following diagram is commutative:
\[
\begin{array}{ccc}
H_n(X, A) & \xrightarrow{\partial_n} & H_{n-1}(A) \\
\downarrow H_n(f) & & \downarrow H_{n-1}(f) \\
H_n(Y, B) & \xrightarrow{\partial_n} & H_{n-1}(B)
\end{array}
\]

This corresponds to the categorical notion of natural transformation between the functors \(H_n(-, -)\) and \(H_{n-1}(-, \emptyset)\).

3. Excision: 
If \(Z \subseteq A \subseteq X\) satisfy \(\overline{Z} \subseteq \text{int} A\) then the inclusion 
\((X \backslash Z, A \backslash Z) \hookrightarrow (X, A)\) induces isomorphisms 
\(H_n(X \backslash Z, A \backslash Z) \cong H_n(X, A)\) for every \(n \in \mathbb{Z}\).

4. Sum: If \(X = \bigcup_{i \in I} X_i\) is a disjoint union (with the disjoint union topology), 
then the inclusions \(X_i \hookrightarrow X\) induce \(\bigoplus_{i \in I} H_n(X_i) \cong H_n(X)\) 
for every \(n \in \mathbb{Z}\). If I is finite, this follows from the previous axioms.

If \((H_n)_{n \in \mathbb{Z}}\) is a homology theory, we call \(H_n(\{p\})\) its coefficients.
If \(H_n(\{p\}) = 0\) for \(n \neq 0\) the theory is ordinary. 

Mayer-Vietoris is interestingly a direct consequence of these axioms (in particular of excision).

State the definition of finitely generated modules
and their structure theorem

For an \(R\)-module \(M\) and a family \(\{x_i\}_{i \in I} \subset M\)
is called a generating set if for each \(m \in M\)
\(m = \sum_{i \in I} \alpha_i x_i\) with only finitely many \(\alpha_i \neq 0\).

If \(|I| < \infty\) we say that \(M\) is finitely generated and \( |I| \) is the rank of the module \( M \).

It is lin. if
\(\sum_{i \in I} \alpha_i x_i = 0 \iff \alpha_i = 0, \forall i \in I\)

If \(\{x_i\}_{i \in I}\) satisfies both these conditions its a basis
and \(M\) is free \(R\)-module.

Equivalently, if there is module homomorphism \( \phi : R^{I} \to M\) with \(\phi(f) = \sum_{i \in I} f(i) x_i\)
which is surjective (gen. set) and injective (lin. ind.).

For every finitely generated module \( M \) over a PID \( R \), there is a unique decreasing sequence of 
ideals \( d_1 \supseteq(d_2) \cdots\supseteq(d_n)\)  such that \( M \) is isomorphic to the direct sum of the cyclic modules:
\( M \simeq R^k \bigoplus_i R/(d_i) = R^k \oplus R/(d_1)\oplus R/(d_2)\oplus\cdots\oplus R/(d_n) \)
The generators \( d_i \) of the ideals are unique up to multiplication by a unit, and are called invariant factors of \( M \). 
Since the ideals should be proper, these factors must not themselves be invertible (this avoids trivial factors in the sum), 
and the inclusion of the ideals means one has divisibility \( d_1\,|\,d_2\,|\,\cdots\,|\,d_n \)
One can also rewrite this into elementary divisors \( q_i \) which are primary ideals and make up the \( d_i \), 
(that is for each \( d_j \) there is some set \( J \), s.t. \( \Pi_{i \in J} q_i = d_j \),
since we are in a PID \( q_i = (p_i)^\alpha_i \) for some prime \( p_i \).
And then 
\( M \simeq R^k \bigoplus_i R/((p_i)^\alpha_i) \)
(note that \( d_i \neq p_i \) except in some trivial cases, since \( p_i | p_j \) can not hold due to primeness.

We call \( \bigoplus_i R/((p_i)^\alpha_i) = T(M) \) the torsion of the module.
Thus we can write \( M = R^\text{rk(M)} \oplus T(M) \).
More generally over non PID but commutative rings, \( T(M) = \{x \in M|r \in R,r\text{ is not a zero divisor and }\ rx = 0 \} \).
For finitely generated groups (which are a special case of modules) this simplifies to 
\( T(G) = \{g \in G \mid \exists n \in \mathbb{Z}_{>0}, ng = 0\} \)

Finally observe that any finitely generated abelian group is just a module with the same generators over the PID \( \mathbb{Z} \)
thus the structure theorem of finitely generated abelian group is just a corollary.

Define the Euler characteristic and Betti numbers and give examples

Let \(X\) be a topological space such that \(H_\ast(X) := \bigoplus_{k \in \mathbb{Z}} H_k(X)\) is finitely generated, i.e. 
each \(H_k(X)\) is finitely generated, and only finitely many of them are non-zero.

1. The \(k\)-th Betti number of \(X\) is \(b_k(X) := \text{rk} \, H_k(X)\).
2. The Euler characteristic of \(X\) is \(\chi(X) := \sum_{k \in \mathbb{Z}} (-1)^k b_k(X)\) 
(which is actually a finite sum under the finiteness assumption above).

Examples:
The \(n\)-dimensional sphere \(S^n\) has Betti numbers \(b_0(S^n) = b_n(S^n) = 1\) and 
\(b_k(S^n) = 0\) for \(k \neq 0, n\). Thus the Euler characteristic is
\[
\chi(S^n) = 1 + (-1)^n = 
\begin{cases}
2 & \text{if } n \text{ even}, \\
0 & \text{if } n \text{ odd}.
\end{cases}
\]

Let \(X\) be a finite cell complex of dimension \(n\). Then
1. \(H_*(X)\) is finitely generated.
2. If \(X\) contains \(a_k\) \(k\)-cells, then \(\chi(X) = \sum_{k=0}^n (-1)^k a_k\).


Define homologies with coefficients

Let \(X\) be a topological space.
A singular \(n\)-chain in \(X\) with coefficients in \(G\) is a formal (finite) sum of the form
\(\sum_{\sigma \in \Delta_n(X)} a_\sigma \sigma, \quad a_\sigma \in G,\)
where \(\Delta_n(X) = \{\sigma : \Delta^n \to X \mid \text{continuous}\}\) denotes the set of all singular \(n\)-simplices in \(X\), 
and \(a_\sigma \neq 0\) for only finitely many \(\sigma\).

More formally: the group of singular \(n\)-chains with coefficients in \(G\) is the abelian group
\(C_n(X; G) := \bigoplus_{\sigma \in \Delta_n(X)} G \sigma\)

The \(n\)-th boundary homomorphism \(\partial_n : C_n(X; G) \to C_{n-1}(X; G),\)

is the group homomorphism defined by
\(\partial_n \left( \sum_{i=0}^n a_\sigma \sigma \right) := \sum_{i=0}^n (-1)^i a_\sigma (\sigma \circ \iota_n^i)\)

(where \((-1)^i a = a\) for even \(i \in \mathbb{Z}\) and \((-1)^i a = -a\) for odd \(i \in \mathbb{Z}\)). 
It is a straighforward verification to show \(\partial_{n-1} \circ \partial_n = 0\) 
thus \((C_\bullet(X; G) = \bigoplus_{n \in \mathbb{Z}} C_n(X; G), \partial)\) is a chain complex.

Another pecularity is the definition of augmented chain complexes with
\(\varepsilon : C_0(X;G) \to C_{-1}(X;G) = G,\ \varepsilon(ap) = a \in G \).

In particular all results built up for singular homology, such as the theorem of small chains
are equally valid for arbitrary abelian groups \( G \) and such theories satisfy the Eilenberg-Steenrod axioms.
Similarly cellular homology also works with arbitrary coefficients, what changes is 
\( H_k(X^n, X^{n-1}) \simeq \bigoplus_{\alpha \in I_n} \tilde{H}_{k-1}(\partial D^n_\alpha;G) \simeq \bigoplus_{\alpha \in I_n} \).
for \( k = n \) and \( 0 \) otherwise.

Continuous maps \( f : X \to Y \) descend into group homomorphisms \( a\sigma \to f_\sharp(a\sigma) = a(f\circ\sigma) \)

If \( X = \{p\} \) is a point, the same computations as with \( \mathbb{Z} \) coeffs.
the singular complex with coefficients in \( G \) has the form

\(\cdots \xrightarrow{} G \xrightarrow{\text{id}_G} G \xrightarrow{0} G \xrightarrow{\text{id}_G} G \xrightarrow{0} G \xrightarrow{} 0, \)

so that \(H_0(\{p\}; G) = G[p]\) and \(H_n(\{p\}) = 0\) for \(n \neq 0\). 
The augmented version gives \(H_n(\{p\}) = 0\) for all \(n \in \mathbb{Z}\).

Thus contractible spaces also have \( H_0(X;G) \simeq G \) and \( 0 \) otherwise.

One can even consider maps between different coefficient homologies, meaning given a homomorphism \( \psi : G_1, \to G_2 \)
this defines a chain map \( \psi_\ast : C_\bullet(X;G_1) \to C_\bullet(X;G_2) \) via \( \psi_\ast(a\sigma) = \psi(a)\sigma,\ a \in G \)
and noting that \( \psi(-a) = -\psi(a) \) for commuting with the differential, thus we have \( \psi_\star : H_\bullet(X;G_1) \to H_\bullet(X;G_2) \).



State and prove the existence of a degree for maps between spheres with arbitrary coefficients for the Homology groups

Let \(f : S^n \to S^n\) be a continuous map of degree \(m\) and \(G\) an abelian group. 
Then \(f_* : \tilde{H}_n(S^n; G) = G \to \tilde{H}_n(S^n; G) = G\) is given by multiplication by \(m\).

Proof. 
We know from Mayer-Vietoris that
\(\varphi_G : G \cong \tilde{H}_0(S^0; G) \cong \ldots \cong \tilde{H}_{n-1}(S^{n-1}) \cong \tilde{H}_n(S^n; G).\)
Further for a group homomorphism \(\psi : G_1 \to G_2\) it holds
(INSERT PAGE diagram 2 from page 67, Lemma 8.1.11)
Let now \([c] \in \tilde{H}_n(S^n; G)\) be any element and set \(g := \varphi_{G}^{-1}([c])\). 
Let \(\psi : \mathbb{Z} \to G\) be the group homomorphism defined by \(\psi(1) = g\) (i.e. \(\psi(k) = kg\) for any \(k \in \mathbb{Z}\)). 
If we now consider the homomorphisms induced by \( f \) in homology, and we deduce 
commutative diagram.
(INSERT PAGE diagram 3 from page 67, Lemma 8.1.11)

State and compute the Homology groups of the real projective space with \( \mathbb{Z}/2\mathbb{Z} \) as coefficients.

The computation of the homology groups \(H_k(\mathbb{P}^n(\mathbb{R}))\) of the real projective spaces was obtained 
via cellular homology using that the cellular differential 
\(
d_k : H_k(\mathbb{P}^k(\mathbb{R}), \mathbb{P}^{k-1}(\mathbb{R})) \to H_{k-1}(\mathbb{P}^{k-1}(\mathbb{R}), \mathbb{P}^{k-2}(\mathbb{R}))
\)

was given by multiplication by \( 1 + (-1)^k \). This computation in turn followed from the fact that the antipodal map 
\( A : S^{k-1} \to S^{k-1} \) has degree \(\text{deg } A = (-1)^k\). Since degrees do not change with coefficients, this still holds for any \( G \), 
so the cellular chain complex for \(\mathbb{P}^n(\mathbb{R})\) with coefficients in \( G \) also looks like

\(
0 \longrightarrow G \longrightarrow G \xrightarrow{0} G \longrightarrow G \xrightarrow{2} G \longrightarrow \dotsb \longrightarrow G \xrightarrow{0} G \longrightarrow G \longrightarrow 0
\)
if \( n \) is even, and

\(
0 \longrightarrow G \longrightarrow G \xrightarrow{2} G \xrightarrow{0} G \longrightarrow \dotsb \longrightarrow G \xrightarrow{2} G \xrightarrow{0} G \longrightarrow 0
\)

if \( n \) is odd.

For \(G = \mathbb{Z}/2\mathbb{Z}\), all differentials become the zero map. Hence it holds

\[
H_k(\mathbb{P}^n(\mathbb{R}); \mathbb{Z}/2\mathbb{Z}) \cong 
\begin{cases}
\mathbb{Z}/2\mathbb{Z} & \text{for any } 0 \leq k \leq n, \\
0 & \text{otherwise.}
\end{cases}
\]

State the prerequisites and theorem of Borsuk-Ulam

The group homomorphism \(\psi_\bullet \colon C_\bullet(\mathbb{P}^n(\mathbb{R}); \mathbb{Z}/2\mathbb{Z}) \to C_\bullet(S^n; \mathbb{Z}/2\mathbb{Z})\) 
is a chain map, and

\(
0 \to C_\bullet(\mathbb{P}^n(\mathbb{R}); \mathbb{Z}/2\mathbb{Z}) \xrightarrow{\psi_\bullet} C_\bullet(S^n; \mathbb{Z}/2\mathbb{Z}) \xrightarrow{\varphi_\bullet} C_\bullet(\mathbb{P}^n(\mathbb{R}); \mathbb{Z}/2\mathbb{Z}) \to 0
\)

is a short exact sequence of chain complexes.

where \( \psi_\sharp \) is the induced map by the covering \( \psi: S^n \to \mathbb{P}^n(\mathbb{R}) \) and 
\( \psi(\sigma) = \sigma_v + \sigma_{-v} = \sigma_v + A_{\sharp}(\sigma)_v\), where \( A: S^n \to S^n \) is the antipodal map
and \( \sigma_v : \Delta^n \to S^n \) the lift of some \( \sigma \in \Delta^k \to \mathbb{P}^n(\mathbb{R}) \), that is a map 
s.t. for \( v \in \phi^{-1}(\sigma(e_0)) \).

This can be used to prove the proposition
Let \(n \in \mathbb{Z}_{\geq 0}\) and \(f : S^n \to S^n\) an odd continuous map (i.e. \(f(-x) = -f(x)\) for every \(x \in S^n\)). 
Then \(\text{deg } f \in \mathbb{Z}\) is an odd number. 

Proof.
The case \(n = 0\) is easy, since there are only two possibilities: 
\(f = \text{id}\) or \(f = -\text{id}\), which have degree 1 and -1 respectively.

Assume from now on that \( n \geq 1 \). Since degrees do not change with coefficients it is enough to show that the map 
\(
f_* : H_n(S^n; \mathbb{Z}/2\mathbb{Z}) \to H_n(S^n; \mathbb{Z}/2\mathbb{Z})
\)
is not the zero map. One then constructs from the SES a LES with connecting homomorphism \( \phi_n \) from which one can derive that 

\( H_n(\mathbb{P}^n(\mathbb{R}); \mathbb{Z}/2\mathbb{Z}) \to H_n(S^n; \mathbb{Z}/2\mathbb{Z}) \)
is an isomorphisms and also 
\( \phi_n : H_n(\mathbb{P}^n(\mathbb{R}); \mathbb{Z}/2\mathbb{Z}) \to H_{n-1}(\mathbb{P}^n(\mathbb{R}); \mathbb{Z}/2\mathbb{Z}) \) 
\( \varphi_\ast : H_0(S^n(\mathbb{R}); \mathbb{Z}/2\mathbb{Z}) \to H_{0}(\mathbb{P}^n(\mathbb{R}); \mathbb{Z}/2\mathbb{Z}) \) 

and last a commutative diagram (and similarly for \( \phi_n, \varphi_\ast \), which forces 
\( f \) to be a an isomorphism and non trivial
(INSERT diagram (8.12) page 70 Proposition 8.2.3.)


Finally we have the theorem of Borsuk-Ulam:
Let \(n > 0\) and \(f : S^n \to \mathbb{R}^n\) be a continuous map. 

1. If \( f \) is odd, then there is a \( p \in S^n \) such that \( f(p) = 0 \). 
2. In any case, there is \( p \in S^n \) such that \( f(p) = f(-p) \).
