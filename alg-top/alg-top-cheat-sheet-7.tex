Define projective/injective modules and their resolutions as well as free ones

1. Projective and Injective Modules

Projective Module: A module \( P \) is projective if for every surjective homomorphism \( f: N → M \) and any homomorphism \( g: P → M \), 
there exists a homomorphism \( h: P → N \) such that \(f ∘ h = g\).



Intuition: You can always "lift" maps into \( M \) to maps into \( N \) if the map into \( M \) starts at a projective module.
Key property: \( \text{Ext}^n(P, A) = 0 \) for all \( n > 0 \) and any module \( A \).
Examples: Free modules, direct summands of free modules.
The converse holds over PID, that is every projective module is also a free module. 

Injective Module: A module \( Q \) is injective if for every injective homomorphism \( f: X → Y \) and any homomorphism \( g: X \to Q \), 
there exists a homomorphism \( h: Y → Q \) such that \( h ∘ f = g \).
Intuition: You can always "extend" maps from \( X \) to maps from \( Y \) if the map from \( X \) ends at an injective module.
Key property: \( \text{Ext}^n(A, Q) = 0 \) for all \( n > 0 \) and any module \( A \).
Examples: Divisible abelian groups (e.g., \( \mathbb{Q}/\mathbb{Z} \)), injective hulls.

2. Resolutions
Projective Resolution: An exact sequence of the form:
\(\cdots \longrightarrow P_1 \longrightarrow P_0 \longrightarrow M \xrightarrow 0\)
where each \( P_i \) is a projective module.
They precisely define split exact seqeuences via their resolution.

Injective Resolution: An exact sequence of the form:
\(0 \longrightarrow M \longrightarrow Q_0 \longrightarrow Q_1 \longrightarrow \cdots \)
where each \( Q_i \) is an injective module.

Free Resolution: A special case of a projective resolution where each \( P_i \) is a free module.
Importance: Resolutions allow us to compute derived functors like \( \text{Ext} \) and \( \text{Tor} \) by replacing modules with 
"nice" (projective or injective) ones.


State some facts about free resolutions

Some examples 
For \(n \in \mathbb{Z}_{>0}\), the group \(A = \mathbb{Z} / n \mathbb{Z}\) has an obvious free resolution given by
\(0 \longrightarrow \mathbb{Z} \xrightarrow{n} \mathbb{Z} \longrightarrow A \longrightarrow 0.\)

Other possible free resolutions could for example be given by 
\(0 \longrightarrow \mathbb{Z}^2 \xrightarrow{f} \mathbb{Z}^2 \longrightarrow A \longrightarrow 0\) with
\(f(x,y) = (nx, y), \text{ or } f(x, y) = (nx-y, y).\)

Every abelian group/module \( A  \)admits a free resolution of length \( 1 \), that is \( F_n = 0,\ n \geq 2 \) with the help of
its presentation for that take elements/generators \( \mathcal{A} \) of the group/module and consider the free group/module 
\( F_0 = \bigoplus_{a \in \mathcal{A}} a\mathbb{Z} \), then use the projection \( \pi : F_0 \to A \) to find the relations \( F_1 = \text{ker}(\pi) \)
e voila \( F_1 \to F_0 \to A \to 0 \) is the presentation in question.

Given presentations/free resolutions \( F^A_{\bullet} \xrightarrow{\pi^A} A \), \( F^B_{\bullet} \xrightarrow{\pi^B} B \) of length one. Then


1. If \(\varphi : A \to B\) is a group homomorphism, there is a chain map \(\psi : F_\bullet A \to F_\bullet B\) lifting \(\varphi\), 
i.e., such that \(\pi_B \circ \psi = \varphi \circ \pi_A\). In other words, there is a commutative diagram of short exact sequences

2. Any two chain maps \(\psi, \psi' : F_\bullet A \to F_\bullet B\) lifting the same group homomorphism \(\varphi : A \to B\) are chain homotopic. 

3. If \(\varphi\) is an isomorphism, any lifting \(\psi\) is a chain homotopy equivalence. In particular, any two free resolutions of \(A\) (of length one) are chain homotopic equivalent.

The chain map is a similar construction to turn SES into SES for the \( \text{Hom} \) functor, namely  \( \pi^A \) is surjective and 
\( F^A_0, F^B_0 \) are free thus for generators \(f_i \in F^B_0, e_i \in F^A_0 \) we have \( \pi_B(f_i) = \phi(\pi_A(e_i)) \) then we define 
\( \psi(e_i) = f_i \). The chain homotopy in question is \( P_i = \psi_i - \psi_i' \).
This is of special interest to show that \( \text{Tor}, \text{Ext} \) are independent of the particular choice of resolution.

Define the Tor/Ext functors and state some properties

Let \(A\) and \(D\) be \(R\)-modules. For any projective resolution of \(A\)
let 
\(
d_n^i : \operatorname{Hom}_R(P_{n-1}, D) \to \operatorname{Hom}_R(P_n, D)
\)
for all \(n \geq 1\). Define
\(
\operatorname{Ext}_R^n(A, D) = \ker d_{n+1}^i / \operatorname{image} d_n^i,
\)

where \( \operatorname{Ext}_R^0(A, D) = \ker d_1^i \). The group \( \operatorname{Ext}_R^n(A, D) \) is called the \( n \)\-th 
cohomology group derived from the contravariant functor \( \operatorname{Hom}_R(-, D) \). 
When \(R = \mathbb{Z}\) the group \( \operatorname{Ext}_R^n(A, D) \) 
is also denoted simply \(\operatorname{Ext}^n(A, D)\) and \(\operatorname{Ext}(A, D) := \operatorname{Ext}^1(A, D)\).
In particular \( \operatorname{Ext}^n \) is the same as the cohomology of interpreting the resolution as a chain complex and replacing 
\( \text{Hom}(A, G) \) with \( 0 \). The functor \( \operatorname{Ext}_R^n(-, D) \) is contravariant (and usually of interest) and 
\( \operatorname{Ext}_R^n(A, -) \) covariant.

Then \( \operatorname{Ext}(A, D) = \text{Hom}(F_1, G) / \text{im}(\iota^\vee) \).

Let \(G\) be any abelian group.
1. If \(A\) is a free group, then \(\operatorname{Ext}(A, G) = 0\). 
2. For any \(n \in \mathbb{Z}_{>0}\), \(\operatorname{Ext}(\mathbb{Z}/n\mathbb{Z}, G) \cong G/nG\), where \(nG := \{ng \mid g \in G\} \subset G\).
3. If \(\{A_i\}_{i \in I}\) is any family of abelian groups, then \(\operatorname{Ext}\left( \bigoplus_{i \in I} A_i, G \right) \cong \prod_{i \in I} \operatorname{Ext}(A_i, G)\).


Let \(D\) be a right \(R\)-module and let \(B\) be a left \(R\)-module. For any projective resolution of \(B\) by left \(R\)-modules, let 
\(
1 \otimes d_n : D \otimes P_n \to D \otimes P_{n-1} 
\)
for all \(n \geq 1\) as in (15). Then
\(
\operatorname{Tor}_n^R(D, B) = \ker(1 \otimes d_n) / \operatorname{image}(1 \otimes d_{n+1})
\)

where \( \operatorname{Tor}_0^R(D, B) = (D \otimes P_0) / \operatorname{image}(1 \otimes d_1) \). 
The group \( \operatorname{Tor}_n^R(D, B) \) is called the \( n \)-th homology group derived from the functor covariant \(D \otimes - \). 
When \(R = \mathbb{Z}\) the group \(\operatorname{Tor}_n^R(D, B)\) is also denoted simply \(\operatorname{Tor}_n(D, B)\) and
\(\operatorname{Tor}(D, B) := \operatorname{Tor}_1(D, B)\)
In particular \( \operatorname{Tor}_n \) is the same as the homology of interpreting the resolution as a chain complex and replacing 
\( D \times B \) with \( 0 \).


Then \( \operatorname{Tor}(A, D) = H_1(F_\bullet \times G) = \text{ker }\iota \times \text{id}_G \)

Let \(G\) be any abelian group.
1. If \(A\) is a free group, then \(\operatorname{Tor}(A, G) = 0\). 
2. For any \(n \in \mathbb{Z}_{>0}\), 
\(\operatorname{Tor}(\mathbb{Z}/n\mathbb{Z}, G) \cong \ker \left( \begin{array}{c} n \\ G \longrightarrow G \end{array} \right) = \{g \in G \mid ng = 0\},\)
the subgroup of \(n\)-torsion elements of \(G\).
3. If \(\{A_i\}_{i \in I}\) is any family of abelian groups, then 
\(\operatorname{Tor}\left( \bigoplus_{i \in I} A_i, G \right) \cong \bigoplus_{i \in I} \operatorname{Tor}(A_i, G). \)

In some very specific sense one can say then that \( \operatorname{Tor},\operatorname{Ext} \) are dual to each other,
in particular the following relation holds if the underyling ring is even a field \( \mathbb{K} \)
\( \operatorname{Ext}^n(A, \text{Hom}(B, \mathbb{K})) \simeq \text{Hom}(\text{Tor}_n(A, B), \mathbb{K}) \)
and of course the more involved universal coefficient theorem.


Prove the indepedence of the resolution for \( \operatorname{Tor}(A, G), \operatorname{Ext}(A, G) \)


For \( \operatorname{Ext}(A, G) \):
Proof. 
If \( F_{\bullet} \xrightarrow{\pi} A \leftarrow F'_{\bullet} \xrightarrow{\pi'} A \) are two different free resolutions, 
there is a chain homotopy equivalence \(\psi: F_{\bullet} \rightarrow F'_{\bullet}\) lifting \(\operatorname{id}_A : A \rightarrow A \), 
i.e. such that \(\pi' \circ \psi_0 = \pi\). 

By a previous Lemma (9.1.7), 
\(\psi^{\vee} : \operatorname{Hom}(F'_{\bullet}, G) \rightarrow \operatorname{Hom}(F_{\bullet}, G) \) is also a chain homotopy equivalence, 
and thus induces isomorphisms \(\psi^{\ast}: H^n(F_{\bullet}; G) \rightarrow H^n(F'_{\bullet}; G)\). 
For \(n = 1\) this gives the wanted assertion.


For \( \operatorname{Tor}(A, G) \):
\textbf{Proof.} 
If \( F_{\bullet} \xrightarrow{\pi} A \leftarrow F'_{\bullet} \xrightarrow{\pi'} A \) are two different free resolutions, 
there is a chain homotopy equivalence \(\psi: F_{\bullet} \rightarrow F'_{\bullet}\) lifting \(\operatorname{id}_A : A \rightarrow A \), 
i.e. such that \(\pi' \circ \psi_0 = \pi\). 

As in the proof of Proposition 10.1.4, if we identify \( F_1 \) with a subgroup of \( F_0 \) and \( F'_1 \) with a subgroup of \( F'_0 \), 
this means that there are group homomorphisms \(\psi_0 : F_0 \rightarrow F'_0 \) and \(\psi'_0: F'_0 \rightarrow F_0 \) such that 
\(\psi_0 (F_1) \subseteq F'_1 \) and \(\psi'_0 (F'_1) \subseteq F_1\), together with chain homotopies 
\( P : F_0 \rightarrow F_1, P' : F'_0 \rightarrow F'_1 \) 
such that \(\psi'_0 \circ \psi_0 - \operatorname{id}_{F_0} = P\) and \(\psi_0 \circ \psi'_0 - \operatorname{id}_{F'_0} = P'\). 
Tensoring with \(\operatorname{id}_G\) shows that 
\(\psi' \otimes \operatorname{id}_G : F_{\bullet} \otimes G \rightarrow F'_{\bullet} \otimes G \) 
is also chain homotopy equivalence 
(with chain homotopy inverse \(\psi' \otimes \operatorname{id}_G\) via the chain homotopies \( P \otimes \operatorname{id}_G\) and 
\( P' \otimes G\)), hence \( H_k (F_{\bullet} \otimes G) \cong H_k (F'_{\bullet} \otimes G)\) for all \( k \in \mathbb{Z}\). 
In particular for \( k = 1\) we obtain the assertion.


State the general version of the universal coefficient theorem with focus on cohomology

Let \(X\) be a topological space, \(A \subseteq X\) a subspace and \(G\) an abelian group. Then:
1. For every \(n \in \mathbb{Z}\) there are split short exact sequences
\(0 \longrightarrow \operatorname{Ext}(H_{n-1}(X, A), G) \xrightarrow{\lambda} H^n(X, A; G) \xrightarrow{\mu} \operatorname{Hom}(H_n(X, A), G) \longrightarrow 0.\),
where \( \lambda \) is given by \( d^\ast : \text{Hom}(B(X, A)_{n-1}, G) \to H^n((X, A_\bullet);G) \), as it can be shown that \( \text{im} d^\ast \simeq \text{Ext}(H_{n-1}, G)\) (so it acts more like an inclusion), and \( \mu : H^n(X, A; G) \to \operatorname{Hom}(H_n(X, A), G) \) is the Kroenecker product, given by \( \phi \in Z^n((X, A);G) \), \( \mu(\phi)([\sigma]) = \phi(a) \).
the descended version of \( d \)
2. If \((Y, B)\) is another pair such that \(H_n(X, A) \cong H_n(Y, B)\) for all \(n \in \mathbb{Z}\), then 
\(
H^n(X, A; G) \cong H^n(Y, B; G) \text{ for all } n \in \mathbb{Z}.
\)
3. The exact sequence \((10.10)\) is even functorial in \((X, A)\), i.e. if \(f : (X, A) \rightarrow (Y, B)\) is a continuous map of pairs, there are commutative diagrams with exact rows
In particular, if \(f\) induces isomorphisms in homology \(f_*\), then the induced homomorphisms \(f^*\) in cohomology are also isomorphisms.

Something that wasn't covered in the lecture is the fact that it also holds for \( X \) of finite dimension
\(0 \longrightarrow H^n(X) \otimes G \longrightarrow H^n(X;G) \longrightarrow \text{Tor}(H^{n+1}(X), G) \to 0.\),
which is also split and is derived from the homology version of the universal coefficient theorem.



State (and prove if possible) some useful identities of the tensor product

1. 
Commutativity
\(
A \otimes_R B \cong B \otimes_R A
\)
- Naturality of the isomorphism means \( \tau_{A, B}(a \otimes b) = b \otimes a \).
However this does not mean for specific elements \( a \otimes b = b \otimes a \), this does only
hold in some subalgebra's of the tensor algebra (e.g. \( \text{Sym} \).
2. Associativity
\(
(A \otimes_R B) \otimes_R C \cong A \otimes_R (B \otimes_R C)
\)
 Naturality of the isomorphism means \(\alpha_{A, B, C}((a \otimes b) \otimes c) = a \otimes (b \otimes c)\).
3. Distributivity over Direct Sums
\(
A \otimes_R (B \oplus C) \cong (A \otimes_R B) \oplus (A \otimes_R C)
\)
4. Tensoring with Free Modules
\(
R^n \otimes_R M \cong M^n
\)
- This is because \(R^n\) can be seen as a direct sum of \(n\) copies of \(R\).
Some degenerate cases are \( R \otimes_R M \cong M \), \( M \otimes_R 0 \cong 0 \)
### Interaction with Hom Functor
1. Hom-Tensor Adjunction
\(
\text{Hom}_R(A \otimes_R B, C) \cong \text{Hom}_R(A, \text{Hom}_R(B, C))
\)
- This is a natural isomorphism providing an adjunction between \(\otimes\) and \(\text{Hom}\).

2. Tensor-Hom Adjunction for Free and Projective Modules
\(
\text{Hom}_R(P, M) \cong M \otimes_R P^*
\)
- Where \(P^* = \text{Hom}_R(P, R)\) if \(P\) is projective (so it holds either way for free \( P \)).

2. Right Exactness of \(\otimes\)
If \(A \rightarrow B \rightarrow C \rightarrow 0\) is exact, then:
\(
A \otimes_R M \rightarrow B \otimes_R M \rightarrow C \otimes_R M \rightarrow 0
\)
is exact for any \(R\)-module \(M\).

3.
There multiple equivalent definitions of flat modules,
1. For all injective \( \phi : K \to L \) of right \( R \)-modules, the map \( \phi \otimes \text{id}_{M} : K \otimes M \to L \otimes M \)
is also injective. As it turns out, this is the same as if \(- \otimes_R M\) preserves exact sequences (in particular over PID's this is the same
as being free, and projective and either way implies torsion free,
Specifically, for a flat module \(M\) and an exact sequence \(0 \rightarrow A \rightarrow B \rightarrow C \rightarrow 0\):
 \(
 0 \rightarrow A \otimes_R M \rightarrow B \otimes_R M \rightarrow C \otimes_R M \rightarrow 0
 \)
 is exact.

3. Tensoring an Exact Sequence
Tensoring with a short exact sequence \(0 \rightarrow A \rightarrow B \rightarrow C \rightarrow 0\) gives:
 \(
 0 \rightarrow M \otimes_R A \rightarrow M \otimes_R B \rightarrow M \otimes_R C
 \)
 but the sequence may not remain exact unless \(M\) is flat.

Change of Rings
1. Base Change
Given a ring homomorphism \(R \rightarrow S\), and an \(R\)-module \(M\):
\(
S \otimes_R M
\)
is an \(S\)-module.
2. Extension and Restriction of Scalars
For a ring homomorphism \(f : R \rightarrow S\), any \(R\)-module \(M\) can be viewed as an \(S\)-module via \(S \otimes_R M\)
that is \( rt = f(r)t \) for \( r \in R \) and \( t \in S \otimes_R M \).

3. Tensor Product of Modules over Different Rings
Given rings \(R\) and \(S\), and \(R\)-module \(A\) and \(S\)-module \(B\):
\(
A \otimes_\mathbb{Z} B
\)
can be made into an \(R \otimes_\mathbb{Z} S\)-module.


1. Tensor Product of Algebras
If \(A\) and \(B\) are \(R\)-algebras, then their tensor product \(A \otimes_R B\) is also an \(R\)-algebra with multiplication defined by:
\(
(a_1 \otimes b_1) \cdot (a_2 \otimes b_2) = (a_1 a_2) \otimes (b_1 b_2)
\)
over a graded ring it holds instead
\(
(a_1 \otimes b_1) \cdot (a_2 \otimes b_2) = (-1)^{\text{deg}(b_1)\text{deg}(a_2)}(a_1 a_2) \otimes (b_1 b_2)
\)


Give some advanced theorems to simplify tensor products

Important Isomorphisms
For ideals \( I, J \) in R, there is a unique R-module isomorphism
\( R/I \otimes_R R/J \simeq R/(I + J) \)
where \( x ⊗ y \mapsto xy \). In particular, taking \( I = J = 0 \), \( R \otimes R \simeq R \) due to \( x \otimes y \simeq xy \).
For an ideal \( I \) in \( R \) and \( R \)-module \( M \), there is a unique \( R \)-module isomorphism
\( (R/I) \otimes_R M \simeq M/IM \)
such that \( r \otimes m \mapsto rm \). 
In particular, taking \( I = (0), R \otimes_R M \simeq M \), by \( r ⊗ m \mapsto rm \), so
\( R \otimes R \simeq R \) as R-modules.
For an ideal \( I \) in \( R \) and \( R \)-module \( M \) there is an R-linear map \( I \otimes_R M \to IM \), 
where \( i \otimes m \mapsto im \), and it’s surjective (not necessarily injective).

Let \( R \) be a domain with fraction field \( K \) and \( V \) be a \( K \)-vector space.
(1) For all \( R \)-modules \( M \), there is an \( R \)-module isomorphism \( V \otimes_R M \simeq V \otimes_R (M/M_{\text{tor}})\) 
is the torsion submodule of M. In particular \( v \otimes m = 0 \) if and only if \( v = 0 \) or \( m \in M_{\text{tor}} \)
and \( M_{\text{tor}} = \text{ker}(M \to K \otimes_R M),\ m \mapsto 1 \otimes m \).
(2) For \( R \)-modules \( M \), if \( M \) is torsion then \( V \otimes_R M = 0 \) and if \( M \) is not torsion and
\( V \) is nonzero then \( V \otimes_R M \neq 0 \).
(3) If \( M \) is an \( R \)-module and \( N \) is a submodule such that \( M/N \) is a torsion \( R \)-module
then \( V \otimes_R N \simeq V \otimes_R M \) as R-modules by \( v \otimes n \mapsto v \otimes n\).


Give examples of how the tensor product acts on certain groups, rings and modules

1. 
Tensor Product of \(\mathbb{Z}\) and \(\mathbb{Q}\)
\( \mathbb{Z} \otimes_{\mathbb{Z}} \mathbb{Q} \cong \mathbb{Q} \)
- This follows because any integer multiplied by a rational number remains a rational number.

2. Tensor Product of \(\mathbb{Z}/n\mathbb{Z}\) and an Abelian Group \(G\)
\(
\mathbb{Z}/n\mathbb{Z} \otimes_{\mathbb{Z}} G \cong G/nG
\)
- Here \(nG = \{ng : g \in G \}\). The tensor product essentially "modulates" the group by \(n\).

3. 
Tensor Product of \(\mathbb{Z}/n\mathbb{Z}\) with \(\mathbb{Z}/m\mathbb{Z}\)**
\(
\mathbb{Z}/n\mathbb{Z} \otimes_{\mathbb{Z}} \mathbb{Z}/m\mathbb{Z} \cong \mathbb{Z}/\gcd(n, m)\mathbb{Z}
\)
- The greatest common divisor (gcd) reflects the largest integer that can consistently divide both \(n\) and \(m\).

4. Tensor Product of \(\mathbb{Z}/n\mathbb{Z}\) and \(\mathbb{Q}\)
\(
\mathbb{Z}/n\mathbb{Z} \otimes_{\mathbb{Z}} \mathbb{Q} \cong 0
\)
- Since \(\mathbb{Q}\) is torsion-free and \(\mathbb{Z}/n\mathbb{Z}\) is torsion, their tensor product is zero.

5. Tensor Product over Different Base Rings
- For example, \(\mathbb{Q} \otimes_{\mathbb{Z}} \mathbb{R}\):
\(
\mathbb{Q} \otimes_{\mathbb{Z}} \mathbb{R} \cong \mathbb{R}
\)
- This holds because \(\mathbb{R}\) can be seen as a divisible \(\mathbb{Z}\)-module.

6. Another Extension
\( \mathbb{C} \otimes_{\mathbb{R}} \mathbb{R}^n \simeq \mathbb{C}^n \)
\( \mathbb{C} \otimes_{\mathbb{R}} M_n(\mathbb{R}) \simeq M_n(\mathbb{C}) \)
due to \( S^n \simeq S \otimes_R R^n,\ (s_1, \dots, s_n) \mapsto \sum_{i=1}^n s_i(1 \otimes e_i) = \sum_{i=1}^n s_i \otimes e_i \).

State the general version of the universal coefficient theorem with focus on homology

Let \( X \) be a topological space, \(A ⊆ X\)  a subspace and \( G \) an abelian group.
Then:
1. For every \( n \in \mathbb{Z} \) there are split short exact sequences
\(0 \longrightarrow H_{n}(X, A) \otimes G \longrightarrow H_n(X, A; G) \longrightarrow \operatorname{Tor}(H_{n-1}(X, A), G) \longrightarrow 0.\),
2. If (Y, B) is another pair such that \( H_n(X, A) \simeq = H_n(Y, B)\) for all \( n \in \mathbb{Z} \), then
\( H_n(X, A; G) \simeq H_n(Y, B; G)\) for all \( n \in \mathbb{Z}\)

3. The exact sequence in (1) is functorial in \( (X, A) \), i.e. if \( f : (X, A) \to (Y, B) \) is a 
continuous map of pairs, there are commutative diagrams with exact rows.

INSERT DIAGRAM FROM PAGE 95 TH. 10.13.18

In particular, if f induces isomorphisms in homology f∗ , then the induced homomorphisms
f∗ in homology with coefficients are also isomorphisms.

In particular, if \( f \) induces isomorphisms in homology \( f_\ast \), then the induced homomorphisms
\( f_\ast \) in homology with coefficients are also isomorphisms.
