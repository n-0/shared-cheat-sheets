Define the cup product and state some properties

Let \(X\) be a topological space. Given singular cochains \(\varphi \in C^n(X; R)\) and \(\psi \in C^m(X; R)\), 
the cup product map \( \cup : C^n(X;R) \times C^m(X;R) \to C^{n+m}(X;R) \) is defined as
\(
(\varphi \cup \psi)(\sigma) = \varphi(\sigma|_{[e_0, \dotsc, e_n]}) \cdot \psi(\sigma|_{[e_n, \dotsc, e_{n+m}]}), \hspace{1cm} (11.1)
\)

where \(\sigma : \Delta^{n+m} = ([e_0, \dotsc, e_n, \dotsc, e_{n+m}]) \to X\) is a singular \((n + m)\)-simplex in \(X\) and the \(\cdot\) is the multiplication in \(R\) 
(and extended \(\mathbb{Z}\)-linearly to any singular chain).

It satisfies for \( \varphi \in C^n(X;R), \psi \in C^m(X;R), \mu \in C^k(X;R) \)
\( (\varphi \cup \psi) \cup \mu = \varphi(\psi \cup \mu) \)
for \( \varphi' \in C^n(X;R), \psi' \in C^m(X;R) \)
\( (\varphi + \phi') \cup \psi = (\varphi \cup \psi) + (\varphi' \cup \psi) \)
and 
\( (\varphi) \cup (\psi + \psi') = (\varphi \cup \psi) + (\varphi \cup \psi') \)

If \( R \) has a multiplicative identity \( 1_R \) then the constant chain \( 1_{C^0} \in C^0(X;R), 1_{C^0}(\sigma) \to 1_R \) 
is an identity element for \( \cup \), that is \( 1_{C^0} \cup \varphi = \varphi \cup 1_{C^0} = 1_R \).

Applying the coboundary operator gives
\( \delta(\varphi \cup \psi) = (\delta \varphi) \cup \psi + (-1)^n \varphi \cup (\delta \psi) \in C^{n+m+1}(X;R) \)
Interestingly its really complicated to define a relation \( h \), s.t. 
\( \psi \cup \varphi = h(\varphi \cup \psi) \), 
however there is a rather simple one for the induced cohomology classes. But first

Let \(f : X \to Y\) be a continuous map and \(R\) a ring. Then it holds:

1. \(f^* (\varphi \cup \psi) = (f^*(\varphi)) \cup (f^*(\psi)) \in C^{n+m}(X; R)\) for every \(\varphi \in C^n(Y; R)\) and \(\psi \in C^m(Y; R)\).
2. If \(R\) has a unit 1, then \(f^*(1) = 1 \in C^0(X; R)\).

Combined with the coboundary formula, one can derive

1. For any topological space \(X\), the formula \([\varphi] \cup [\psi] := [\varphi \cup \psi]\) gives well-defined maps 
\(\cup : H^n(X; R) \times H^m(X; R) \to H^{n+m}(X; R),\)
also called \textbf{cup-product}, satisfying the same properties as the regular one.
2. If \(f : X \to Y\) is a continuous map, then \(f^*([\varphi] \cup [\psi]) = (f^*([\varphi])) \cup (f^*([\psi]))\) for every \([\varphi] \in H^n(X; R)\) and \([\psi] \in H^m(X; R)\).

Assuming that \( R \) is commutative then at least in cohomology the following relation holds
\( [\psi] \cup [\varphi] = (-1)^{mn}[\varphi] \cup [\psi] \).

One can extend the concept to relative cohomology, where it is defined as
\( \cup : H^n(X, A;R) \times H^m(X, B;R) \to H^{n+m}(X, A \cup B;R) \) with \( [\varphi] \cup [\psi] \coloneqq [\varphi \cup \psi] \).
It is possible to show that \(H^{n+m}(X, A \cup B;R) \simeq H^{n+m}(X, A + B;R) \), where \( C^n(A+B;R) = \text{ker}(C^{n+m}(X;R) \xrightarrow{\iota^\vee} C^{n+m}(A+B;R)) \) 
are the chains derived from \( C^n(A;R) \oplus C^n(B;R) \).

From a broader perspective, \( \cup \) turns the complex \( C^\bullet(X;R) \coloneqq \oplus_{n \geq 0} C^n(X;R) \) into a (usual not commutative) ring with identity.
The equality \( [\psi] \cup [\varphi] = (-1)^{mn}[\varphi] \cup [\psi] \) is sometimes used to call \( H^\bullet(X;R) \) a graded commutative ring 
(which is truly commutative if \( R = \mathbb{Z}/2\mathbb{Z} \).

What was only mentioned in the lecture, but not in the script is the fact that the de Rham theorem implies that on differentiable manifolds
the cup product is the same as the wedge product, explaining the strong similarity. 


Give an explicit computation of the cup product 

Consider the Torus via the following diagram

(INSERT DIAGRAM FROM PAGE 101 EXAMPLE 11.1.8)

In particular we have the paths \(a = (p_1, p_2) = (p_3, p_4)\) and \(b = (p_1, p_3) = (p_2, p_4)\), 
as well as the 2-simplices \(\sigma_1 := (p_1, p_2, p_4)\) and \(\sigma_2 := (p_1, p_3, p_4)\). 
Moreover we know that 
\(
H_1(T) = \mathbb{Z} [a] \oplus \mathbb{Z} [b] \quad \text{and} \quad H_2(T) = \mathbb{Z} [\sigma_1 - \sigma_2].
\)

Since the homology groups are all free, the universal coefficient theorem for cohomology gives \(H^k(T; \mathbb{Z}) \cong \operatorname{Hom}(H_k(T), \mathbb{Z})$ for all $k \in \mathbb{Z}\). 
Thus
\(
H^1(T) = \mathbb{Z} [\alpha] \oplus \mathbb{Z} [\beta] \quad \text{and} \quad H^2(T) = \mathbb{Z}[\varphi],
\)

where \(\{[\alpha], [\beta]\}\) is the dual basis of \(\{[a], [b]\}\), resp. \(\{[\varphi]\}\) is the dual basis of \(\{[\sigma_1 - \sigma_2]\}\). 
This means: \(\alpha\) and \(\beta\) are 1-cocycles such that \(\alpha(a) = \beta(b) = 1\) and \(\alpha(b) = \beta(a) = 0\), while \(\varphi\) is a 2-cocycle such that 
\(\varphi(\sigma_1 - \sigma_2) = 1\).

Let's now compute the cup products \([\alpha] \cup [\beta]\) and \([\beta] \cup [\alpha]\). Since \([\alpha] \cup [\beta] \in H^2(T; \mathbb{Z}) = \mathbb{Z}[\varphi]\), 
there is an integer \(n \in \mathbb{Z}\) such that \([\alpha] \cup [\beta] = n[\varphi]\). Since \(\varphi(\sigma_1 - \sigma_2) = 1\), the integer \(n\) can be computed as

\(n = n \cdot \varphi (\sigma_1 - \sigma_2) = ([\alpha] \cup [\beta]) (\sigma_1 - \sigma_2) = (\alpha \cup \beta) (\sigma_1 - \sigma_2)\)
\(= (\alpha \cup \beta) (\sigma_1) - (\alpha \cup \beta) (\sigma_2) = \alpha (\sigma_1|_{[p_1, p_2]}) \beta (\sigma_1|_{[p_2, p_4]}) - \alpha (\sigma_2|_{[p_1, p_3]}) \beta (\sigma_2|_{[p_3, p_4]})\)
\(= \alpha (a) \beta (b) - \alpha (b) \beta (a) = 1\).

Thus \([\alpha] \cup [\beta] = [\varphi]\). A similar computation gives however \([\beta] \cup [\alpha] = - [\alpha] \cup [\beta]\).


Define the cap product and state some properties

For \(n \geq m \geq 0\), define the cap-product map \(\cap : C^m(X; R) \times C^n(X; R) \to C_{n-m}(X; R)\) as follows:

1. For \(\varphi \in C^m(X; R)\), \(\sigma : \Delta^n = ([e_0, \dotsc, e_n]) \to X\) and \(r \in R\), set 

\(\varphi \cap (\sigma) := \varphi(\sigma|_{[e_{n-m}, \dotsc, e_n]}) \cdot r \sigma|_{[e_0, \dotsc, e_{n-m}]} \in C_{n-m}(X; R).\)
and extend this definition \( R \)-linearly.

It holds for \( \varphi, \varphi' \in C^m(X;R), \sigma \in C_m(X;R), r \in R \)
\( (\varphi + \varphi') \cap c = \varphi \cap c + \varphi' \cap c, \ (r\varphi) \cap c = r(\varphi \cap c) \)
For \( 1_{C_0} \in C^0(X;R) \) it holds
\( 1_{C_0} \cap \sigma = 1_{C_0}(\sigma(e_n))\sigma = \sigma \)
and for \( \psi \in C^n(X;R), \nu \in C_n(X;R) \)
\( \psi \cap \nu = \psi(\nu)\nu(e_0) \in C_0(X;R) \)
so with the augmentation morphism, \( \varepsilon : C_0(X;R) \to R \), \( \varepsilon \circ \cap \) coincides with the Kroenecker map, since \( \varepsilon(\psi \cap \nu) = \psi(\nu) \).

Moreover for \( \varphi \in C^n(X;R), \psi \in C^k(X;R) \) it holds
\( (\varphi \cup \psi) \cap c = \varphi \cap (\psi \cap c) \)

and
\( \partial(\varphi \cap c) = (-1)^m(\delta \varphi) \cap c + \varphi \cap (\partial c) \in C_{n-m-1}(X;R) \).
Similarly as for the cup product the cap product decends into (co-)homology via
\( \cap : H^m(X;R) \times H_n(X;R) \to H_{n-m}(X;R), [\varphi] \cap [c] = [\varphi \cap c] \).

For induced chain maps we have the following commutative diagram

(INSERT DIAGRAM 11.11 FROM Corollary 11.2.5, PAGE 104)

and the projection formula
For any \(\varphi \in C^m(Y; R)\) and \(c \in C_n(X; R)\) it holds
\(
f_* (f^* (\varphi) \cap c) = \varphi \cap (f_*(c)) \in C_{n-m}(Y; R).
\)
In particular, for any \([\varphi] \in H^m(Y; R)\) and \([c] \in H_n(X; R)\) it holds
\(
f_* (f^* ([\varphi]) \cap [c]) = [\varphi] \cap (f_*([c])) \in H_{n-m}(Y; R).
\).

Last the cap product extends to relative (co)homology via
\( \cap : H^m(X, A;R) \times H_n(X, A \cup B;R) \to H_{n-m}(X,B;R),\ [\varphi] \cap [c] = [\varphi \cap c] \)
The reason for \( A \) vanishing in the codomain is that \( \varphi : H^m(X,A;R) \) removes the \( A \) part of any singular \( n \)-chain
(which must look sth. like \( \partial c = c_A = c_B \), so \( \varphi \cap (\partial c) = \varphi \cap c_B \), which is a cycle only relative to \( B \).


State the Künneth formula

The Künneth formula over a PID \( R \)  is the SES
\( 0 \longrightarrow \bigoplus_{i+j=k} H_i(X;R) \otimes_R H_j(Y;R) \longrightarrow H_k(X \times Y;R) \longrightarrow \bigoplus_{i+j=k-1} \text{Tor}_1^R(H_i(X;R), H_j(Y;R)) \longrightarrow 0 \)
Which is split thus
\( H_k(X \times Y;R) = \bigoplus_{i+j=k} H_i(X;R) \otimes_R H_j(Y;R) \oplus \bigoplus_{i+j=k-1} \text{Tor}_1^R(H_i(X;R), H_j(Y;R)) \)

this simplifies immensely over a field \( F \) because then \( \text{Tor}_1^R(H_i(X;R), H_j(Y;R) = 0 \), for all \( i,j \).

This was only mentioned in the lecture, but can sometimes come in handy.



Define the orientation of a manifold via homology


Let \( M \) be a \( n \)-manifold then for any open \( U \) homeomorphic to some open ball in \( \mathbb{R}^n \) it holds

\( H_k(M, M \setminus \{p\}; R) \simeq H_k(U, U \setminus \{q\};R) \simeq \tilde{H}_{k-1}(U \setminus \{p\};R) \simeq H_{k-1}(S^{n-1};R) \)
\( \simeq R \) if \( k = n \) and \( 0 \) otherwise.

A local orientation at \( p \) of \( M \)  is an element \( o_p \in H_n(M, M \setminus \{p\};R) \), s.t. \( H_n(M, M \{p\};R) = Ro_p \)  a generator of the 
local homology group. the special case \( R = \mathbb{Z} \) is called a local orientation.
Two local \( R \)-orientation \( o_p, o_p' \) differ at most by a unit, thus for \( R = \mathbb{Z}, R^\ast = \{\pm 1\} \) implies two local orientations at each point,
while for \( R = \mathbb{Z}/2\mathbb{Z} \) a unique local \( \mathbb{Z}/2\mathbb{Z} \)-orientation at each point exists.

Introduce the notation for the inclusion of pairs \( (M, M \setminus A) \to (M, M \setminus \{p\}) \), that is for \( a \in H_n(M, M \setminus A;R) \)
let \( \alpha_p \in H_n(M, M \setminus \{p\}) \) be its image under the descended inclusion.

A family \(\{ \alpha_p \in H_n(M, M \backslash \{p\}; R) \}_{p \in M}\) of elements in the local homology groups is called \(\textbf{coherent}\) 
if for every \(p \in M\) there is an open neighbourhood \(p \in U \subseteq M\) and a class \(\alpha_U \in H_n(M, M \backslash U; R)\) 
such that \((\alpha_U)_q = \alpha_q\) for all \(q \in U\).

An \( R \)-orientation of a topological manifold \( M \) (without boundary) is a coherent family \( \{o_p\}_{p \in M} \) of local \( R \)-orientations at each \( p \in M \).

The following diagram for some open \( V \subseteq U \) whose closure is still in \( U \) explains, why a relative homology class \( \alpha_V \in H_n(M, M\setminus V; R)\)
is uniquely defined for each \( \alpha_p \) s.t. \( ((\alpha_V)_{p} = \alpha_p \).



State the theorems of orientable compact connected manifolds and the Poincare Duality operators

Let \(M\) be a connected \(n\)-dimensional topological manifold and 
\(\{\alpha_p \in H_n(M, M \backslash \{p\}; R)\}_{p \in M}\) a coherent family. Then for any compact subset \(K \subseteq M\) it holds:

a) There is a \( \textbf{unique} \), \(\alpha_K \in H_n(M, M \backslash K; R)\) such that \((\alpha_K)_p = \alpha_p\) for every \(p \in K\).
b) \(H_m(M, M \backslash K; R) = 0\) for all \(m > n\).

The proof is quite involved and consists of showing this for unions of compact sets, then for convex compact sets \( K \), then
for \( K \) in the charts of \( M \) and last for any compact \( K \), by covering of \( M \) via compact sets in these charts.

As a Corollary we obtain for \( M \) itself compact,
1. If \(M\) is \(R\)-orientable, then the natural maps \(H_n(M; R) \to H_n(M, M \backslash \{p\}; R) \cong R\) are isomorphisms for all \(p \in M\).
2. If \(M\) is not \(R\)-orientable, then \(H_n(M; R) \to H_n(M, M \backslash \{p\}; R) \cong R\) has image \(\{r \in R \mid 2r = 0\}\) for every \(p \in M\).
3. \(H_m(M; R) = 0\) for all \(m > n\).

Again the proof requires a lot of work, but 3 follows directly by the choice \( K = M \) and the second via the universal coefficient theorem
to sett that \( H_n(M, M \setminus \{p\};R) \simeq H_n(M, M \setminus \{p\};R) \otimes R \), thus an orientation is given by \( o_p \times r \) for some \( R \),
however if the manifold is not orientable, a coherent manifold must satisfy along a path which would reverse orientation \( o_p \times r = - o_p \times r \), yielding the statement.

We call a generator \( o_M \), s.t. \( H_n(M;R) = Ro_m \) a \( R \)-orientation class or fundamental class of \( M \).

The Poincare-Duality then states
Let \(M\) be an \(n\)-dimensional compact connected \(R\)-orientable manifold and \(o_M \in H_n(M; R)\) an \(R\)-orientation class. 
Then the cap-product with \(o_M\) induces isomorphisms 
\(
D_M := (o_M \cap -) \colon H^k(M; R) \xrightarrow{\cong} H_{n-k}(M; R)
\)
for all \(k \in \mathbb{Z}\), usually called Poincaré-Duality operators.



Give a list of common homology groups

\[ H_k(\mathbb{P}_n(\mathbb{R})) = 
\begin{cases} 
\mathbb{Z} & \text{if } k = 0 \text{ and } k = n \text{ odd} \\
\mathbb{Z}/2\mathbb{Z} &  k \text{ odd}, 0 < k < n \\
0 & \text{otherwise}
\end{cases} \]

\[ H_k(\mathbb{P}_n(\mathbb{C})) = 
\begin{cases} 
\mathbb{Z} & \text{if } k \text{ even} \\
0 & \text{otherwise}
\end{cases} \]

The sphere can be found with cellular homology or inductively via Mayer-Vietoris:

\[ H_k(S^n) = 
\begin{cases} 
\mathbb{Z} & \text{if } k = n \\
0 & \text{otherwise}
\end{cases} \]

except if \( n = 0 \) then \( H_0(S^0) = \mathbb{Z} \oplus \mathbb{Z} \) and otherwise \( 0 \)

For a genus \( g \)-surface \( \Sigma_g \), cellular homology is used with the fundamental polygon \( [a_1,b_1]\dots[a_g,b_g] \) to derive the insight

\[ H_k(\Sigma^g) = 
\begin{cases} 
\mathbb{Z} & \text{if } k = 0,2 \\
\mathbb{Z}^{2g} & \text{if } k = 1 \\
0 & \text{otherwise}
\end{cases} \]

The \( n \)-torus \( T_n \) (the special case of \( n = 2 \) is also solvable via Mayer-Vietoris)
has the CW-complex structure with \( 1 \times 0 \)-cell, \( n \times 1 \)-cells, \( \dots \) and \( 1 \times n\)-cell.
The cell complex is
\( 0 \to \mathbb{Z}^\binom{n}{n} \to \mathbb{Z}^\binom{n}{n-1} \to \dots \to \mathbb{Z}^\binom{n}{1} \to \mathbb{Z}^\binom{n}{0} \to 0\)
with all \( \partial_k = 0 \) one can then see the pattern
\( H_k(T_n) \simeq \mathbb{Z}^\binom{n}{k}\)

For the Mobius strip one can even derive just with simplicial homology
\[ H_k(M) = 
\begin{cases} 
\mathbb{Z} & \text{if } k = 0,1 \\
0 & \text{otherwise}
\end{cases} \]

For the Klein Bottle we have
\[ H_k(K) = 
\begin{cases} 
\mathbb{Z} \oplus \mathbb{Z}/2\mathbb{Z} & \text{if } k = 1 \\
\mathbb{Z} & \text{if } k = 0 \\
0 & \text{otherwise}
\end{cases} \]

In general the homology of a nonorientable closed surface \( N_g \) of genus \( g \) can be found via cellular homology
and is

\[ H_k(N_g) = 
\begin{cases} 
\mathbb{Z}^{g-1} \oplus \mathbb{Z}/2\mathbb{Z} & \text{if } k = 1 \\
\mathbb{Z} & \text{if } k = 0 \\
0 & \text{otherwise}
\end{cases} \]



Derive the homology of (non-)orientable surfaces

Let \(M_g\) be the closed orientable surface of genus \(g\) with its usual CW structure consisting of one 0-cell, \(2g\) 1-cells, 
and one 2-cell attached by the product of commutators \([a_1, b_1] \dotsm [a_g, b_g]\). The associated cellular chain complex is

\(0 \longrightarrow \mathbb{Z} \xrightarrow{d_2} \mathbb{Z}^{2g} \xrightarrow{d_1} \mathbb{Z} \longrightarrow 0\)

As observed above, \(d_1\) must be 0 since there is only one 0-cell. 
Also, \(d_2\) is 0 because each \(a_i\) or \(b_i\) appears with its inverse in \([a_1, b_1] \dotsm [a_g, b_g]\), so the maps \(\Delta_{\beta}^\alpha\) are homotopic to constant maps. 
Since \(d_1\) and \(d_2\) are both zero, the homology groups of \(M_g\) are the same as the cellular chain groups, namely, 
\(\mathbb{Z}\) in dimensions 0 and 2, and \(\mathbb{Z}^{2g}\) in dimension 1.


The closed nonorientable surface \(N_g\) of genus \(g\) has a cell structure
with one 0-cell, \(g\) 1-cells, and one 2-cell attached by the word \(a_1^2 a_2^2 \dotsm a_g^2\). Again
\(d_1 = 0\), and \(d_2 : \mathbb{Z} \to \mathbb{Z}^g\) is specified by the equation \(d_2(1) = (2, \dotsc , 2)\) since each \(a_i\)
appears in the attaching word of the 2-cell with total exponent 2, which means that
each \(\Delta_{\beta}^\alpha\) is homotopic to the map \(z \mapsto z^2\), of degree 2. Since \(d_2(1) = (2, \dotsc , 2)\), we
have \(d_2\) injective and hence \(H_2(N_g) = 0\). If we change the basis for \(\mathbb{Z}^g\) by replacing
the last standard basis element \((0, \dotsc, 0, 1)\) by \((1, \dotsc , 1)\), we see that \(H_1(N_g) \cong 
\mathbb{Z}^{g-1} \oplus \mathbb{Z}_2\).


Derive the homology of the projective space, both real and complex. 

The cell complex of the real projective space  is one \( k \)-cell for all \( k \leq n \)
The attaching maps are the 2-fold covering maps \( \phi_k : S^{k-1} \to \mathbb{P}_{k-1}(\mathbb{R}) \),
to find the degree of the map

\( \mathcal{x}_k : S^{k-1} \xrightarrow{\phi_k} \mathbb{P}_{k-1}(\mathbb{R}) \xrightarrow{q_k} \mathbb{P}_{k-1}(\mathbb{R})/(\mathbb{P}_{k-1}(\mathbb{R}) \setminus e_{k-2}) \simeq S^{k-1}\),
where we can simplify \( \mathbb{P}_{k-1}(\mathbb{R})/(\mathbb{P}_{k-1}(\mathbb{R}) \setminus e_{k-2}) \simeq \mathbb{P}_{k-1}(\mathbb{R})/(\mathbb{P}_{k-2}(\mathbb{R}) \).

This can be done, by noticing that \( \phi_{k-1}^{-1}(\mathbb{P}_{k-2}(\mathbb{R}) \simeq S^{k-2} \) and removing this from \( S^{k-1} \) leaves two connected components,
w.l.o.g. on one component \( \mathcal{x}_k \) acts like a homeomorphism while on the other antipodal, thus  \( \text{deg}(\mathcal{x}_k) = 1 + (-1)^k \) which is \( 2 \) if 
\( k \) is even and otherwise \( 0 \). Inserting this into

\(0 \longrightarrow \mathbb{Z} \xrightarrow{\partial_n} \dots \xrightarrow{2} \mathbb{Z} \xrightarrow{0} \mathbb{Z} \longrightarrow \dots \)
gives everything necessary to finish the proof.

The complex projective space can be done without cellular boundary formula 
To this end note that \( \mathbb{P}_1(\mathbb{C}) \simeq S^2, \mathbb{P}_0(\mathbb{C}) \simeq \{p\} \) which shall serve as a basis for induction,
since \( \mathbb{P}_{n+1}(\mathbb{C}) \) is obtained from \( \mathbb{P}_{n}(\mathbb{C}) \) by attaching a \( 2(n+1) \)-cell, we can then use that
for cell complexes \( H_p(X^{(n)}) \simeq H_p(X^{(p)}) \) for \( n \geq p \), 
(note that this not completely correct since \( p \) goes in \( 2(n+1) \) intervals, but makes it easier to understand what theorem is meant).

By induction hypothesis we can then assert that the left side is only nonzero for even \( p \). For \( p = n, n+1 \) (actually \( p = 2n + 2, 2n + 1 \)
we have the exact sequence

\( 0 \longrightarrow H_{p+1}(\mathbb{P}_n(\mathbb{C})) \longrightarrow H_{p+1}(\mathbb{P}_{n+1}(\mathbb{C})) \longrightarrow \mathbb{Z}\)
\( \longrightarrow H_{p}(\mathbb{P}_n(\mathbb{C})) \longrightarrow H_{p}(\mathbb{P}_{n+1}(\mathbb{C})) \longrightarrow 0\)

Since \(H_{p+1}(\mathbb{P}_n(\mathbb{C})) = 0 =  H_{p+1}(\mathbb{P}_n(\mathbb{C})) \) by induction it follows that
\( H_{p+1}(\mathbb{P}_{n+1}(\mathbb{C})) \simeq \mathbb{Z} \) and \( H_{p}(\mathbb{P}_{n+1}(\mathbb{C})) = 0 \) as desired.



