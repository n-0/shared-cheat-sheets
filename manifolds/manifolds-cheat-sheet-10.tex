State and offer proof of some properties of the exterior derivative



State some useful algebraic identities of the Lie derivative 

For \( f \in C^\infty(M) \)
\( \mathcal{L}_{V}f = Vf \)
Proof:
\( \mathcal{L}_{V}f = \frac{d}{dt}|_{t=0} \theta^\ast_t(f) = \frac{d}{dt}|_{t=0} f(\theta_t) = df(\theta'_0) = df(V) = Vf \)

For \( W \in \mathfrak{M} \)
\( \mathcal{L}_V(W) = [V, W] \)
(already proven on another flash card)

For \( \eta \in \Omega^1(M) \)
\( \mathcal{L}_{V}(\eta)(W) = V(\eta(W)) - \eta([V, W]) \)
Proof:
\[
\begin{aligned}
\mathcal{L}_{V}(\eta(W)) &= \frac{\partial \eta_k}{\partial x^l} V^l W^k + \eta_l \frac{\partial V^l}{\partial x^k} W^k \\
&= \frac{\partial}{\partial x^l} (\eta_k W^k) V^l - \eta_k \frac{\partial W^k}{\partial x^l} V^l + \eta_k \frac{\partial V^k}{\partial x^l} W^l \\
&= V (\eta(W')) - \eta([V, W]).
\end{aligned}
\]
\( \mathcal{L}_{[X, Y]}\omega = [\mathcal{L}_{X}, \mathcal{L}_{Y}]\omega \)

From Cartan's magic formula one can derive 
\( L_{V}(d\omega) = dL_{V}(\omega) \) and \( L_{V}(W \lrcorner \omega) = W \lrcorner dL_{V}(\omega) \)
naturally this can't be used for the formula itself, except for the case \( \omega \in \Omega^{0}(M) = C^\infty(M) \), where this is already known.

Product rules:
\( \mathcal{L}_{V}(\nu(W)) = (\mathcal{L}_V \nu)(W) + \nu \mathcal{L}_V(W) \)
For arbitrary \(\eta \in \Omega^p(M), , \sigma \in \Omega^q(M) \)
\( \mathcal{L}_V(\eta \wedge \sigma) = \mathcal{L}_X(\eta) \wedge \sigma + \eta \wedge \mathcal{L}_V(\sigma) \)
And for arbitrary mixed tensors \( \tau_1, \tau_2 \)
\( \mathcal{L}_{V}(\tau_1 \otimes \tau_2) = (\mathcal{L}_{V}(\tau_1)) \otimes \tau_2 + \tau_1 \otimes (\mathcal{L}_{V}(\tau_2)) \)
Further for \( f \in C^\infty(M) \) and any \( \tau \in \Gamma(T^{k,l}M) \)
\( \mathcal{L}_V(f\tau) = Vf + f\mathcal{L}(\tau) \)

An explicit expression for the Lie Derivative of mixed tensors in Einstein's sum convention is
\[
\begin{align}
(\mathcal{L}_X T) ^{a_1 \ldots a_r}{}_{b_1 \ldots b_s} ={}
& X^c(\partial_c T^{a_1 \ldots a_r}{}_{b_1 \ldots b_s}) \\
& {}-{} (\partial_c X ^{a_1}) T ^{c a_2 \ldots a_r}{}_{b_1 \ldots b_s} - \ldots - (\partial_c X^{a_r}) T ^{a_1 \ldots a_{r-1}c}{}_{b_1 \ldots b_s} \\
& + (\partial_{b_1} X^c) T ^{a_1 \ldots a_r}{}_{c b_2 \ldots b_s} + \ldots + (\partial_{b_s}X^c) T ^{a_1 \ldots a_r}{}_{b_1 \ldots b_{s-1} c}
\end{align}
\]


This reduces for purely covariant tensors and arbitrary vector fields \( Y_1 \in \mathfrak{X}(M) \) to
\( (\mathcal{L}_X T)(Y_1, \dots, Y_k) = X(T(Y_1, \dots, Y_k)) - A([X, Y_1], \dots, Y_k) - \dots - A(Y_1, \dots, [X, Y_k]) \)


E.g. for a covariant rank 2 tensor field \( T = T_{ab}(x^c)dx^a \otimes dx^b \) we have:
\[
\begin{align} 
(\mathcal {L}_X T) &= (\mathcal {L}_X T)_{ab} dx^a\otimes dx^b\\
&= X(T_{ab})dx^a\otimes dx^b + T_{cb} \mathcal{L}_X (dx^c) \otimes dx^b + T_{ac}  dx^a \otimes \mathcal{L}_X (dx^c)\\
&= (X^c \partial_c T_{ab}+T_{cb}\partial_a X^c+T_{ac}\partial_b X^c)dx^a\otimes dx^b\\
\end{align}
\]


State and prove the regular value theorem for the tangent space of the submanifold

Let \( N, K \) be smooth manifolds of dimension \( n,k \) repspectively and \( n \geq k \).
Let \( f \in C^\infty(N, K) \), if \( q \in K \) is a regular value of \( f \) and \( M \coloneqq f^{-1}(q) \neq \emptyset \)
the corresponding submanifold, then \( T_pM = \text{ker } df|_p \subset T_pN \).
Proof:
Let \( c : (-\varepsilon, \varepsilon) \to M \) be an arbitrary smooth curve with \( c(0) = p \).
It suffices to show \( c'(0)\in \text{ker } df|_p \). Since \( \gamma = f \circ c : (-\varepsilon, \varepsilon) \to K \)
is constant (\( \text{im } \gamma = \{q\} \)) the chain rule implies
\( 0 = d\gamma|_0 = df_p \circ dc|_0 \), yielding the result.


State the implicit function theorem and highlight its relation to the regular value theorem

Let \( f: \mathbb{R}^{n+m} \to \mathbb{R}^m \) be a continuously differentiable function, and let \( \mathbb{R}^{n+m} \) have coordinates \( (\textbf{x}, \textbf{y}) \). 
Its Jacobian can be written as: 

\[(Jf)(\mathbf{a},\mathbf{b})
= \left[\begin{array}{ccc|ccc}
 \frac{\partial f_1}{\partial x_1}(\mathbf{a},\mathbf{b}) & \cdots & \frac{\partial f_1}{\partial x_n}(\mathbf{a},\mathbf{b}) &
 \frac{\partial f_1}{\partial y_1}(\mathbf{a},\mathbf{b}) & \cdots & \frac{\partial f_1}{\partial y_m}(\mathbf{a},\mathbf{b}) \\
 \vdots & \ddots & \vdots & \vdots & \ddots & \vdots \\
 \frac{\partial f_m}{\partial x_1}(\mathbf{a},\mathbf{b}) & \cdots & \frac{\partial f_m}{\partial x_n}(\mathbf{a},\mathbf{b}) &
 \frac{\partial f_m}{\partial y_1}(\mathbf{a},\mathbf{b}) & \cdots & \frac{\partial f_m}{\partial y_m}(\mathbf{a},\mathbf{b})
\end{array}\right]
= \left[\begin{array}{c|c} X & Y \end{array}\right]\]

Fix a point \( (\textbf{a}, \textbf{b}) = (a_1, \dots, a_n, b_1, \dots, b_m) \) with \( f(\textbf{a}, \textbf{b}) = \textbf{0} \) (the origin)
If the submatrix \( Y \) of the Jacobian matrix of \( f \):
\( J_{f, \mathbf{y}} (\mathbf{a}, \mathbf{b}) = \left [ \frac{\partial f_i}{\partial y_j} (\mathbf{a}, \mathbf{b}) \right ] \)
is invertible, then there exists an open set \( U \subset \mathbb{R}^n\) containing \( \textbf{a} \) such that there exists a unique function 
\( g: U \to \mathbb{R}^m \) such that \( g(\mathbf{a}) = \mathbf{b} \) and \( f(\mathbf{x}, g(\mathbf{x})) = \mathbf{0}, \forall \mathbf{x} \in U \)   

Moreover, \( g \) is continuously differentiable and denoting the \( X \) part of the overall Jacobian of \( f \) as
\(J_{f, \mathbf{x}} (\mathbf{a}, \mathbf{b}) = \left [ \frac{\partial f_i}{\partial x_j} (\mathbf{a}, \mathbf{b}) \right ]\),
the Jacobian matrix of partial derivatives of \( g \) in \( U \) is given by 

\[
\left[\frac{\partial g_i}{\partial x_j} (\mathbf{x})\right]_{m\times n} =- \left [ J_{f, \mathbf{y}}(\mathbf{x}, g(\mathbf{x})) \right ]_{m \times m} ^{-1} \, \left [ J_{f, \mathbf{x}}(\mathbf{x}, g(\mathbf{x})) \right ]_{m \times n}
\]
Solving these partial differential equations would give an explicit form for \( g \).
Intuitively speaking \( g \) parameterizes the \( \textbf{y} \) in the neighborhood \( U \), s.t. \( f(\textbf{x}, \textbf{y}) = 0 \),
and turns an algebraic relation into the graph of a function.


This was used for the regular value theorem for some smooth \( F : M \to N \) with constant rank \( n \) that 
in local coordinates \( \phi : U \to \mathbb{R}^{n} \times \mathbb{R}^{m+n-n} \simeq \mathbb{R}^{m}, \phi(x) = (Q(x), x_{n+1}, \dots, x_m) \),
where \( Q \) is the restriction of \( F \) to the first \( n \)-th coordinates for which the Jacobian of \( F \) is invertible,
must have rank \( n+m \), thus we can find a right inverse \( \phi^{-1} \).

Some like to think of this as an application of the first part of the inverse function theorem instead, which states
that for cont. diff. \( f : U \to \mathbb{R}^{m} \)
1. \( df|_a \) is surjective iff. there is some cont. diff. \( g \) with \( f \circ g = \text{id} \)
2. \( df|_a \) is injective iff. there is some cont. diff. \( g \) with \( g \circ f = \text{id} \)
The implicit function theorem and the inverse function theorem are completely equivalent to each other.


State the theorem of commuting flows

Let \( V, W \in \mathfrak{X}(M) \) be complete.
Then tfoa equivalent
1. \( [V, W] = 0 \)
2. The flows \( \Phi^V_s, \Phi^W_t \) are commuting, meaning \( \Phi^V_s \circ \Phi^W_t = \Phi^W_t \circ \Phi^V_s \).

Proof:

Define \( \delta_s(t) = F(s, t) = \Phi^V_s \circ \Phi^W_t \)
then \( V, W \) to commute is equivalent to \( \delta_s \) being an integral curve of \( W \)
for all \( s \in \mathbb{R} \), (since this implies that \( \Phi^V_s \) does not change the direction/tangent vector of the flow)
that is
\( \dot{\delta}(t)_s = W(\delta_s(t)) \).
Define
\( X(s) = (d\Phi^V_s)^{-1}\mid_{\Phi^W_t}(\dot{\delta}(t)_s - W(\delta_s(t)) \)
then \( X(s) = 0 \iff \dot{\delta}(t)_s = W(\delta_s(t)) \).
Observe that \( \delta_0(t) = \Phi^W_t \), so that

\( \dot{\delta}(t) = \frac{d}{dt} \Phi^V_s \circ \Phi^W_t = d\Phi^V_s(W(\Phi^W_t)) = d\Phi^V_s((\dot{\delta}_0(t))\)
so
\( (d\Phi^V_s)^{-1}\dot{\delta}(t) = \dot{\delta}_0(t)\).
Rewriting our original expression
\[
\begin{align}
X(s) = (D\Phi^V_{\Phi^V(p)})^{-1} (\dot{\delta}(t) - W(\delta(t))) \\
= \dot{\delta}(t) (D\Phi^V_{\Phi^V(p)})^{-1} (W(\delta(t)))^{-1} \\
= \dot{\delta}(t) (D\Phi^V_{\Phi^V(p)})^{-1} (W \circ \Phi^V (p))) \\
= \dot{\delta}(t) ((\Phi^V)^* W) (\Phi^V (p)) \\
= ((W - (\Phi^V_s)^* W) (\delta_0(t)))
\end{align}
\]

Because \(\Phi^V_{s + s_0} = \Phi^V_{s_0} \circ \Phi^V_s$ is $(\Phi^V_{s + s_0})^* W = (\Phi^V_{s_0})^* ((\Phi^V_s)^* W)\) and therefore

\[
\begin{align}
X'(s_0) = \frac{d}{ds} \bigg|_{s=0} X(s + s_0) 
= \frac{d}{ds} \bigg|_{s=0} ((W - (\Phi^V_{s + s_0})^* W) (\delta_0(t))) \\
= \frac{d}{ds} \bigg|_{s=0} ((\Phi^V_{s_0})^* ((\Phi^V_s)^* W) (\delta_0(t))) \\
= ((\Phi^V_{s_0})^*( \frac{d}{ds} \bigg|_{s=0} (\Phi^V_s)^* W) (\delta_0(t))) \\
= ((\Phi^V_{s_0})^*(L_V W)) (\delta_0 (t)) \\
= - ((\Phi^V_{s_0})^*[V, W]) (\delta_0(t)).
\end{align}
\]

Since the differential of a diffeomorphism is an isomorphism, it holds
\[
(\Phi^V_{s_0})^* \circ L_V W = L_W \circ (\Phi^V_{s_0})^* \iff X = 0 \iff [V, W] = 0.
\]

Use a partition of unity to ensure existence of bump functions and extend a function defined on a closed subset.

A bump function for (closed) \( A \) supported in (open) \( U \) is some \( \psi : M \to \mathbb{R} \), s.t. \( \psi|_A = 1 \)
and \( \text{supp }\psi \subseteq U \) and \(0 \leq \psi \leq 1 \).
The theorem is then that such a \( \psi \) exists for some closed \( A \subseteq M \) of a smooth manifold.
Proof.
Let \( U_0 = U, U_1 = M \setminus A \)and let \( \{\psi_0, \psi_1\} \) be a smooth partition of
unity subordinate to the open cover \( \{U_0, U_1\} \). Because \( \psi_1 = 0 \) on \( A \), thus \( \psi_0 = \sum_i \psi_i = 1 \),
\( \psi_0 \) has the required properties.



Smooth in the extension lemma actually requires something more than usual, namely
that for each point \( p \in A \), there is some open \( W_p \subseteq M \)
and \( F : W_p \to \mathbb{R}^k \) s.t. \( F|_A = f \).

Extension Lemma for Smooth Functions:
Suppose \(M\) is a smooth manifold with or without boundary, \(A \subseteq M\) is a closed subset, and \(f : A \rightarrow \mathbb{R}^k\) is a smooth function. 
For any open subset \(U\) containing \(A\), there exists a smooth function \(\tilde{f} : M \rightarrow \mathbb{R}^k\) such that \(\tilde{f}|_A = f\) and \(\text{supp} \, \tilde{f} \subseteq U\).

Proof: 
For each \(p \in A\), choose a neighborhood \(W_p\) of \(p\) and a smooth function \(\tilde{f}_p : W_p \rightarrow \mathbb{R}^k\) that agrees with \(f\) on \(W_p \cap A\). 
Replacing \(W_p\) by \(W_p \cap U\), we may assume that \(W_p \subseteq U\). The family of sets \(\{W_p : p \in A\} \cup \{M \backslash A\}\) is an open cover of \(M\). 
Let \(\{\psi_p : p \in A\} \cup \{\psi_0\}\) be a smooth partition of unity subordinate to this cover, 
with \(\text{supp} \, \psi_p \subseteq W_p\) and \(\text{supp} \, \psi_0 \subseteq M \backslash A\).

For each \(p \in A\), the product \(\psi_p \tilde{f}_p\) is smooth on \(W_p\) and has a smooth extension to all of \(M\) if we interpret it to be zero on \(M \backslash \text{supp} \, \psi_p\). 
(The extended function is smooth because the two definitions agree on the open subset \(W_p \cap \text{supp} \, \psi_p\) where they overlap.) 
Thus we can define \(\tilde{f} : M \rightarrow \mathbb{R}^k\) by

\(\tilde{f}(x) = \sum_{p \in A} \psi_p(x) \tilde{f}_p(x).\)

Because the collection of supports \(\{\text{supp} \, \psi_p\}\) is locally finite, this sum actually has only a finite number of nonzero terms in a neighborhood of any point of \(M\), 
and therefore defines a smooth function. If \(x \in A\), then \(\psi_0(x) = 0\) and \(\tilde{f}_p(x) = f(x)\) for each \(p\) such that \(\psi_p(x) \neq 0\), so

\(\tilde{f}(x) = \sum_{p \in A} \psi_p(x) f(x) = \left( \psi_0(x) + \sum_{p \in A} \psi_p(x) \right) f(x) = f(x),\)

so \(\tilde{f}\) is indeed an extension of \(f\). It follows from Lemma 1.13(b) that 

\(\text{supp} \, \tilde{f} = \bigcup_{p \in A} \text{supp} \, (\psi_p \tilde{f}_p) = \bigcup_{p \in A} (\text{supp} \, \psi_p \cap \text{supp} \, \tilde{f}_p) \subseteq U.\)



Manifolds with boundaries have some tooling unique to them,
provide some examples.

Per definition a manifold \( M \) with boundary is the same as a manifold except that the charts
\( \phi_i : U_i \to \mathbb{H}^m \) instead of \( \mathbb{R}^m \). The boundary \( \partial M \)
are those points for which \( \phi_i(p) \in \partial \mathbb{H}^m \). In particular this means
that in local coordinates such \( p \) has the shape \( \phi_i(p) = (x_1, x_2, \dots, x_{m-1}, 0) \).

Let \( S \) be embedded in \( M \), then the tangent space satisfies

\( T_p S = \{v \in T_pM: vf = 0, \text{ whenever } f \in C^\infty(M) \text{ and } f|_S = 0\}\)
(this is proven with the help of the local slice condition for \( S \)).

Since boundaries \( \partial M \) are embedded \((m-1)\)-submanifolds, 
this completely describes their tangent space.

If \( p \in \partial M \), a vector \( v \in T_p M \setminus T_p S \) is called inward pointing
if there is some smooth \( \gamma : [0, \varepsilon) \to M \), with \( \gamma(0) = p,\ \gamma'(0) = v \)
and is outward pointing if the domain is \( (-\varepsilon, 0] \). 
A useful proposition states that it suffices to check if in local coordinates of \( v \), 
\( x_m > 0\) if \( v \) is inward pointing,
\( x_m < 0\) if \( v \) is outward pointing,
\( x_m = 0\) if \( v \in T_p S\).

Most generally one can find for any hypersurface (such as boundaries) normals (outward pointing tangent vectors),
with the help of a (boundary) defining function, that is a  \( F : M \to [0, \infty) \) with \( F^{-1}(0) = \partial M \) (or the hypersurface in question),
and \( dF|_{\partial M} \neq 0\), then the negative gradient \( -\nabla F \) is nowhere tangent to \( \partial M \), so we set \( n = -\frac{1}{\|\nabla F\|}\nabla F \).
(usual the norm here refers to a Riemmanian metric).

Such defining functions always exist, via the following insight: Take any cover via smooth charts \( (U_i, \phi_i) \) of \( M \)
and define \( f_i : U_i \to [0, \infty) \) as \( f_i = 1 \) if \( U_i \) is purely in the interior of \( M \) and otherwise \( f(x_1, \dots, x_m) = x_m \).
Take a partition of unity \( \psi_i \) subordinate to \( U_i \) and set \( F = \sum_i \psi_i f_i \).
Let \( v_p \) be inward pointing then \( f(p) = 0 \) since we are on the boundary and
\( dF(v) = \sum_i \psi_i df_i(v) + f_i(v)d\psi_i(v) = \sum_i \psi_i df_i(v) = \sum_i \psi_i dx_m(v) \)
and since \( dx_m(v) > 0 \) we are done.

An example for such \( F \), would be \( F = 1 - |x|^2 \) for the closed unit ball.

Recall from the definition of the hypersurface orientation induced by an orientation form \( \omega \) on \( M \)
and on the boundary vanishing section \( \sigma \in \Gamma(TM) \), local frame \( E_1, \dots, E_{m-1} \)
\( \omega(\sigma, E_1, \dots, E_{m-1}) \) is an orientation for \( M \) if and only if \( \omega(E_1, \dots, E_{m-1}) \) is an orientation for \( \partial M \).
A natural choice would be \( \sigma = \frac{d}{dx^m} \), and \( E_i = \frac{d}{dx^i} \), thus we would assert \( [\frac{d}{dx^1}, \dots, \frac{d}{dx^{m-1}}] \)
as a positive orientation for \( \partial M \). If inserting this as \(\sigma\) into \( \omega \) and reordering coordinates this would give 
\( [\frac{d}{dx^m},\frac{d}{dx^1}, \dots, \frac{d}{dx^{m-1}}] = (-1)^m [\frac{d}{dx^1}, \dots, \frac{d}{dx^{m-1}}, \frac{d}{dx^m}]\) as the orientation on \( M \),
since we want it to be the other way around (\( M \)'s orientation should define \( \partial M \)), we see that for even \( m \), \( \partial M \) has the same sign
as \( M \) and for odd an opposite one.

