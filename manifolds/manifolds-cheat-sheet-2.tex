Define a smooth map between manifolds

A somewhat idiosyncratic style is to refer to use the term function only for maps with codomain \( \mathbb{R}^m \).

Suppose \(M\) is a smooth \(n\)-manifold, \(k\) is a nonnegative integer, and \(f : M \rightarrow \mathbb{R}^k\) is any function. We say that \(f\) is a smooth function
if for every \(p \in M\), there exists a smooth chart \((U, \varphi)\) for \(M\) whose domain contains \(p\) and 
such that the composite function \(f \circ \varphi^{-1}\) is smooth on the open subset \(\hat{U} = \varphi(U) \subseteq \mathbb{R}^n\).

Given a function \( f : M \to \mathbb{R}^k \) and a chart \( (U, \phi) \) for \( M \), the function \( \hat{f}: \phi(U) \to \mathbb{R}^k \)
defined by \( \hat{f}(x) = f \circ \phi^{-1}(x) \) is called the coordinate representatin of \( f \). By definition \( f \) is smooth iff.
its coord. repr. is smooth in some smooth chart around each point. In fact a smooth function satisfies that \( \hat{f} \) is smooth for all smooth charts.

The definition of smooth functions generalizes easily to maps between manifolds.
Let \(M, N\) be smooth manifolds, and let \(F : M \rightarrow N\) be any map. We say that \(F\) is a smooth map if for every \(p \in M\), there exist smooth charts \((U, \varphi)\) containing \(p\) and 
\((V, \psi)\) containing \(F(p)\) such that \(F(U) \subseteq V\) and the composite map \(\psi \circ F \circ \varphi^{-1}\) is smooth from \(\varphi(U)\) to \(\psi(V)\). If \(M\) and \(N\) are smooth manifolds.

These notions can be generalized to not necessarily smooth maps, but only continuously \( k \)-times differentiable maps.
Be careful that the notion of continuity is less restrictive in the case of manifolds (see also other flash card) and that differentiability requires adaptedness.
Let \( C^k(M, N) \) be the space of all these maps between manifolds \( M, N \), then this is a unital Banach algebra under addition and
multiplication over \( \mathbb{R} \) and \( C^{\infty}(M, N) = \cap_{k \in \mathbb{N}} C^{k}(M, N) \). We abbreviate \( C^{k}(M) = C^{k}(M, \mathbb{R}^m) \)

A helpful result is
Gluing Lemma for Smooth Maps
Let \(M\) and \(N\) be smooth manifolds with or without boundary, and let \(\{U_\alpha\}_{\alpha \in A}\) be an open cover of \(M\). Suppose that for each \(\alpha \in A\), we are given a smooth map \(F_\alpha : U_\alpha \to N\) such that the maps agree on overlaps: \(F_\alpha |_{U_\alpha \cap U_\beta} = F_\beta |_{U_\alpha \cap U_\beta}\) for all \(\alpha\) and \(\beta\). Then there exists a unique smooth map \(F : M \to N\) such that \(F|_{U_\alpha} = F_\alpha\) for each \(\alpha \in A\).


Define continuity of maps between manifolds, adaptedness and differentiability

A map \( f : M \to N \) between top. spaces \( (M, \tau_M),\ (N, \tau_N) \) is continuous in \( p \in M \), if for each open neighborhood \( V \in \tau_N\) 
of \( f(p) \) the preimage \( f^{-1}(V) \) is a neighborhood of \( p \).

Be careful that this does not require \( f^{-1}(V) \) to be open as well (only to contain some open set from \( \tau_M \)).
An example highlighting this is \( f(x) = x \) if \( x \neq 1 \) and otherwise \( 0 \) then \( f \) is continuous at \( 0 \) even though
\( (-\epsilon, \epsilon) \cup \{1\} \) is not open.
However a continuous function \( f \) is defined to be continuous in every point and then we have \( f^{-1}(V) \in \tau_M \).

Let \( (M, \mathcal{A}_M), (N, \mathcal{N}) \) be \( C^k \)-manifolds, and \( f \) continuous in \( p \in M \) and \( \phi : V \to \Lambda \)
any chart from \( \mathcal{A}_N \) containing \( q = f(p) \). Then there is a chart \( \psi: U \to \Omega \) containing \( p \) compatible with \( \mathcal{A}_M \) 
with \( f(U) \subset V \).
Proof:
Take any chart \( (\tilde{\psi}, \tilde{U}) \) containing \( p \). Since \( f^{-1}(V) \) contains some open set \( p \in \hat{U} \) we set \( U = \tilde{U} \cap \hat{U} \)
and restrict \( \psi = \tilde{\psi}|_{U} \) to obtain the desired chart.

Local charts \( (U, \psi), (V, \phi) \) with \( p \in U \) and \( f(p) \in V \) with \( f(U) \subset V \) are referred to as adapted and exist according to the previous lemma.

A map \( f : M \to N \) is called differentiable in \( p \), if there are adapted charts \( (U, \psi), (V, \phi) \) s.t.
\( \phi \circ f \circ \psi^{-1} : \mathbb{R}^m \to \mathbb{R}^n \) is differentiable.
