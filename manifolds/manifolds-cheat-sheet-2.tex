Define a smooth map between manifolds

A somewhat idiosyncratic style is to refer to use the term function only for maps with codomain \( \mathbb{R}^m \).

Suppose \(M\) is a smooth \(n\)-manifold, \(k\) is a nonnegative integer, and \(f : M \rightarrow \mathbb{R}^k\) is any function. We say that \(f\) is a smooth function
if for every \(p \in M\), there exists a smooth chart \((U, \varphi)\) for \(M\) whose domain contains \(p\) and 
such that the composite function \(f \circ \varphi^{-1}\) is smooth on the open subset \(\hat{U} = \varphi(U) \subseteq \mathbb{R}^n\).

Given a function \( f : M \to \mathbb{R}^k \) and a chart \( (U, \phi) \) for \( M \), the function \( \hat{f}: \phi(U) \to \mathbb{R}^k \)
defined by \( \hat{f}(x) = f \circ \phi^{-1}(x) \) is called the coordinate representatin of \( f \). By definition \( f \) is smooth iff.
its coord. repr. is smooth in some smooth chart around each point. In fact a smooth function satisfies that \( \hat{f} \) is smooth for all smooth charts.

The definition of smooth functions generalizes easily to maps between manifolds.
Let \(M, N\) be smooth manifolds, and let \(F : M \rightarrow N\) be any map. We say that \(F\) is a smooth map if for every \(p \in M\), there exist smooth charts \((U, \varphi)\) containing \(p\) and 
\((V, \psi)\) containing \(F(p)\) such that \(F(U) \subseteq V\) and the composite map \(\psi \circ F \circ \varphi^{-1}\) is smooth from \(\varphi(U)\) to \(\psi(V)\). If \(M\) and \(N\) are smooth manifolds.

These notions can be generalized to not necessarily smooth maps, but only continuously \( k \)-times differentiable maps.
Be careful that the notion of continuity is less restrictive in the case of manifolds (see also other flash card) and that differentiability requires adaptedness.
Let \( C^k(M, N) \) be the space of all these maps between manifolds \( M, N \), then this is a unital Banach algebra under addition and
multiplication over \( \mathbb{R} \) and \( C^{\infty}(M, N) = \cap_{k \in \mathbb{N}} C^{k}(M, N) \). We abbreviate \( C^{k}(M) = C^{k}(M, \mathbb{R}^m) \)

A helpful result is
Gluing Lemma for Smooth Maps
Let \(M\) and \(N\) be smooth manifolds with or without boundary, and let \(\{U_\alpha\}_{\alpha \in A}\) be an open cover of \(M\). Suppose that for each \(\alpha \in A\), we are given a smooth map \(F_\alpha : U_\alpha \to N\) such that the maps agree on overlaps: \(F_\alpha |_{U_\alpha \cap U_\beta} = F_\beta |_{U_\alpha \cap U_\beta}\) for all \(\alpha\) and \(\beta\). Then there exists a unique smooth map \(F : M \to N\) such that \(F|_{U_\alpha} = F_\alpha\) for each \(\alpha \in A\).


Define continuity of maps between manifolds, adaptedness and differentiability

A map \( f : M \to N \) between top. spaces \( (M, \tau_M),\ (N, \tau_N) \) is continuous in \( p \in M \), if for each open neighborhood \( V \in \tau_N\) 
of \( f(p) \) the preimage \( f^{-1}(V) \) is a neighborhood of \( p \).

Be careful that this does not require \( f^{-1}(V) \) to be open as well (only to contain some open set from \( \tau_M \)).
An example highlighting this is \( f(x) = x \) if \( x \neq 1 \) and otherwise \( 0 \) then \( f \) is continuous at \( 0 \) even though
\( (-\epsilon, \epsilon) \cup \{1\} \) is not open.
However a continuous function \( f \) is defined to be continuous in every point and then we have \( f^{-1}(V) \in \tau_M \).

Let \( (M, \mathcal{A}_M), (N, \mathcal{N}) \) be \( C^k \)-manifolds, and \( f \) continuous in \( p \in M \) and \( \phi : V \to \Lambda \)
any chart from \( \mathcal{A}_N \) containing \( q = f(p) \). Then there is a chart \( \psi: U \to \Omega \) containing \( p \) compatible with \( \mathcal{A}_M \) 
with \( f(U) \subset V \).
Proof:
Take any chart \( (\tilde{\psi}, \tilde{U}) \) containing \( p \). Since \( f^{-1}(V) \) contains some open set \( p \in \hat{U} \) we set \( U = \tilde{U} \cap \hat{U} \)
and restrict \( \psi = \tilde{\psi}|_{U} \) to obtain the desired chart.

Local charts \( (U, \psi), (V, \phi) \) with \( p \in U \) and \( f(p) \in V \) with \( f(U) \subset V \) are referred to as adapted and exist according to the previous lemma.

A map \( f : M \to N \) is called differentiable in \( p \), if there are adapted charts \( (U, \psi), (V, \phi) \) s.t.
\( \phi \circ f \circ \psi^{-1} : \mathbb{R}^m \to \mathbb{R}^n \) is differentiable.

A map \( f : M \to N \) is a diffeomorphism, if it's a topological homeomorphism and differentiable.

The local diffeomorphism criterion
Let \( M, N \) be differentiable manifolds of the same dimension \( m \) and \( f \in C^1(M, N) \).
If \( \text{rk}\ f|_p = m \) for \( p \in M \), then there exists an open neighborhood \(p \in U\), s.t.
\( f|_{U} : U \to f(U) \) becomes a diffeomorphism.
Proof:
We know already that the local representation \( \phi \circ f \circ \psi^{-1} \), \( (U, \psi), (V, \phi) \) is a diffeomorphism
(since we have full rank at \( p \) by assumption), so we can simply take \( U = \psi^{-1}(V) \).

Diffeomorphisms and Homeomorphism are equivalence relation on the set of topological manifolds.
Should there be one differentiable structure on a manifold, there are infinitely many of them
(by simply shifting a chart by a constant for example). While different by construction, most of them are diffeomorphic. 

For \( \text{dim}(M) = \text{dim}(N) \leq 3 \), homeomorphic suprisingly implies diffeomorphic.



Define critical/regular values/points and the Jacobi matrix of a differentiable map and state some basic facts

If \( \Phi : M \to N \) is a
smooth map, a point \( p \in M \) is said to be a regular point of \( \Phi \) if \( d\Phi_p : T_p M \to T_{\phi(p)}N \)
is surjective; it is a critical point of \( \Phi \) otherwise. This means, in particular,
that every point of M is critical if \( \text{dim}\ M < \text{dim}\ N \), and every point is regular if and
only if \( \Phi \) is a submersion. 

A point \( c \in N \) is said to be a regular value of \( \Phi \) if
every point of the level set \( \Phi^{-1}(c) \) is a regular point, and a critical value otherwise.
In particular, if \( \Phi^{-1}(c) = \emptyset \), then \( c \) is a regular value.

If we have a local representation of \( \Phi \) via adapted charts \( (U, \phi), (V, \psi) \)
we define the \( \alpha \)-component of \( \Phi \) as

\( \Phi^{\alpha} = \psi^{\alpha} \circ \Phi \circ \phi^{-1} \)
and if \( \Phi \) is differentiable at \( p \) we can then describe the differential locally as

\( J_{\Phi_{\phi,\psi}}(\phi(p))(\frac{\partial f^{\alpha}}{\partial \phi^i}(\phi(p)) \)
for \( \alpha \in [1, \dots, n], i \in [1, \dots, m] \).

As \( \text{dim}\ \text{col}\ J = \text{dim}\ \text{row}\ J\) we have \( \text{rk} J = \text{min}\{m, n\} \).
This rank is independent of charts, since the transition maps as diffeomorphism must have the maximal rank
possible.
It should be clear then that a regular point is equivalent to
\( \text{rk} J(p) = \text{min}\{m, n\} \).

We then have the following facts 

For the set of critical points \(P^{\text{cr}}[\Phi]\) and the set of critical values \(W^{\text{cr}}[\Phi]\) we have:
\(W^{\text{cr}}[\Phi] = \Phi(P^\text{{cr}}[\Phi]).\)
Furthermore, for the set of regular points \(P^{\text{reg}}[\Phi]\) and the set of regular values \(W^{\text{reg}}[\Phi]\) we have:
\(P^{\text{reg}}[\Phi] \cup P^{\text{cr}}[\Phi] = M, \quad W^{\text{\text{reg}}}[\Phi] \cup W^{\text{cr}}[\Phi] = N\)
and the union is disjoint in each case. However, because of \(\Phi^{-1}(W^{\text{reg}}[\Phi]) \subset P^{\text{reg}}[\Phi]\), in general only
\(W^{\text{reg}}[\Phi] \cap \Phi(M) \subset \Phi(P^{\text{reg}}[\Phi]).\)

Open/Closedness of regular/criticial points/values
Let \( M, N \) be of the same dimension \( m \) and \( \Phi \in C^1(M, N) \), then
1. The set of critical points \( P^{\text{cr}}[\Phi] \) is closed, while \( P^{\text{reg}}[\Phi] \) is open.
2. If \( M \) is compact, then \( W^{\text{cr}}[\Phi] \subset N \) is closed and \( W^{\text{reg}}[\Phi] \) open.
Proof:

The determinant is a continuous function and \( \text{det}(J)^{-1}(0) = P^{\text{cr}}[\Phi] \) must then be closed.
Since \( P^{\text{cr}}[\Phi] = M \setminus P^{\text{reg}}[\Phi] \), the first part is done.

For 2, note that \( P^{\text{cr}}[\Phi] \) closedness implies also compactness and by continuity then
also \( f(P^{\text{cr}}[\Phi]) = W^{\text{cr}}[\Phi] \) is compact, which implies with 
\( W^{\text{reg}}[\Phi] = N \setminus W^{\text{cr}}[\Phi] \) the rest.


State the prerequisites and the theorems themselves from Brown and Sard.

Let \(M\) and \(N\) be two differentiable manifolds of dimension \(m\) and \(n\), 
and a mapping \(f \in C^1(M, N)\). 
Furthermore, let \(M\) be compact. If \(q \in N\) is a regular value, then the following holds:

1. The preimage \(f^{-1}(q)\) is finite.
2. There exists an open neighborhood \(V\) of \(q\), such that 
\(\#(f^{-1}(q')) = \#(f^{-1}(q)), \quad \forall q' \in V.\)


2.5.7 Lemma
Let \(A \subset \mathbb{R}^m\) be a null set, that is, let \(\lambda_m(A) = 0\), where \(\lambda_m\) denotes the m-dimensional Lebesgue measure.
Then for every \(\phi \in C^1(\mathbb{R}^m)\) also \(\phi(A)\) is a null set. 

Definition (Null Set)
\((M, [\omega])\) be a differentiable manifold. We say \(A \subset M\) is a null set, if for every compatible chart \(x : U \to \Omega\) the set \(x(U \cap A) \subset \mathbb{R}^m\) is a Lebesgue null set.

Lemma
\((M, [\omega])\) be a differentiable manifold. Then \(A \subset M\) is already a null set, if for every chart \(x : U \to \Omega\) from a compatible atlas \(\mathscr{A}\) the set \(x(U \cap A) \subset \mathbb{R}^m\) is a Lebesgue null set.

Lemma
Let \(M, N\) be differentiable manifolds of dimensions \(m\) and \(n\) respectively, and let \(f: M \to N\) be continuously differentiable.
1. If \(m = n\) and \(A \subset M\) is a null set, then \(f(A) \subset N\) is also a null set.
2. For \(m < n\), \(f(M) \subset N\) is a null set.
Proof:
The first part follows from Lemma 2.5.7, as it implies that \( (\psi \circ f)(A) \) must be a null set and the definition of null sets for manifolds does the rest.
Build the product manifold \( M \times \mathbb{R}^{n - m} \) and the new function \(F: M \times \mathbb{R}^{n - m} \to N,\ F(x, p) = f(x) \),
since \(M \times \{x_0\}\) is a null set for all \( x_0 \in M \times \mathbb{R}^{n - m}\) and \( F \) a function from two manifolds of the same dimension
we can apply part 1.

Theorem (Sard's Theorem)
Let \(M, N\) be differentiable manifolds with \(m := \dim M \leq n := \dim N\) and a 
mapping \(f \in C^1(M, N)\). Then the set \(W^{cr}[f]\) of critical values of \(f\) is a null set.

Theorem (Brown's Theorem)
Let \(M, N\) be differentiable manifolds with \(m := \dim M \leq n := \dim N\) and a mapping \(f \in C^1(M, N)\). 
Then the set \(W^{reg}[f] \subset N\) of regular values of \(f\) is a dense subset of \(N\).
Null sets have empty interior, hence by Sard's theorem its complement must be dense.


Define partitions of unity and state their existence for
manifolds.

Let \(M\) be a differentiable manifold.
(a) A partition of unity on \(M\) is a family \((f_i)_{i \in I}\) of functions on \(M\), here \(I\) is an arbitrary index set, with the following properties:
(i) \(f_i \geq 0\) on \(M\).
(ii) \((A_i)_{i \in I}\) with \(A_i := \text{supp} f_i\), \(i \in I\), forms a locally finite cover of \(M\). Here, \(\text{supp} f_i = \{x \in A_i : f_i(x) \neq 0\}\) denotes the support of \(f_i\).
(iii) It is \(\sum_{i \in I} f_i(p) = 1\) for all \(p \in M\).

Note that, due to (ii), the sum in (iii) is well-defined.
The partition is called smooth, if all functions \(f_i\) are smooth.

(b) If \((U_i)_{i \in I}\) is an open cover of \(M\) and \((f_i)_{i \in I}\) is a partition of unity with \(\text{supp} f_i \subset U_i\), we say \((f_i)_{i \in I}\) is a partition of unity with respect to \((U_i)_{i \in I}\).

Let \( M \) be a differentiable manifolds and \( \{U_i\}_{i \in I}\) an open cover. Then there exists a locally finite refinement \( \{V_j\}_{j \in J}\)
and a smooth partition of unity with respect to \( \{V_j\}_{j \in J} \).
Proof:

We set \(B(0,r) := \{x \in \mathbb{R}^m : \|x\| < r\}\). There exists a smooth function \(g: \mathbb{R}^m \to \mathbb{R}\) with
*\(g(x) = 1\), for \(x \in B(0,1)\)
*\(g(x) \geq 0\), for \(x \in B(0,2)\)
*\(g(x) = 0\), for \(x \notin B(0,2)\).

This can be seen as follows. Let 
\[h(t) := 
\begin{cases}
0, & t \leq 0 \\
e^{-1/t}, & t > 0
\end{cases}.
\]

\(h\) is smooth. Then the function 
\( g(x) = \frac{h(4 - \|x\|^2)}{h(4 - \|x\|^2) + h(\|x\|^2 - 1)} \)
is smooth as well and has the desired properties.

(Insert image from page 27 script (Th. 2.3.3))

As manifolds are paracompact it immediately follows the existence of the locally finite refinement \( \{V_j\}_{j \in J}\),
updating the charts such that \( B(0, 2) \subset \phi(V_j) \) is possible without issues and moreover
\( \bar{V}_j = \phi^{-1}_j(B(0, 1)) \) is still a locally finite refinement .

Setting \( g_j(p) = (g \circ \phi_j)(p) \) for \( p \in V_j \) and otherwise \( g_j(p) = 0 \) and then \( f_j = \frac{g_j}{\sum_k g_k} \)
is the desired partition of unity.
