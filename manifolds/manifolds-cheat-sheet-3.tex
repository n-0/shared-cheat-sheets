Define the tangent space of a manifold

The tangent space is a linear (more rarely, affine) space of vectors tangent to a smooth (differentiable) manifold \( M \) at a given point \( a \in M \),
denoted by \( T_a M \)

For curves, surfaces and submanifolds embedded in a Euclidean subspace \( \mathbb{R}^n \) the tangent subspace can be identified with an affine subset 
(of the corresponding dimension \( 1,2,\dots \)) in the ambient space, passing through \( a \). 
For abstract manifolds \( T_aM \) is naturally isomorphic to the linear space of derivations \( D : C^\infty(M) \to \mathbb{R}^m \) of 

of the ring of smooth functions on \( M \) satisfying the Leibniz rule:
\(  D\in T_a M\iff D:C^\infty(M)\to\R\) 
\(D(f\pm g)=Df\pm Dg,\ D(\lambda f)=\lambda Df\) 
\(D(f\cdot g)=f(a)\cdot Dg+g(a)\cdot Df. \)

The main reason for the abstraction is that we do not have any ambient space in which we could define the vectors tangent to the manifold.
The (disjoint) union of tangent spaces to a manifold to all its points has a natural structure of a over \( M \), called the tangent bundle.

To make the relation between the affine space of vectors tangent to a point and the ring of derivations more concrete,
say \( M \subset \mathbb{R}^m \) and take any \( v \in \mathbb{R}_{a}^m \) (vectors tangent to the point \( a \)) 
and \( f \in C^{\infty}(\mathbb{R}^n) \) then we can build the directional derivative at \( a \) along \( v \) via

\( D_v(f)|_a = D_v f(a) = \lim_{h \to 0} \frac{f(a + hv) - f(a)}{h} = \frac{d}{dt} f(a + tv)|_{t=0} \)
this is linear and satisfies of course the Lebniz rule.
If we write \( v_a = v^i e_i|_a \) in a standard basis the chain rule gives
\( \frac{d}{dt} f(a + tv)|_{t=0} = v(\frac{df}{dx}(a + tv))|_{t=0} = v^i\frac{df}{dx^i}(a) \).

In fact one can prove that there is an isomorphism of \( v \in T_a M \) and derivations \( D_v \) at \( a \),
but the latter exists without any prior assumption of \( M \) being in \( \mathbb{R}^m \).

Conversely one can interpret \( D(f) : \mathbb{R}^m \to \mathbb{R}^n \) (as was it done in the lecture), although this requires
additional effort because \( \mathbb{R}^m \) actually refers here to a local parameterization of \( f \) via some chart
\( (\phi, U) \) and actually \( f : U \to \mathbb{R}^n \) such that \( D(f) \coloneqq D(\tilde{f} \) and \( \tilde{f} = f \circ \phi^{-1} \).

This causes then the issue of verifying compatibility with transition maps, that is our tangent vectors \( v \) are taken from the tangent space 
to \( \phi(a) \in \mathbb{R}^m \) and one has to show that if \( (\psi, V) \), \( a \in V \) and any \( f : U \to \mathbb{R}^n \)
parameterized via \( \phi \) by default and \( \tilde{f} \) its parameterization via \( \psi \):

\( Df(v)|_a = D(f \circ \phi^{-1})(v)|_a = D(f \circ \psi^{-1} \circ \psi \circ \phi^{-1})(v)|_a = D(\tilde{f} \circ \psi \circ \phi^{-1})(v)|_a \)
\( = D\tilde{f})(D \psi \circ \phi^{-1}(v))|_{\psi(a)}\)

so if \( Df(v) = D\tilde{f}(\tilde{v})\) then it must be \( \tilde{v} = D \psi \circ \phi^{-1}(v) \).
Thus the tangent space at \( a \) is here defined as the quotient space of all \( v \in T_{\phi(a)}, \tilde{v} \in T_{\psi} \)
s.t.   
\( \tilde{v} = D(\psi \circ \phi^{-1})(v)|_{a} \)


Define the differential of a map \( F : M \to N \) at \( p \)

Assuming that \( F \) is \( C^k \) differentiable at \( p \) the differential can be defined via the derivation construction
as the map \( dF_p : T_p M \to T_{F(p)} N\) that is  \( dF_p \) refers to the derivation in \( T_p N \) satisfying
for \( f \in C^k(N)\) and derivation \( (v : C^k(M) \to \mathbb{R}) \in T_p M \)
\( dF_p(v)(f) = v(f \circ F) \)
since \( f \circ F \in C^k(M) \), so \( v(f \circ F) \) is well defined.

Properties of Differentials. 
Let \(M\), \(N\), and \(P\) be smooth manifolds with or without boundary, let \(F: M \to N\) and \(G: N \to P\) be smooth maps, and let \(p \in M\).
(a) \(dF_p: T_pM \to T_{F(p)}N\) is linear.
(b) \(d(G \circ F)_p = dG_{F(p)} \circ dF_p: T_pM \to T_{G \circ F(p)}P\).
(c) \(d(\text{Id}_M)_p = \text{Id}_{T_pM}: T_pM \to T_pM\).
(d) If \(F\) is a diffeomorphism, then \(dF_p: T_pM \to T_{F(p)}N\) is an isomorphism, and \((dF_p)^{-1} = d(F^{-1})_{F(p)}\).

With that its also relatively easy to obtain the lecture definition (which is again dependent on local charts).
For any chart \( \phi : U \to \mathbb{R}^m \) we have that \( \{\frac{d}{dx^i}|_{\phi(p)}\}_{i \in [m]} \) forms a basis of \( T_{\phi(p)} \mathbb{R}^m \)
and as \( \phi \) is a diffeomorphism \( d\phi_p : T_p M \to T_{x(p)} \mathbb{R}^m \) is a isomorphism between and for any \( f \in C^k(U) \)
we can define the action of a basis vector \( \frac{d}{d\tilde{x}^i}|_p = (d\phi_p)^{-1}(\frac{d}{dx^i}|_{\phi(p)}\) of \( T_p M \) via
\( \frac{d}{d\tilde{x}^i}|_p f = \frac{d}{dx^i}|_{\phi(p)}(f \circ \phi^{-1}) \).
More generally if \( v = v_i \frac{d}{d\tilde{x}^i} \in T_p M \) and \( F : M \to N \) then
\( dF_p(v) = dF_p(v_i \frac{d}{d\tilde{x}^i}) = v_i dF(\frac{d}{d\tilde{x}^i}) = v_i \frac{dF^j}{d\tilde{x}^i}\frac{d}{d\tilde{y^j}} \)
Note that we are still using the Einstein sum notaiton so 
\( v_i \frac{dF^j}{d\tilde{x}^i}\frac{d}{d\tilde{y^j}} = \sum_i^m \sum_j^n v_i \frac{dF^j}{d\tilde{x}^i}\frac{d}{d\tilde{y^j}} \)
since \(dF_j \) generally does not map one basis vector \( \frac{d}{d\tilde{x}^i} \) to some \( \frac{d}{d\tilde{y}^i} \) but rather
a linear combination of the \( \frac{d}{d\tilde{y}^i} \). In particular we can see that this is nothing but the Jacobi matrix in disguise
with each column encoding how \( \frac{d}{d\tilde{x}^i} \) translates into \( \frac{d}{d\tilde{y}^i} \).

so we can simply use 
\( \{\frac{d}{dx^i}|_{x(p)}\) to describe the basis vectors of \( T_p M \).

kand we can rather inspect

Simply by noting that \(  \)

