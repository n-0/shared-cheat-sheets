Define the tangent space of a manifold

The tangent space is a linear (more rarely, affine) space of vectors tangent to a smooth (differentiable) manifold \( M \) at a given point \( a \in M \),
denoted by \( T_a M \)

For curves, surfaces and submanifolds embedded in a Euclidean subspace \( \mathbb{R}^n \) the tangent subspace can be identified with an affine subset 
(of the corresponding dimension \( 1,2,\dots \)) in the ambient space, passing through \( a \). 
For abstract manifolds \( T_aM \) is naturally isomorphic to the linear space of derivations \( D : C^\infty(M) \to \mathbb{R}^m \) of 

of the ring of smooth functions on \( M \) satisfying the Leibniz rule:
\(  D\in T_a M\iff D:C^\infty(M)\to\R\) 
\(D(f\pm g)=Df\pm Dg,\ D(\lambda f)=\lambda Df\) 
\(D(f\cdot g)=f(a)\cdot Dg+g(a)\cdot Df. \)

The main reason for the abstraction is that we do not have any ambient space in which we could define the vectors tangent to the manifold.
The (disjoint) union of tangent spaces to a manifold to all its points has a natural structure of a over \( M \), called the tangent bundle.

To make the relation between the affine space of vectors tangent to a point and the ring of derivations more concrete,
say \( M \subset \mathbb{R}^m \) and take any \( v \in \mathbb{R}_{a}^m \) (vectors tangent to the point \( a \)) 
and \( f \in C^{\infty}(\mathbb{R}^n) \) then we can build the directional derivative at \( a \) along \( v \) via

\( D_v(f)|_a = D_v f(a) = \lim_{h \to 0} \frac{f(a + hv) - f(a)}{h} = \frac{d}{dt} f(a + tv)|_{t=0} \)
this is linear and satisfies of course the Lebniz rule.
If we write \( v_a = v^i e_i|_a \) in a standard basis the chain rule gives
\( \frac{d}{dt} f(a + tv)|_{t=0} = v(\frac{df}{dx}(a + tv))|_{t=0} = v^i\frac{df}{dx^i}(a) \).

In fact one can prove that there is an isomorphism of \( v \in T_a M \) and derivations \( D_v \) at \( a \),
but the latter exists without any prior assumption of \( M \) being in \( \mathbb{R}^m \).

Conversely one can interpret \( D(f) : \mathbb{R}^m \to \mathbb{R}^n \) (as was it done in the lecture), although this requires
additional effort because \( \mathbb{R}^m \) actually refers here to a local parameterization of \( f \) via some chart
\( (\phi, U) \) and actually \( f : U \to \mathbb{R}^n \) such that \( D(f) \coloneqq D(\tilde{f} \) and \( \tilde{f} = f \circ \phi^{-1} \).

This causes then the issue of verifying compatibility with transition maps, that is our tangent vectors \( v \) are taken from the tangent space 
to \( \phi(a) \in \mathbb{R}^m \) and one has to show that if \( (\psi, V) \), \( a \in V \) and any \( f : U \to \mathbb{R}^n \)
parameterized via \( \phi \) by default and \( \tilde{f} \) its parameterization via \( \psi \):

\( Df(v)|_a = D(f \circ \phi^{-1})(v)|_a = D(f \circ \psi^{-1} \circ \psi \circ \phi^{-1})(v)|_a = D(\tilde{f} \circ \psi \circ \phi^{-1})(v)|_a \)
\( = D\tilde{f})(D \psi \circ \phi^{-1}(v))|_{\psi(a)}\)

so if \( Df(v) = D\tilde{f}(\tilde{v})\) then it must be \( \tilde{v} = D \psi \circ \phi^{-1}(v) \).
Thus the tangent space at \( a \) is here defined as the quotient space of all \( v \in T_{\phi(a)}, \tilde{v} \in T_{\psi} \)
s.t.   
\( \tilde{v} = D(\psi \circ \phi^{-1})(v)|_{a} \)


Define the differential of a map \( F : M \to N \) at \( p \)

Assuming that \( F \) is \( C^k \) differentiable at \( p \) the differential can be defined via the derivation construction
as the map \( dF_p : T_p M \to T_{F(p)} N\) that is  \( dF_p \) refers to the derivation in \( T_p N \) satisfying
for \( f \in C^k(N)\) and derivation \( (v : C^k(M) \to \mathbb{R}) \in T_p M \)
\( dF_p(v)(f) = v(f \circ F) \)
since \( f \circ F \in C^k(M) \), so \( v(f \circ F) \) is well defined.

Properties of Differentials. 
Let \(M\), \(N\), and \(P\) be smooth manifolds with or without boundary, let \(F: M \to N\) and \(G: N \to P\) be smooth maps, and let \(p \in M\).
(a) \(dF_p: T_pM \to T_{F(p)}N\) is linear.
(b) \(d(G \circ F)_p = dG_{F(p)} \circ dF_p: T_pM \to T_{G \circ F(p)}P\).
(c) \(d(\text{Id}_M)_p = \text{Id}_{T_pM}: T_pM \to T_pM\).
(d) If \(F\) is a diffeomorphism, then \(dF_p: T_pM \to T_{F(p)}N\) is an isomorphism, and \((dF_p)^{-1} = d(F^{-1})_{F(p)}\).

With that its also relatively easy to obtain the lecture definition (which is again dependent on local charts).
For any chart \( \phi : U \to \mathbb{R}^m \) we have that \( \{\frac{d}{dx^i}|_{\phi(p)}\}_{i \in [m]} \) forms a basis of \( T_{\phi(p)} \mathbb{R}^m \)
and as \( \phi \) is a diffeomorphism \( d\phi_p : T_p M \to T_{x(p)} \mathbb{R}^m \) is a isomorphism between and for any \( f \in C^k(U) \)
we can define the action of a basis vector \( \frac{d}{d\tilde{x}^i}|_p = (d\phi_p)^{-1}(\frac{d}{dx^i}|_{\phi(p)}\) of \( T_p M \) via
\( \frac{d}{d\tilde{x}^i}|_p f = \frac{d}{dx^i}|_{\phi(p)}(f \circ \phi^{-1}) \).
More generally if \( v = v_i \frac{d}{d\tilde{x}^i} \in T_p M \) and \( F : M \to N \) then
\( dF_p(v) = dF_p(v_i \frac{d}{d\tilde{x}^i}) = v_i dF(\frac{d}{d\tilde{x}^i}) = v_i \frac{dF^j}{d\tilde{x}^i}\frac{d}{d\tilde{y^j}} \)
Note that we are still using the Einstein sum notation so 
\( v_i \frac{dF^j}{d\tilde{x}^i}\frac{d}{d\tilde{y^j}} = \sum_i^m \sum_j^n v_i \frac{dF^j}{d\tilde{x}^i}\frac{d}{d\tilde{y^j}} \)
since \(dF_j \) generally does not map one basis vector \( \frac{d}{d\tilde{x}^i} \) to some \( \frac{d}{d\tilde{y}^i} \) but rather
a linear combination of the \( \frac{d}{d\tilde{y}^i} \). In particular we can see that this is nothing but the Jacobi matrix in disguise
with each column encoding how \( \frac{d}{d\tilde{x}^i} \) translates into \( \frac{d}{d\tilde{y}^i} \).

State how the basis vectors of the tangent space behave under a change of coordinates

As we have seen for the description of the differential, if \( \{\frac{d}{d\tilde{x}^i}\} \) is the basis
induced by the chart \( (U, \phi) \) via \( \frac{d}{d\tilde{x}^i} = d\phi^{-1}(\frac{d}{d\tilde{x}^i}) \in T_{\phi(p)} \mathbb{R}^k \) 
at the point \( p \) and \( \{\frac{d}{d\tilde{y}^i}\} \) for a chart \( (V, \psi) \) and for ease of notation let \( F = \psi \circ \phi^{-1} \)
then by the Jacobian representation of the differential

\(d_{\phi(p)}(\psi \circ \phi^{-1})(\frac{d}{dx^i}|_{\phi(p)} = \frac{dF^j}{dx^i}(\phi(p))\frac{d}{dy^j} \)

put even more succinct let be two local representations of the same tangent vector be \( v = v^i\frac{d}{dx^i} = = w^i\frac{d}{dy^i} \) 
(living in \(\mathbb{R}^m\)) 
we have that 
\( w^j = \frac{dF^j}{dx^i}(\phi(p))v^i \)
which is simply the sum of the entries in the \(j\)-th row of the Jacobian.

\( d_{\phi(p)}(\psi \circ \phi^{-1})(\frac{d}{dx^i}|_{\phi_p} = \frac{d\tilde{x}^j}{dx^i}(\phi(p))\frac{d}{d} \)

A vector bundle \(E\) over a base \(M\). 
A point \(m_1\) in \(M\) corresponds to the origin in a fibre \(E_{m_1}\) of the vector bundle \(E\), and this fibre is mapped down to the point \(m_1\) by the 
projection \(\pi: E \to M\).]]

Define a vector bundle

A real vector bundle consists of:
1. topological spaces \(X\) base space and \(E\) total space.
2. a continuous surjection \(\pi:E \to X\) bundle projection.
3. for every \(x\) in \(X\), the structure of a finite-dimensional real vector space on the fiber \(\pi^{-1}(\{x\}\))

where the following compatibility condition is satisfied: 
for every point \(p\) in \(X\), there is an open neighborhood \(U\subseteq X\) of \(p\), a natural number \(k\), and a homeomorphism
\(\varphi\colon \pi^{-1}(U) \to U \times \mathbb{R}^k\) 


such that for all \(x\) in \(U\) and \( v \in \pi^{-1}(x) \),
* \( \text{pr}_1 \circ \varphi(v) = \pi(v) = x \) for all vectors \(v\) in \(\mathbb{R}^k\)
* The restriction of \( \phi \) to \( \pi^{-1}(x) \) is a vector space isomorphism from \( \pi^{-1}(x) \) to \( \{q\}\times \mathbb{R}^k \simeq \mathbb{R}^k \)

Pictorially the following diagram commutes

We refer to \( (U, \phi) \) as  a local trivialization and intuitively says the locally the space \( E \) looks like \(U\times \mathbb{R}^k\).
If there exists a set \( \{(U_i, \phi_i)\} \) with such homeomorphism and \( U_i \) covering all of \( E \), the vector bundle is called trivial.

A generalization are fiber bundles, where we have a triple \( (E, X, \pi, F) \) where \( E \) is again a total space,
\( X \) the base space, but this time we do not require \( F = \mathbb{R}^k \) but any topological space, 
so we do not have necessarily a vector space structure locally.


Describe the special case of the Tangent bundle for vector bundles
and show that it is a manifold

Let the total space \( E = TM = \amalg_{p \in M} T_p\ M \) and the natural projection map \( \pi : TM \to M,\ \pi(v_p) = p \)
where \( v_p \) is a tangent vector at \( p \). The local trivialization is given by \( \phi : \pi^{-1}(U) \to U \times \mathbb{R}^m \)
via \( \phi(v_p) = (p, v) \).

For it to be a manifold, we begin by defining the maps that will become our smooth charts. 
Given any smooth chart \((U, \varphi)\) for \(M\), note that \(\pi^{-1}(U) \subset TM\) 
is the set of all tangent vectors to \(M\) at all points of \(U\). 
Let \((x^1, \dots, x^n)\) denote the coordinate functions of \(\varphi\), 
and define a map \(\tilde{\varphi} : \pi^{-1}(U) \to \mathbb{R}^{2n}\) by 
\(\tilde{\varphi} \left( \left. v^i \frac{\partial}{\partial x^i} \right|_p \right) = (x^1(p), \dots, x^n(p), v^1, \dots, v^n). \qquad (3.13)\)

Its image set is \(\varphi(U) \times \mathbb{R}^n\), which is an open subset of \(\mathbb{R}^{2n}\). 
It is a bijection onto its image, because its inverse can be written explicitly as
\(\tilde{\varphi}^{-1}(x^1, \dots, x^n, v^1, \dots, v^n) = \left. v^i \frac{\partial}{\partial x^i} \right|_{\varphi^{-1}(x)}.\)

Now suppose we are given two smooth charts \((U, \varphi)\) and \((V, \psi)\) for \(M\), 
and let \((\pi^{-1}(U), \tilde{\varphi}), (\pi^{-1}(V), \tilde{\psi})\) be the corresponding charts on TM. 

The sets 
\(\tilde{\varphi}(\pi^{-1}(U) \cap \pi^{-1}(V)) = \varphi(U \cap V) \times \mathbb{R}^n\) and
\(\tilde{\psi}(\pi^{-1}(U) \cap \pi^{-1}(V)) = \psi(U \cap V) \times \mathbb{R}^n\)

are open in \(\mathbb{R}^{2n}\), and the transition map 
\(\tilde{\psi} \circ \tilde{\varphi}^{-1}: \varphi(U \cap V) \times \mathbb{R}^n \to \psi(U \cap V) \times \mathbb{R}^n\) 
can be written explicitly using our knowledge of the differential for transition maps as
\(\tilde{\psi} \circ \tilde{\varphi}^{-1} (x^1, \dots, x^n, v^1, \dots, v^n) 
= \left( F^1(x), \dots, F^n(x), \frac{\partial F^1}{\partial x^j}(x) v^j, \dots, \frac{\partial F^n}{\partial x^j}(x) v^j \right).\),
where \( F = \tilde{\psi} \circ \tilde{\varphi}^{-1} \).
This is clearly smooth as the coordinate maps are and also

Choosing a countable cover \(\{U_i\}\) of \(M\) by smooth coordinate domains, 
we obtain a countable cover of \(TM\) by coordinate domains \(\{\pi^{-1}(U_i)\}\).
To check the Hausdorff condition (v), just note that any two points in the same fiber of \(\pi\) lie in one chart, 
while if \((p, v)\) and \((q, w)\) lie in different fibers, 
there exist disjoint smooth coordinate domains \(U, V\) for \(M\) such that \(p \in U\) and \(q \in V\), 
and then \(\pi^{-1}(U)\) and \(\pi^{-1}(V)\) are disjoint coordinate neighborhoods containing \((p, v)\) and \((q, w)\), respectively.

To see that \(\pi\) is smooth, note that with respect to charts \((U, \varphi)\) for \(M\) and \((\pi^{-1}(U), \tilde{\varphi})\) for \(TM\), 
its coordinate representation is \(\pi(x, v) = x\). The \( (x^1, \dots, x^n, v^1, \dots, v^n) \) are also referred to as natural coordinates.


Define the cotangent space

For any vector space \( V \) over a field \( \mathbb{K} \), we call linear functions \( f : V \to \mathbb{K} \)
covectors. The space under addition and multiplication is itself a vector space called the dual space \( V^\ast \).

We have in particular that for each finite dimensional space there is even an isomorphism \(\phi : V \to V^\ast \),
which is defined on the basis \( \{x_i\}_{i \in [\text{dim}\ V]} \) as \( \phi(x_i) = y_i \), s.t. \( y_i(x_j) = \delta_{i,j} \).

The cotangent space \( T^\ast_p\ M \) is then nothing else but the dual of the tangent space of a point.

We use the special notation \( dx^i \) for its basis vectors and similarly as for tangent vectors it holds for different
charts \( \phi, \psi \) inducing basis vectors \( d\tilde{x}^i \), \( d\tilde{y}^i \)

\( d\tilde{x}^k|_{\phi(p)} = \frac{dF^k}{d\tilde{x}^j}(\phi(p))dy^j|_{\phi(p)} \).


As a prerequisite note the following 

Good bye World!