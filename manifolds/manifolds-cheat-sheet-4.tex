Define a covering space

(This mainly played a role in the homework and is mostly related to the lecture through Lie groups)
Let \(X\) be a topological space. 
A covering of \(X\) is a continuous map \(\pi : \tilde{X} \rightarrow X\) such that for every \(x \in X\) there exists an open neighborhood \(U_x\) of \(x\) 
and a discrete (topology) space \(D_x\) such that 
\(\pi^{-1}(U_x)= \displaystyle \bigsqcup_{d \in D_x} V_d \) and 
\(\pi|_{V_d}:V_d \rightarrow U_x \) is a homeomorphism for every \(d \in D_x \).

\( X \) is then the base and \( \tilde{X} \)the covering space, the open sets \(V_{d}\) are called sheets, 
which are uniquely determined up to homeomorphism, if \(U_x\) is connected.
For each \(x \in X\) the discrete set \(\pi^{-1}(x)\) is called the fiber of \(x\). 
If \(X\) is connected, it can be shown that \(\pi\) is surjective, and the cardinality of \(D_x\) is the same for all \(x \in X\); 
this value is called the degree of the covering. 
If \(\tilde X\) is path-connected, then the covering \( \pi : \tilde X \rightarrow X\) is called a path-connected covering. 
This definition is equivalent to the statement that \(\pi\) is a locally trivial fiber bundle.

In the case of \( C^k \) manifolds it is typically preferred to also require that \( \pi \) is \( C^k \) and similarly the \( \pi|_{V_d} \).

If \( \tilde{X} \) is simply connected, then \( \pi \) is the so called universal cover, which is for unique in the sense
that any other universal cover \( (\hat{X}, \hat{\pi}) \) is diffeomorphic via some 
\( \Phi : \hat{X} \to \tilde{X} \) to the original one and \( \pi \circ \Phi  = \hat{\pi} \).
A non-trivial fact is that this object always exists for smooth connected manifolds.

We have the following facts
(a) Every smooth (\( C^\infty \)) covering map is a local diffeomorphism, a smooth submersion,
an open map, and a quotient map.
(b) An injective smooth covering map is a diffeomorphism.
(c) A topological covering map is a smooth covering map if and only if it is a local
diffeomorphism.

Some examples are:
\( r : \mathbb{R} \to S^1 \) via \( r(t) = (\cos(2\pi t), \sin(2\pi t)) \)

Consider mapping opposite points on a sphere to the same point, making \( S^2 \) a covering space of \( \mathbb{P}^2(\mathbb{R}) \)
of degree 2.


Define homeomorphisms of Lie Groups and give some examples

If \(G\) and \(H\) are Lie groups, a Lie group homomorphism from \(G\) to \(H\) is a smooth map \(F: G \to H\) that is also a group homomorphism. 
It is called a Lie group isomorphism if it is also a diffeomorphism, which implies that it has an inverse that is also a Lie group homomorphism.   
In this case we say that \(G\) and \(H\) are isomorphic Lie groups.

The map \(\varepsilon: \mathbb{R} \to S^1\) defined by \(\varepsilon(t) = e^{2\pi i t}\) is a Lie group homomorphism whose kernel is the set \(\mathbb{Z}\) of integers. 
Similarly, the map \(\varepsilon^n: \mathbb{R}^n \to T^n\) defined by \(\varepsilon^n (x^1, ..., x^n) = (e^{2\pi i x^1}, ..., e^{2\pi i x^n})\) 
is a Lie group homomorphism whose kernel is \(\mathbb{Z}^n\).

The determinant function \(\det: GL(n, \mathbb{R}) \to \mathbb{R}^*\) is smooth because \(\det A\) is a polynomial in the matrix entries of \(A\). It is a Lie group homomorphism because \(\det(AB) = (\det A) (\det B)\). Similarly, \(\det: GL(n, \mathbb{C}) \to \mathbb{C}^*\) is a Lie group homomorphism.

If \(G\) is a Lie group and \(g \in G\), conjugation by \(g\) is the map \(C_g: G \to G\) given by \(C_g (h) = ghg^{-1}\). 
Because group multiplication and inversion are smooth, \(C_g\) is smooth, and a simple computation shows that it is a group homomorphism. 
In fact, it is an isomorphism, because it has \(C_{g^{-1}}\) as an inverse. 
A subgroup \(H \subset G\) is said to be normal, if \(C_g(H) = H\) for every \(g \in G\). 

We have the facts: 
1. Every Lie group homomorphism has constant rank.
2. For any Lie homomorphism to prove that it is an isomorphism it suffices
to show that it is bijective (opposed to also showing diffeomorphic).

Define Lie subgroups

A Lie subgroup of \( G \) is a subgroup of \( G \) endowed with a topology and smooth structure 
making it into a Lie group and an immersed submanifold of \( G \).

We have the following standard case
Suppose \( H \subsetseq G \) is a subgroup (in the algebraic sense) 
that is also an embedded submanifold. Then \( H \) is a Lie subgroup.

Let \( F : G \to H \) be a Lie group homomorphism. The kernel of \( F \)
is a properly embedded Lie subgroup of \( G \), whose codimension is equal to the rank
of \( F \).
Proof:
Because \( F \) has constant rank, its kernel \( F^{-1}(e) \) is a properly embedded
submanifold of codimension equal to the rank \( F \).

If \( F : G \to H \) is an injective Lie group homomorphism, the image of 
\( F \) has a unique smooth manifold structure such that \( F(G) \) is a Lie subgroup
of \( H \) and \( F : G \to F(G) \) is a Lie group isomorphism.
Proof:
Since a Lie group homomorphism has constant rank, it follows from the
global rank theorem that \( F \) is a smooth immersion, so its smooth submanifold of \( H \). 
By injectivity its image is also an algebraic subgroup and thus a Lie subgroup. The rest follows.

Example
 The set \(SL(n, \mathbb{R})\) of \(n \times n\) real matrices with determinant equal to 1 is called the special linear group of degree \(n\). 
 Because \(SL(n, \mathbb{R})\) is the kernel of the Lie group homomorphism \(\det: GL(n, \mathbb{R}) \to \mathbb{R}^*\), 
 it is a properly embedded Lie subgroup. 
 Because the determinant function is surjective, it is a smooth submersion by the global rank theorem,
 so \(SL(n, \mathbb{R})\) has dimension \(n^2 - 1\).


Define group actions on manifolds

The left action of a Lie group \(G\) on a smooth manifold \(M\) is defined as the map \(\theta: G \times M \to M, (g, p) \mapsto g \cdot p\) 
such that the following holds:
(i) \(\forall g, g' \in G, \forall p \in M: g \cdot (g' \cdot p) = (gg') \cdot p\)
(ii) \(\forall p \in M: e \cdot p = p\)
We'll sometimes use \(\theta_g(p)\) instead of \(g \cdot p\) to make the map more explicit. 
With this notation the above requirements become for the left action
(i) \(\forall g, g' \in G: \theta_g \circ \theta_{g'} = \theta_{gg'}\)
(ii) \(\theta_e = \text{id}_M\)

A right action is defined exactly the same except that one has to modify the order of arguments due to \( \theta : M \times G, (p, g) \mapsto p \cdot g \).

If \(M\) is a topological space and \(G\) is a topological group, then the left/right action of \(G\) on \(M\) is said to be continuous, 
if the defining map is continuous. In this case we say that \(M\) is a left/right \(G\)-space. 
Similarly if \(M\) is a smooth manifold and \(G\) is a Lie group, then the left/right action is said to be smooth if the defining map is a smooth map.

For each \(p \in M\), the orbit of \(p\), denoted by \(G \cdot p\), is the set of all images of \(p\) under the action by elements of \(G\):
\(G \cdot p = \{g \cdot p : g \in G\}.\)
For each \(p \in M\), the isotropy group or stabilizer of \(p\), denoted by \(G_p\), 
is the set of elements of \(G\) that fix \(p\):
\(G_p = \{g \in G : g \cdot p = p\}.\)

The definition of a group action guarantees that \(G_p\) is a subgroup of \(G\).
The action is said to be transitive if for every pair of points \(p, q \in M\), there exists \(g \in G\) such that \(g \cdot p = q\), 
or equivalently if the only orbit is all of \(M\).   

The action is said to be free if the only element of \(G\) that fixes any element of \(M\) is the identity: 
\(g \cdot p = p\) for some \(p \in M\) implies \(g = e\), or equivalently if every isotropy group is trivial.   

Some examples are:
If \( G \) is any Lie group and \( M \) is any smooth manifold, the trivial action of \( G \)
on \( M \) is defined by \( g \cdot p = p \) for all \( g \in G \) and \( p \in M \). It is a smooth action,
for which each orbit is a single point and each isotropy group is all of G.

The natural action of \( \text{GL}(n, \mathbb{R}) \) on \( \mathbb{R}^n \) is the left action given by matrix 
multiplication: \( (A,x) \mapsto Ax \). It is smooth because the components of \( Ax \) depend polynomially on
the matrix entries of \( A \) and the components of \( x \). The orbits are \( \{0\}, \mathbb{R}^{n} \setminus \{0\} \)
due to invertibility of \( A \).

Every Lie group G acts smoothly on itself by left translation \( L_{g_1}(g_2) \), which is transitive and free.
 

Define deck transformations

(Not in the lecture but so ubiquitous in mathematics that I wanted to include it)

Let \( E, M \) be top. spaces and \( \pi : E \to M \) a covering map. An automorphism of \( \pi \),
referred to as deck transformation, is a homeomorphism \( \phi : E \to E \) satisfies \( \pi \circ \phi = \pi \).
Expressed in a diagram


The set \(\text{Aut}_\pi(E)\) of all automorphisms of \(\pi\), called the automorphism group of \(\pi\), is a group under composition, 
acting on \(E\) on the left. 
It can be shown that \(\text{Aut}_\pi(E)\) acts transitively on each fiber of \(\pi\) if and only if \(\pi\) is a normal covering map, 
which means that \(\pi_*(\pi_1(E, q))\) is a normal subgroup of \(\pi_1(M, \pi(q))\) for every \(q \in E\) (this is quite involved).

Suppose \( E, M \) are smooth manifolds and \( \pi : E \to M \) is a smooth covering map. 
With the discrete topology, the automorphism group \( \text{Aut}_{\pi}(E) \) is a zero-dimensional Lie group acting 
smoothly and freely on E.


Define Lie-Algebras and explain how they can be motivated by Lie groups

As already noted that a Lie group \( G \) defines a free, transitive group action on itself
via the left translation \( L : G \times G \to G,\ (g, g') \mapsto g\cdot g' \).

A vector field \( X \) on \( G \) is said to be left-invariant if for all \( g, g' \in G \)
\( d(L_g)|_{g'}(X_{g'} = X_{gg'} \)
that is we obtain just again our vector field shifted by \( g \). 
This is often abbreviated to \( (L_g)_{\ast}X = X \),
since \( (L_g)_{\ast}(aX + bY) = a(L_g)_{\ast}X + b(L_g)_{\ast}Y \) all such left-invariant vector fields
are a linear subspace of \( \mathfrak{X}(G) \), call it \( \text{Lie}(G) \).

Define the Lie Bracket for vector fields \( X, Y \in \mathcal{X}(G) \) as 
\( [X, Y] = XY - YX \)

Since pushforwards distribute over Lie Brackets we have
\( (L_{g})_{\ast}[X, Y] = [(L_{g})_{\ast}X,(L_{g})_{\ast}Y] = [X, Y]  \) hence the resulting v.f. \( [X, Y] \)
is also left-invariant.

This makes \( \text{Lie}(G) \) an algebra over \( \mathbb{R} \)
with the multiplication defined by \( [X, Y] \) satisfying the following,
for \( a, b \in \mathbb{R} \) we bilinearity
\([aX + bY, Z] = a[X, Z] + b[Y, Z]\)
\([Z, aX + bY] = a[Z, X] + b[Z, Y]\)
Antisymmetry:
\( [X, Y] = -[Y, X] \)
and the Jacobi identity
\( [X, [Y, Z]] + [Y, [Z, X]] + [Z, [X, Y]] = 0\)
(easy to remember as anti-clockwise rotation of the letters).
the Jacobi identity is a substitute for associativity, which does not hold
in general for brackets in a Lie algebra

Conversely in general, a Lie algebra is vector space with a multiplication called the bracket satisfying the above properties.

Further examples of Lie algebras are \( \mathcal(X)(G) \) itself or \( \text{Mat}_n(\\mathbb{R}) \).

