Define the integral curve of a vector field \( V \)
and prove its existence

If \( V \) is a vector field on \( M \) an integral curve of \( V \)
is a differentiable \( \gamma : J \to M \) with \( \gamma'(t) = V_{\gamma(t)} \)
for all \( t \in J \).

We then have for \( V \in \mathfrak{X}(M) \) with \( M \) differentiable
that

1. For each point \( p \in M \); there exist \( \varepsilon > 0 \) and a smooth curve \( \gamma_p : (-\varepsilon, \varepsilon) \to M \); 
that is an integral curve of V starting at p.
2. For each \( p \in M \), exists some open neighborhood of \( U \) and \( \varepsilon > 0 \)
s.t. for all \(q \in U\) the integral curve of 1. exists and 
\( \phi : (-\varepsilon, \varepsilon) \times U \to M, \ \phi(s,q) = \gamma_q(s) \) 
is smooth.

For a smooth coordinate domain \(U \subseteq M\), we can write a curve \(\gamma\) in local coordinates as
\[\gamma(t) = (\gamma^1(t), \dots, \gamma^n(t)).\] 
Then the condition \(\gamma'(t) = V_{\gamma(t)}\) for \(\gamma\) to be an integral curve of \(V\) can be written
\(\dot{\gamma}^i(t) \frac{\partial}{\partial x^i} \bigg|_{\gamma(t)} = V^i(\gamma(t)) \frac{\partial}{\partial x^i} \bigg|_{\gamma(t)},\)
which reduces to the following autonomous system of ordinary differential equations (ODEs):

\begin{align*}
\dot{\gamma}^1(t) &= V^1(\gamma^1(t), \dots, \gamma^n(t)), \\
&\vdots \\
\dot{\gamma}^n(t) &= V^n(\gamma^1(t), \dots, \gamma^n(t)). \tag{9.1}
\end{align*}

with initial condition \( \gamma(0)^i = x^i(p) \), where \( x^i \) is the \( i \)-th coordinate of some chart.
According to Picard–Lindelöf (which requires Lipschitz continuity of the \( V^i \) granted by the smoothness of \( V \))
this has then a unique smooth solution.

Moreover the theory of ODE's states on compact subsets ODE's vary smoothly on their initial conditions, hence
we can choose some smaller neighborhood \( U' \), which is relatively compact in \( U \) and obtain the second statement.


What are some useful properties of integral curves?

Lemma 9.3 (Rescaling Lemma). 
Let \(V\) be a smooth vector field on a smooth manifold \(M\), let \(J \subseteq \mathbb{R}\) be an interval, 
and let \(\gamma: J \to M\) be an integral curve of \(V\). 
For any \(a \in \mathbb{R}\), the curve \(\tilde{\gamma}: \tilde{J} \to M\) defined by \(\tilde{\gamma}(t) = \gamma(at)\) 
is an integral curve of the vector field \(aV\), where \(\tilde{J} = \{t: at \in J\}\).
Proof:
We can examine the action of \(\tilde{\gamma}'(t)\)
on a smooth real-valued function \(f\) defined in a neighborhood of a point \(\tilde{\gamma}(t_0)\). By
the chain rule and the fact that \(\gamma\) is an integral curve of \(V\),

\[\tilde{\gamma}'(t_0)f 
= \frac{d}{dt}\bigg|_{t=t_0} (f \circ \tilde{\gamma})(t)
= \frac{d}{dt}\bigg|_{t=t_0} (f \circ \gamma)(at) 
= a (f \circ \gamma)' (at_0) 
= a \gamma' (at_0) f 
= a V_{\tilde{\gamma}(t_0)} f.
\qquad \square
\]

Lemma 9.4 (Translation Lemma). Let \(V\), \(M\), \(J\), and \(\gamma\) be as in the preceding lemma. 
For any \(b \in \mathbb{R}\), the curve \(\hat{\gamma}: \hat{J} \to M\) defined by \(\hat{\gamma}(t) = \gamma(t + b)\) is also an integral curve of \(V\), 
where \(\hat{J} = \{t: t + b \in J\}\).


Define local flow/global flow and a one parameter subgroup
as well as complete vector fields.

The map \( \phi : (-\varepsilon, \varepsilon) \times U \to M \), for \( U \subset M \) with \( \phi(s,p) = \gamma_p(s) \),
and \( \gamma \) being the integral curve starting at \( p \), is called the local flow of \( V \) over \( U \).

Let \( A \) be an index set, \( \{U_{\alpha}\}_{\alpha \in A}\) an open cover of \( M \)
and \( \{I_{\alpha}\}_{\alpha \in A}\) an Interval in \( \mathbb{R} \) with \(0 \in I_{\alpha} \).
A family of maps \( \{\phi_{s}^\alpha\}_{\alpha \in U_{\alpha}, s \in I_{\alpha}} \)
with each being diffeomorphism \( \phi^\alpha_s : U_\alpha \to \phi^\alpha_{s}(U_{\alpha})\),
s.t. 
\( \phi^\alpha_s(p) = \gamma_p(s) \) (matches a local flow) 
with the group laws 
\( \phi^\alpha_s \circ ^\alpha_t = \phi^\alpha_{s + t}\) for \( s + t \in I_{\alpha} \)
\( \phi^\alpha_0 = \text{id}_{U_\alpha} \).
is called a local \( 1 \)-parameter group.

We call \( \mathcal{D}^\alpha = I_\alpha \times U_\alpha \) the flow domains.

A global flow is a diffeomorphism \( \phi : \mathbb{R} \times M \to M \), which is defined
for all \( t \in \mathbb{R} \) and its derivative fits a vector field at each point.

Vector fields for which we can find a global flow are called complete.



Prove that a compactly supported smooth vector field is complete.

(Uniform Time Lemma). 
Let \(V\) be a smooth vector field on a smooth manifold \(M\), and let \(\phi\) be its flow. 
Suppose there is a positive number \(\varepsilon\) such that for every \(p \in M\), 
the domain of \(\phi(p)\) contains \((-\varepsilon, \varepsilon)\). Then \(V\) is complete.

Proof. 
Suppose for the sake of contradiction that for some \(p \in M\), 
the domain \(D(p)\) of \(\phi(p)\) is bounded above. (A similar proof works if it is bounded below.) 
Let \(b = \sup D(p)\), let \(t_0\) be a positive number such that \(b - \varepsilon < t_0 < b\), and let \(q = \phi(p)(t_0)\).

The hypothesis implies that \(\phi(q)(t)\) is defined at least for 
\(t \in (-\varepsilon, \varepsilon)\). 
Define a curve \(\gamma : (-\varepsilon, t_0 + \varepsilon) \to M\) by 

\[\gamma(t) = \begin{cases}
\phi(p)(t), & -\varepsilon < t < b, \\
\phi(q)(t - t_0), & t_0 - \varepsilon < t < t_0 + \varepsilon.
\end{cases}\]

These two definitions agree where they overlap, because 
\(\phi(q)(t - t_0) = \phi_{t-t_0}(q) = \phi_{t-t_0} \circ \phi_{t_0} (p) = \phi_{t} (p) = \phi(p)(t)\) 
by the group law for \(\phi\). 
By the translation lemma, \(\gamma\) is an integral curve starting at \(p\). 
Since \(t_0 + \varepsilon > b\), this is a contradiction.

Theorem: Every compactly supported smooth vector field on a smooth manifold is complete.

Proof. 
Suppose \(V\) is a compactly supported vector field on a smooth manifold \(M\), and let \(K = \text{supp } V\). 
For each \(p \in K\), there is a neighborhood \(U_p\) of \(p\) and a positive number \(\varepsilon_p\) 
such that the flow of \(V\) is defined at least on \((-\varepsilon_p, \varepsilon_p) \times U_p\). 
By compactness, finitely many such sets \(U_{p_1}, \dots, U_{p_k}\) cover \(K\). 
With \(\varepsilon = \min \{\varepsilon_{p_1}, \dots, \varepsilon_{p_k}\}\), 
it follows that every maximal integral curve starting in \(K\) is defined at least on \((-\varepsilon, \varepsilon)\). 

Since \(V = 0\) outside of \(K\), every integral curve starting in \(M \setminus K\) is constant 
and thus can be defined on all of \(\mathbb{R}\). 
Thus the hypotheses of the uniform time lemma are satisfied, so \(V\) is complete.

Define the Lie-Derivative of vector fields

(Note that this a little simpler than the most general case 
for tensor fields)

Suppose we wanted to create a derivative in the direction \( v \in T_p M \)
of some smooth vector field \( W \), naively we would try to evaluate

\(D_v W(p) = \frac{d}{dt} \bigg|_{t=0} W_{p + tv} = \lim_{t \to 0} \frac{W_{p + tv} - W_p}{t}.\)

An easy calculation shows that \( D_v W(p) \) can be evaluated by applying \( D_v \) 
to each component of \( W \) separately:

\( D_v W(p) = D_v W^i(p) \frac{\partial}{\partial x^i} \bigg|_p. \)

This is however ill-defined since \( W_{p + tv}, W_p \) live in different tangent spaces.

This problem can be circumvented if we replace the vector \(v \in T_pM\) with a vector field \(V \in \mathfrak{X}(M)\), 
so we can use the flow of \(V\) to push values of \(W\) back to \(p\) and then differentiate. 

Suppose \(M\) is a smooth manifold, \(V\) is a smooth vector field on \(M\), and \(\theta\) is the flow of \(V\). 
For any smooth vector field \(W\) on \(M\), define a rough vector field on \(M\), 
denoted by \(\mathcal{L}_V W\) and called the Lie derivative of \(W\) with respect to \(V\), by

\((\mathcal{L}_V W)_p = \frac{d}{dt}\bigg|_{t=0} d(\theta_{-t})_{\theta_t(p)}(W_{\theta_t(p)})\) 
\(\lim_{t \to 0} \frac{d(\theta_{-t})_{\theta_t(p)}(W_{\theta_t(p)}) - W_p}{t}, \)

provided the derivative exists. For small \(t \neq 0\), at least the difference quotient makes sense: 
\(\theta_t\) is defined in a neighborhood of \(p\), and \(\theta_{-t}\) is the inverse of \(\theta_t\), 
so both \(d(\theta_{-t})_{\theta_t(p)}(W_{\theta_t(p)})\) and \(W_p\) are elements of \(T_pM\).
Should \( M \) have a boundary this definition only works if \( V \) is tangent to \( \partial M \)
otherwise a more involved construction is required.


Insert here image from page 229 (Fig. 9.13)

Prove that the Lie Derivative is a smooth vector field along some other vector field

If we consider this derivative at each point of \( M \) in the direction given by some smooth
vector field \( V \), \( (\mathcal{L}_v W) \) is in fact again a vector field.

Let \( \theta \) be the flow of \( V \). For any \( p \in M \) let \( (U, (x^i)) \)
be smooth chart containing \( p \). Choose an open interval \(0 \in J_0\) and open \(U_{0} \subseteq U\)
containing \(p\) such that \(\theta\) maps \(J_0 \times U_0\) into \(U\). 
For \((t, x) \in J_0 \times U_0\), write the component functions of \(\theta\) as 
\((\theta^1(t,x), \dots, \theta^n(t, x))\). Then for any \((t, x) \in J_0 \times U_0\), 
the matrix of \(d(\theta_{-t})_{\theta_t(x)} : T_{\theta_t(x)}M \to T_x M\) is

\(\left( \frac{\partial \theta^i}{\partial x^j}(-t, \theta(t, x)) \right).\)

Therefore,

\(d(\theta_{-t})_{\theta_t(x)}(W_{\theta_t(x)})\)
\(= \frac{\partial \theta^i}{\partial x^j}(-t, \theta(t, x)) W^j(\theta(t, x)) \frac{\partial}{\partial x^i} \bigg|_x.\)

Because \(\theta^i\) and \(W^j\) are smooth functions, the coefficient of \(\frac{\partial}{\partial x^i}|_x\) depends smoothly on \((t, x)\). 
It follows that \((\mathcal{L}_V W)_x\), which is obtained by taking the derivative of this expression with respect to \(t\) 
and setting \(t = 0\), exists for each \(x \in U_0\) and depends smoothly on \(x\).

Prove that the the Lie-Derivative is the Lie bracket for smooth vector fields

Let \(W = W^k \frac{\partial}{\partial x^k}\), \(V = V^l \frac{\partial}{\partial x^l}\) (sum over \(k\) or \(l\), respectively) 
be the local representations of \(V\) and \(W\). Then:

\(\mathcal{L}_V W = \frac{\partial}{\partial s}\bigg|_{s=0} \left( \phi_s^* (W^k \frac{\partial}{\partial x^k} ) \right) \)
\(= \frac{\partial}{\partial s}\bigg|_{s=0} \left( (\phi_{-s})_* (W^k \frac{\partial}{\partial x^k} ) \right)\)
\(= \frac{\partial}{\partial s}\bigg|_{s=0} \left( W^k \circ \phi_{-s} \frac{\partial \phi_{-s}^l}{\partial x^k} \frac{\partial}{\partial x^l} \right)\)
\(= \frac{\partial W^k}{\partial x^l} V^l \frac{\partial}{\partial x^k} + W^k \cdot \left( \frac{\partial V^l}{\partial x^k} \frac{\partial}{\partial x^l} \right)\)
\(= \left( V^l \frac{\partial W^k}{\partial x^l} - W^l \frac{\partial V^k}{\partial x^l} \right) \frac{\partial}{\partial x^k}\)
\(= [V, W].\)

Here we used the chain rule, \(\phi_0 = \text{Id}_M\) and \(\frac{\partial}{\partial s}\bigg|_{s=0} \phi_{-s} = -V^l\)


State the theorem of commuting flows

Let \( V, W \in \mathfrak{X}(M) \) be complete.
Then tfoa equivalent
1. \( [V, W] = 0 \)
2. The flows \( \Phi^V_s, \Phi^W_s \) are commuting, meaning \( \Phi^V_s \circ \Phi^W_s = \Phi^W_t \circ \Phi^V_s \).

(Not sure if the steps of the proof should be dissected because its huge and tedious)


Generalize the Lie Derivative to tensor fields

Let \( p \in M \) \( V \in \mathfrak{X}\) and \( \theta \) its flow, if \(t\) is sufficiently close to zero, 
then \(\theta_t\) is a diffeomorphism from a neighborhood of \(p\) to a neighborhood of \(\theta_t(p)\), 
so \(d(\theta_t)_p^*\) pulls back tensors at \(\theta_t(p)\) to ones at \(p\)
by the formula

\(d(\theta_t)_p^* (A_{\theta_t(p)})(v_1, \dots, v_k) = A_{\theta_t(p)} (d(\theta_t)_p(v_1), \dots, d(\theta_t)_p(v_k)).\)

Note that \(d(\theta_t)_p^* (A_{\theta_t(p)})\) is just the value of the pullback tensor field \(\theta_t^* A\) at \(p\).

Given a smooth tensor field \(A \in \Gamma(T^{(k,l)}M)\) on \(M\), we define the Lie derivative of \(A\)
with respect to \(V\), denoted by \(\mathcal{L}_V A\), by

\((\mathcal{L}_V A)_p = \frac{d}{dt}\bigg|_{t=0} (\theta_t^* A)_p 
= \lim_{t \to 0} \frac{d(\theta_{-t})_{\theta_t(p)}^* (A_{\theta_t(p)}) - A_p}{t}\)

provided the derivative exists.
Because the expression being differentiated lies in \(T^k (T_p^* M)\) for all \(t\), 
\((\mathcal{L}_V A)_p\) makes sense as an element of \(T^k (T_p^* M)\).

The Lie Derivative is a connection and sends tensors to tensors of the same type.
