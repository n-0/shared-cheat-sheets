Define differential, symmetric forms and the exterior derivative with some properties:

Let \(M\) be a differentiable manifold of dimension \(m\).

1. For \(0 \leq r \leq m\) let \(\bigwedge^r T^*M\) be the vector bundle with fibers \(\bigwedge^r T_p^*M\). 
It has the rank \(\binom{m}{r}\). Furthermore, we set 

\(\Omega^r(M) := \Gamma \left( \bigwedge^r T^*M \right)\)

and call the elements \(\eta \in \Omega^r(M)\) differential forms of degree \(r\) or also simply \(r\)-forms. 
Obviously, \(\Omega^0(M) = \Gamma(M \times \mathbb{R}) = C^\infty(M)\) holds, that is, 0-forms are smooth functions.

2. \(V^r(T^*M)\) is the vector bundle with fibers \(V^r T_p^*M\). For \(r \in \mathbb{N}_0\) let
\(\text{Sym}^r(M) := \Gamma \left( V^r T^*M \right).\)
Elements \(h \in \text{Sym}^r(M)\) are called symmetric \(r\)-linear forms on \(M\).

Let  \( d : \Omega^r(M) \to \Omega^{r+1}(M) \) be defined in local coordinates as
\( d( \sum_{J} \omega_J dx^J ) = \sum_{J} d\omega_J \wedge dx^J \)
in more detail
\( d(\sum_J \omega_J dx^{j_1} \wedge \dots \wedge dx^{j_k}) = \sum_{J} \sum_{i} \frac{\partial \omega_J}{\partial x^i} dx^i \wedge dx^{j_1} \dots \wedge dx^{j_k} \)
(that is the derivative \( \frac{\partial \omega_J}{\partial x^i} \) replaces \( \omega_J \) as the smooth coefficient of the elementary forms \( dx^{j_1} \wedge \dots \wedge x^{j_k} \)
and wedges \( x^i \) in front of it)
and one can even show that
\( d\omega(V_0, V_1, \dots, V_m) =\)
\(\sum_{k=0}^r (-1)^k V_k(\omega(V_0, \dots, \hat{V}_k, \dots, V_r)) + \sum_{0 \leq i < j \leq r} (-1)^{i+j}\omega([V_i, V_j], V_0, \hat{V}_k, \dots, V_r) \).


To prove its existence one observes that both sides are skew-symmetric and that for each \( f \in C^\infty(M) \)
\( d\omega(fV_0, \dots, V_m) = fd\omega(V_0, V_1, \dots, V_r) \)

Some special cases are \( 0 \)-forms \( f : M \to \mathbb{R} \), then \( df \in \Omega^1(M) \) with
\( df(V) = Vf, \forall V \in \mathfrak{X}(M) \)
and in local coordinates \( df =  \frac{\partial f}{\partial x^k} dx^k \)

Its naturally linear
and for \( \nu \in \Omega^p(M),\ \sigma \in \Omega^{q}(M)\)
\( d(\nu \wedge \sigma) = d\nu \wedge \sigma + (-1)^p\nu\wedged\sigma \)
\( d^2 = 0 \)

Give an alternative expression for the ext. derivative of a \( 1 \)-form

For any smooth 1-form \(\omega\) and smooth vector fields \( X \) and \( Y \), 
\(
d\omega(X,Y) = X(\omega(Y)) - Y(\omega(X)) - \omega([X, Y]). \hspace{1cm} (14.28)
\)
Proof.
Since any smooth 1-form can be expressed locally as a sum of terms of the form $u \, dv$ for smooth functions $u$ and $v$, it suffices to consider that case. 
Suppose \(\omega = u \, dv\), and \(X, Y\) are smooth vector fields. Then the left-hand side of (14.28) is

\[
\begin{aligned}
d(u \, dv)(X,Y) &= du \wedge dv(X,Y) = du(X) dv(Y) - dv(X) du(Y) \\
&= Xu Yv - Xv Yu. 
\end{aligned}
\]

The right-hand side is

\[
\begin{aligned}
X(u \, dv(Y)) - Y(u \, dv(X)) - u \, dv ([X, Y]) &= X(u Yv) - Y(u Xv) - u [X, Y] v \\
&= (Xu Yv + uXYv) - (Yu Xv + uYXv) - u(XYv - YXv). 
\end{aligned}
\]

After the two \(uXYv\) terms and the two \(uYXv\) terms are canceled, this is equal to the left-hand side.




\( (\iota_v dx_1 \wedge \dots \wedge dx_p)(X) = \sum^{p}_{r=1} (-1)^{r-1}f_r dx_1 \wedge \dots \wedge \hat{dx}_r \wedge \dots \wedge dx_p \)

Define the interior product and state some properties

In the most abstract setting \( \iota_V : \Lambda^k(V^\ast) \to \Lambda^{k-1}(V^\ast) \)
sends a \( p \)-form \( \omega \) to another \( p-1 \) form \( (\iota_V \omega) \)satisfying
\( (\iota_V\omega)(w_1, w_2, \dots, w_{p-1}) = \omega(v, w_1, \dots, w_{p-1})  \)
more concretely in local coordinates for covectors \( \omega^1, \dots \omega^k \) 
and a v.f. \( X = f_i \frac{\partial}{\partial x^i} \)

\( (\iota_V \omega^1 \wedge \dots \wedge \omega^p) = \sum^{p}_{r=1} (-1)^{r-1} \omega^i(v) \omega^1 \wedge \dots \wedge \hat{\omega}^r \wedge \dots \wedge d\omega^p \)

It thus satisfies
\( \iota_V \circ \iota_V = 0 \)
and for \( \omega \in \Lambda^k(V^\ast), \nu \in \Lambda^l(V^\ast)\)
\( \iota_V(\omega \wedge \nu) = (\iota_V \omega) \wedge \nu + (-1)^k \omega \wedge (\iota_V \nu) \)

Especially useful in the context of Cartans magic formula is that for \( \omega \in \Omega^1(M) \)
\( \iota_V \omega = \omega(V) \)
which means for \( f \in C^\infty(M) \) (0-form) and the ext. derivative
\( \iota_V df = Vf \).

Often the operation is simply abbreviated to
\( v \lrcorner \omega = \iota_V \omega \).


Elaborate how pullbacks on forms and exterior derivatives interact with each other

For \( F : M \to N \) smooth and \( \omega \in \Omega^k(N) \) the pullback 
is just the usual
\( (F^\ast \omega)_p(v_1, \dots, v_k) = \omega_{F(p)}(dF_p(v_1), \dots, dF_p(v_k)) \)

it satisfies linearity and
\( F^\ast(\omega \wedge \nu) = (F^\ast \omega) \wedge (F^\ast \nu) \)
and in any smooth chart \( \{y_i\}_{i \in [k]} \)
\( F^\ast(\sum_{I} \omega_I dy^{i_1}, \wedge \dots \wedge dy^{i_k} = \sum_{I} (\omega_I \circ F) d(y^{i_1} \circ F) \wedge \dots \wedge d(y^{i_k} \circ F) \).
note that the \( d \) stands here for the differential interpreted as a covector.

The last part has in particular the consequence that for charts \( (U, \{x^i\}_{i \in [k]}), (V, \{y^i\}_{i \in [k]}) \)
on \( M \) and \( N \) respectively and \( u \in C^{0}^(V) \) on \( U \cap F^{-1}(V) \) it holds
\( F^\ast (u dy^1 \wedge \dots \wedge dy^k) = (\u \circ F)(\text{det} DF)dx^1 \wedge \dots \wedge dx^k \),
where \( DF \) is the Jacobian of \( F \) with respect to these charts.

(This is Proposition 14.20 in Lee, if we want to include the proofs later on)

Finally for the exterior derivative \( d : \Omega^{k-1}(M) \to \Omega^{k}(M)\) we have
\( F^\ast(d\omega) = d(F^\ast\omega) \).


Prove Cartan's Magic formula

Note that for \( V \in \mathfrak{X}(M) \) and \( \omega, \nu \in \Omega^\ast(M) \)
it holds
\( \mathcal{L}_{V}(\omega \wedge \nu) = (\mathcal{L}_{V} \omega) \wedge \nu + \omega \wedge (\mathcal{L}_{V}\nu) \)

We then state Cartan's Magic formula
On a smooth manifold \( M \) for any smooth vector field \( V \) and any smooth differential form \( \omega \)
it holds
\( \mathcal{L}_{V}(\omega) = V \lrcorner(d\omega) + d(V \lrcorner \omega) \qquad (14.32) \).

Proof:
We prove that (14.32) holds for smooth \(k\)-forms by induction on \(k\). 
We begin with a smooth 0-form \(f\), in which case
\(\iota_V (df) + d(\iota_V f) = \iota_V df = df(V) = Vf = \mathfrak{L}_V f,\)
which is (14.32).

Now let \(k \geq 1\), and suppose (14.32) has been proved for forms of degree less 
than \(k\). Let \(\omega\) be an arbitrary smooth \(k\)-form, written in smooth local coordinates as
\(\omega = \sum_I' \omega_I dx^{i_1} \wedge \dots \wedge dx^{i_k}.\)
Writing \(u = x^{i_1}\) and \(\beta = \omega_I dx^{i_2} \wedge \dots \wedge dx^{i_k}\), we see that each term in this sum 
can be written in the form \(du \wedge \beta\), where \(u\) is a smooth function and \(\beta\) is a
smooth \((k-1)\)-form. The special case of smooth functions \(\mathfrak{L}_V du = d(\mathfrak{L}_V u) = d(Vu)\) was shown somewhere else before. 
Thus the product rule of the Lie derivative with the exterior product and the induction hypothesis imply
\(\mathfrak{L}_V (du \wedge \beta) = (\mathfrak{L}_V du) \wedge \beta + du \wedge (\mathfrak{L}_V \beta) \)
\(= d(Vu) \wedge \beta + du \wedge (\iota_V d\beta + d(\iota_V \beta)). \qquad (14.33) \)

On the other hand, using the facts that both \(d\) and interior multiplication by \(V\) are
antiderivations, and \(\iota_V du = du(V) = Vu\), we compute
\(\iota_V d(du \wedge \beta) + d(\iota_V (du \wedge \beta))\)
\(= \iota_V (-du \wedge d\beta) + d((Vu)\beta - du \wedge (\iota_V \beta))\)
\(= -(Vu)d\beta + du \wedge (\iota_V d\beta) + d(Vu) \wedge \beta + (Vu)d\beta + du \wedge d(\iota_V \beta).\)
After the \((Vu)d\beta\) terms are canceled, this is equal to (14.33). \qquad \square

A useful Corollary is \( \mathcal{L}_V(d\omega) = d(\mathcal{L}_V \omega) \)
which follows by directly substituting Cartan's magic formula and \( d^2 = 0 \).


Define the deRham-Cohomology with the preliminaries of exact/closed forms.

We call a \( p \)-form \( \nu \in \Omega^p(M) \) closed, if
\( d\nu = 0 \)
We call a \( p \)-form \( \nu \in \Omega^p(M) \) exact, if
there is a \( p-1 \) form \( \sigma \Omega^{p-1}(M) \) with
\( \nu = d\sigma \).


A \( 1 \)-form \( \nu = \nu_i dx^i \) is closed iff. 
\( \frac{\partial \nu_i}{\partial x^j} = \frac{\partial \nu_j}{\partial x^i} \)
for all \( i,j \)

If \( \nu \in \Omega^p(M), \sigma \in \Omega^q(M) \) are closed then \( \nu \wedge \sigma \)
is as well since
\( d(\nu \wedge \sigma) = d\nu \wedge \sigma + (-1)^p \nu \wedge d\sigma = 0 + 0 = 0 \)

If \( \nu \in \Omega^p(M) \) is exact and \( \omega \in \Omega^q(M) \) closed
then \( \nu \wedge \omega \) is closed as well, since
\( d(\sigma \wedge \omega) = d\sigma \wedge \omega + (-1)^{p-1}\sigma \wedge d\omega = d\sigma \wedge \omega = \nu \wedge \omega \)

Define \( Z^P(M) \subset \Omega^p(M) \) has the space of closed forms (cocycles)
and \( B^p(M) \subset \Omega^p(M) \) the space of exact forms (coboundaries),
then \( B^p(M) \subset Z^p(M) \).

Define on \( Z^P(M) \) the equivalence relation \( \nu_1 \sim \nu_2 \) if \( \nu_1 - \nu_2 = d\sigma \)
for some \( \sigma \in \Omega^p(M) \) that is exact. We can then define the quotient space
\( H^p_{\text{dR}} \coloneqq Z^p(M)/B^p(M) \)
better known as the \( p \)-th deRham cohomology, which is an \( \mathbb{R} \)-vector space.

For compact manifolds this space has finite dimensions which is also referred to as the \( p \)-th
betti number, denoted by \( b_p \coloneqq \text{dim}\ H^p_{\text{dR}}(M)\).

Some examples:

If \( M \) is connected then \( H^0_{\text{dR}}(M) \simeq \mathbb{R} \)

Since \( \Omega^0(M) \simeq C^\infty(M) \) and one assumes \( \Omega^{-1}(M) = \{0\} \)
the only exact form (coboundary) is \( d(0) = 0 \). Further \( df = 0 \iff f(m) = c, \forall m \in M \)
thus \( H^0_{\text{dR}}(M) \simeq Z^0(M) / B^0(M) = Z^0(M) \simeq \mathbb{R} \).


State Poincare's Lemma

A star domain is a subset of \( S \subset \mathbb{R}^n \) such that there exists an \( s_0 \in S \),
for which all line segments \( \overline{s_0s} \) to other points \( s \) in \( S \), are also in \( S \).

Let \(M\) be a differentiable manifold, \(p \in M\) and \(x: U \to \Omega\) be a chart around \(p\) 
with a star-shaped set \(\Omega \subset \mathbb{R}^m\). 
Then for every closed \(q\)-form \(\omega \in \Omega^q(M)\), \(q \geq 1\)
\(\omega|_U = d\eta\)
with a (not uniquely determined) \((q-1)\)-form \(\eta \in \Omega^{q-1}(U)\). 
In particular, every closed \(q\)-form is locally exact and for the de Rham cohomology groups 
of star-shaped subsets \(\Omega \subset \mathbb{R}^m\) holds \(H^q_{dR}(\Omega) = \{0\}\) for all \(q \geq 1\).

In relation to algebraic topology this more or less the statement, that contractible spaces have the same cohomology
as a single point. Note that this can't be derived in this direction, because the Poincare lemma is used in the proof
that deRham cohomology and singular cohomology are equivalent on manifolds.


Prove that the de-Rham cohomology is homotopy invariant

More concretely this means for homotopic maps \( f,g : M \to N \) that the induced cochain maps 
\( f^\ast = f^\ast[\omega] = [f^\ast \omega] : H^\bullet(M) \to H^\bullet(N)\) are the same \( f^\ast = g^\ast \).

To this end consider the special map \( i_t : M \to M \times I,\ i_t(x) = (x, t) \)

We then have with the help of Cartans magic formula
that there exists a chain homotopy between \( i_0^\ast, i_1^\ast : \Omega^\ast(M \times I) \to \Omega^\ast(M)\)

Equivalently for each \(p\), we need to define a linear map \(h:\Omega^{p}(M\times I)\rightarrow\Omega^{p-1}(M)\)
such that
\(h(d\omega) + d(h\omega) = i_1^*\omega - i_0^*\omega. \qquad (17.4)\)
Proof:
Let \(s\) denote the standard coordinate on \(\mathbb{R}\), and let \(S\) be the vector field on \(M \times \mathbb{R}\)
given by \(S_{(q,s)} = (0, \partial / \partial s)|_{(q,s)}\) under the usual identification \(T_{(q,s)}(M \times \mathbb{R}) \cong T_qM \times T_s \mathbb{R}\). 
Given a smooth \(p\)-form \(\omega\) on \(M \times I\), define \(h\omega \in \Omega^{p-1}(M)\) by
\(h\omega=\int_{0}^{1}i_{t}^{*}(S \lrcorner \omega)dt.\)
More specifically, for any \(q \in M\), this means
\((h\omega)_{q}=\int_{0}^{1}i_{t}^{*}((S \lrcorner \omega)_{(q,t)})dt,\)

where the integrand is interpreted as a function of \(t\) with values in the vector space
\(\bigwedge^{p-1}(T^*_q M)\). On any smooth coordinate domain \(U \subseteq M\), the components of the
integrand are smooth functions of \((q,t) \in U \times I\). so the integral defines a smooth
\((p-1)\)-form on \(M\). We can compute \(d(h\omega)\) at any point by differentiating under
the integral sign in local coordinates, which yields

\(d(h\omega)=\int_{0}^{1}d(i_{t}^{*}(S \lrcorner \omega))dt\)
Therefore, using Cartan's magic formula, we obtain
\(h(d\omega)+d(h\omega)=\int_{0}^{1}(i_{t}^{*}(S \lrcorner d\omega)+d(i_{t}^{*}(S \lrcorner \omega)))dt\)
\(=\int_{0}^{1}(i_{t}^{*}(S \lrcorner d\omega)+i_{t}^{*}d(S \lrcorner \omega))dt\)
\(=\int_{0}^{1}i_{t}^{*}(\mathcal{L}_{S}\omega)dt. \qquad (17.5)\)

To simplify this last expression, we use the flow of \(S\) on \(M \times \mathbb{R}\). 
The flow is given explicitly by \(\theta_{t}(q,s)=(q,t+s)\), so \(S\) is complete. It
follows that we can write \(i_{t}=\theta_{t} \circ i_{0}\), and therefore by a previously derived identity for the Lie-Derivative 
\( \frac{d}{dt}|_{t = t_0}(\theta^\ast_t \omega)_p = (\theta^\ast_{t_0}(\mathcal{L}_V(\omega))_p \)
we have
\(i_{t}^{*}(\mathcal{L}_{S}\omega)=i_{0}^{*}(\theta_{t}^{*}(\mathcal{L}_{S}\omega))=i_{0}^{*}(\frac{d}{dt}(\theta_{t}^{*}\omega))=\frac{d}{dt}i_{0}^{*}(\theta_{t}^{*}\omega)=\frac{d}{dt}i_{t}^{*}\omega.\)

Inserting this into (17.5) and applying the fundamental theorem of calculus, we
obtain (17.4). \qquad \square.

We then have for arbitrary homotopic maps  \( F, G : M \to N \) that 
\( F^\ast = G^\ast \) since \( F = H \circ i_0,\ G = H \circ i_1 \)
so \( F^\ast (H \circ i_0)^\ast = i_1^\ast \circ H^\ast = G^\ast \).

Last this implies that manifolds of the same homotopy type have the same de-Rham cohomology,
by the same argument as in algebraic topology actually, which is 
that by assumption there are \( F : M \to N, G : N \to M \) with \( F \circ G \) homotopic to \( \text{id}_{M} \)
and \( G \circ F \) homotopic to \( \text{id}_{N} \) so \( F^\ast : H_{\text{dR}}^p(N) \to H_{\text{dR}}^p(M) \)
is an isomorphism.

Finally note that we can use the homotopy \( H(x, t) = s_0 + t(x - s_0) \) for our star shaped domain to obtain the Poincare Lemma. 
