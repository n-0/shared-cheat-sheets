Define Riemannian manifolds and their metric

(This concept was used quite often in the last section without introducing it)

Let \( M \) be a smooth manifold. A Riemannian metric on M
is a smooth symmetric covariant 2-tensor field on M, that is positive definite at each point.
More conretely its some \( g \in \Gamma(T^{(0,2)}(M)) \) with \( g(u,v) = g(v,u) \)
and \( g_p(u,v) > 0 \) for all non-zero \( u,v \in T_pM \).
The pair \( (M,g) \) is a Riemannian manifold.

Note that \( g \) is not necessarily a metric by the definition from Topology/Analysis
but it is under most conditions, which hold in most situations of interest.

In local coordinates we may write \( g = g_{i,j} dx^i \otimes dx^j \), which is in this 
context shortened to \( g_{ij}dx^idx^j \). This is the reason why its common to call
a \( m \times m \) matrix also a Riemman metric as it encodes the \( g_{ij} \).

A standard example is of course \( (\mathbb{R}^n, \bar{g}) \) with
\( \bar{g} = \delta_{ij}dx^idx^j \) and clearly \( \bar{g}_p(v,w) = \langle v, w \rangle \)
for \( v, w \in T_p \mathbb{R}^n \simeq \mathbb{R}^n \).

In fact we have that all smooth manifolds \( M \) admit a Riemannian metric,
by the following.

In each chart \( (U_{\alpha}, \phi_\alpha) \) there is a Riemannian metric
\( g_\alpha = \phi^\ast_{\alpha}\bar{g} \). As usual we can create a partition of unity
\( \{\psi_\alpha\}_{\alpha} \) subordinate to \( \{U_\alpha\}_{\alpha} \) s.t. we can
define
\( g = \sum_{\alpha} \psi_\alpha g_{\alpha} \)
which turns out to be a Riemannian metric on all of \( M \).

This enables various geometric notions for manifolds such as
The length/norm
\( |v|_g = \sqrt{g_p(v, v)} \)
Angles: The unique \( \theta \in [0, \pi] \) satisfying
\( \cos \theta = \frac{g_p(v, w)}{|v|_g|w|_g}\).
Orthogonality:
\( v \perp w \iff g_p(v,w) = 0 \).

And also the concept of distance, which is a little bit more tricky:
Given points \( p_1,p_2 \in M \) and a smooth curve segment \( \gamma : [a, b] \to M \)
with \( \gamma(a) = p_1,\ \gamma(b) = p_2 \) the length of \( \gamma \) is defined as 

\( L_g(\gamma) = \int_{a}^{b} |\gamma'(t)|_g dt \)
and the distance as 
\( d(p_1, p_2) \coloneqq \inf_{\gamma} L_g(\gamma) \).

Define the musical isomorphisms in the context of Riemannian manifolds

(This played a marginal role in the final lecture, so its not ncessary for the exam)

A Riemannian metric introduces an isomorphism \( \hat{g}: TM \to T*M \) via
\( \hat{g}(v)(w) = g_p(v,w)  = g_ij V^iW^j \).
It is customary to write for a vector field \( X \in \mathfrak{X}(M) \)
\( \hat{g}(X) = X_j dx^j \), where \( X_j = g_{ij}X^i \),
thus one says \( \hat{g} \) lowers the index of \( X \) which is also written
as \( X^\flat \).

Since \( \hat{g} \) is an isomorphism it has an inverse \( \hat{g}^{-1} : T^\ast M \to TM \)
whose entries in the matrix representation we denote by \( g^{ij} \).
Then 
\( \hat{g}^{-1}(\omega) = \omega^i \frac{\partial}{\partial x^i} \), where \( \omega^i = g^{ij} \omega_j \)
and one says that \( \hat{g}^{-1} \) raises the index of \( \omega \), also termed \( \omega^{\sharp} \).

A helpful menmonic is to note that \( \omega^\sharp \) refers to a (sharp) tangent vector,
while \( X^\flat \) is a (flat) covector.

One can encode a great deal of information with this construction quite compactly
for example \( \text{grad} f = (df)^\sharp \).


Define the orientation of a manifold

Given an \( n \)-dimensional vector space \( V \) and ordered bases \( \{E_j\}_{j \in [n]}, \{\tilde{E}_j\}_{j \in [n]} \),
we say that the bases are consistently orientated if the transition matrix \( B^j_i \) defined by
\( E_i = B^j_i \tilde{E}_j \) has positive determinant.

As it turns out this defines an equivalence relation on the set of all bases for \( V \), which has precisely 2 equivalence classes.
After picking one of the bases as the standard basis \( \{E_j\}_{j \in [n]} \), we can consistenly say that another basis 
\(\{\tilde{E}_j\}_{j \in [n]}\) is positively orientated or negatively depending on if its in the same equivalence class.

In the euclidean case it is custom to choose the standard basis as the columns of the identity matrix in the ascending order. 

We can choose for each \( p \in M \) an orientation of \( T_p M\) called a pointwise orientation
A local frame \( \{\sigma_j\}_{j \in [n]} \in \bigtimes^n \Gamma(TM) \) can then referred to as positively/negatively orientated 
with respect to this pointwise orientation. If each point is in the domain of an orientated local frame then the pointwise orienation is called continuous.
Finally an orientation for \( M \) is a continuous pointwise orientation.

Interestingly we can somewhat easier define an orientation then via a nonvanishing \( n \)-form \( \omega \) on \( M \)
and conversely each such \( \omega \) induces an orientation on \( M \).

To see this simply note that \( \sigma_1(p), \sigma_2(p), \dots, \sigma_n(p) \) defines a dual-frame
\( \epsilon^i(p) \) for each \( p \) and we can define \( \omega = f \epsilon^1 \wedge \epsilon^2 \dots \wedge \epsilon^n \)
for some \( f \in C^\infty(M) \), which needs to be non-zero and always positive or negative by connectednes and this can actually be influenced
by replacing \( \epsilon^1 \) with \( - \epsilon^1 \), if necessary. Note that a partition of unity argument is required to make this work
on all of \( M \), since we worked only on the local representation of the frames.

We call \( \omega \) an orientation form of \( M \), which has a lot of connections 
to the theory of integration and Frobenius theorem.

Observe that all of this is consistent with the initial definition in the lecture, which we call
an orientation determined by a coordinate atlas. Namely each chart determines an orientation of \( T_pM \)
via the standard orientation of \( \mathbb{R}^n \) and to be consistent two bases for the same vector space
(the preimages of the standard basis via different charts on their intersection), must have a positive determinant
on their transition matrix \( B^j_i \sim \psi \circ \phi^{-1} \).

Explain how submanifolds (and with that boundaries) inherit an orientation

Call a vector field along \( S \), a section \( \sigma \in \Gamma(TM)|_S,\ \sigma : S \to TM \)
s.t. \( \sigma(p) \in T_p M \) for all \( p \in S \).

One then has the theorem:
Suppose \( M \) is an oriented smooth n-manifold with or without
boundary, \( S \) is an immersed hypersurface with or without boundary in \( M \); and \( \sigma \) is
a vector field along \( S \) that is nowhere tangent to \( S \). Then \( S \) has a unique orientation
such that for each \( p \in S, (E_1, \dots, E_{n-1} \) is an oriented basis for \( T_pS \) if and only if
\( (\sigma(p), E_1, \dots, E_{n-1}) \) is an oriented basis for \( T_p M \). If \( \omega \) is an orientation form for \( M \),
then \( \iota^\ast_S(\sigma \lrcorner\omega) \) is an orientation form for \( S \) with respect to this orientation, where
\( \iota_S : S \to M \) is the inclusion.

This creates the following hack to determine if a submanifold is orientable,
suppose \( S \subset M \)  is the level set of some \( f \in C^\infty(M) \). Then \( S \) orientable.
Proof:
Choose any Riemmanian metric on \( M \) and let \( \sigma = \text{grad} f|_S \). Then \( \sigma \)
is nowhere tangent to \( S \) and we're done.

Further any boundary \( \partial M \) is a hypersurface and there always exists at least one outward pointing vector field,
which is then nowhere tangent to \( \partial M \), so should \( M \) be orientable any orientation form \( \omega \) on \( M \) 
can be made one for \( \partial M \) by \( \iota^\ast_{\partial M}(\sigma \lrcorner \omega) \).

Orientated Riemmanian manifolds enjoy a canonical choice for their orientation form, elaborate 

It can be shown that there exists an orientation form \( \omega_g \in \Omega^n(M) \) called 
the Riemmanian volume form satisfying for each local orientated orthonormal frame \( (E_i) \)
\( \omega_g(E_1, \dots, E_n) = 1 \)
and can expressed locally as
\( \omega_g = \sqrt{\text{det}(g_{ij})} dx^1 \wedge \dots \wedge dx^n \).

This is useful for integration  as seen in Definition 7.1.25 and Lemma 7.1.16 in the lecture.


Define the integral of an \( m \)-form

Let M be a manifold, \(\epsilon > 0\) and \(x: U \to W_\epsilon\) be a chart around \(p \in U\) with
\[W_\epsilon := \{\xi \in \mathbb{R}^m : |\xi^i| < \epsilon, i = 1, \dots, m\} \]
and \(x(p) = 0\). Then we call \(x: U \to W_\epsilon\) a cubic coordinate system around \(p\) with radius \(\epsilon\).

Let \( \omega  \) be an \( m \)-form with \( \text{supp } \omega \subset U \) for some pos. orientated cubic coord. sys. \( x : U \to W_\varepsilon \),
s.t.
\( \omega = f dx^1 \wedge \dots \wedge x^m \)
we define the local integral of \( \omega \) as
\( \int_M \omega \coloneqq \int_{[\varepsilon, \varepsilon]} \dots \int_{[\varepsilon, \varepsilon]} f(x) dx^1 \dots dx^m \),
this is actually independent of particular choices of the charts since we know that for \( \phi = \tilde{x} \circ x^{-1} \)
\( \omega = f dx^1 \wedge \dots \wedge x^m = f(\phi(\tilde{x})) |\text{det}(D\phi)| d\tilde{x}^1 \wedge \dots \wedge d\tilde{x}^m \)
\( = \tilde{f} \wedge \dots \wedge d\tilde{x}^m\)
so by change of variables theorem
\( \int_{x(U \cap \tilde{U}} f(x) dx = \int_{\tilde{x}(U \cap \tilde{U}} f(\phi(\tilde{x})) |\text{det}(D\phi)| d\tilde{x} = \int_{\tilde{x}(U \cap \tilde{U}} \tilde{f} d\tilde{x} \)

We can globalize this notion for an \( \omega \) with compact \( \text{supp } \omega \). Let \( (f_\alpha, \{U_\alpha\}_{\alpha}) \) be a partition of unity 
with respect to some locally finite covering of \( M \), thus \( U_\alpha \cap \text{supp } \neq \emptyset \) but also always only finitely many \( \alpha \)
are involved then note that
\( \omega = \sum_\alpha f_\alpha \omega \)
so set \( \omega_\alpha = f_\alpha \omega \).

This is actually independent of the choice for the partition, which can be seen that for \( (U_\alpha, \omega_\alpha) \), \( (U_\beta, \omega_\beta) \)
we can define a new one \((U_\gamma, \omega_\gamma) \) with \( K = \bigcup_{\alpha, \beta} (\text{supp } \omega_\alpha \cap U_\alpha) \cup (\text{supp } \omega_\beta \cap U_\beta) \) and \( U_\gamma \) being a partition of unity of \( K \).
And we have
\( \int_M f_\gamma \omega = \sum_\alpha \int_M f_\gamma \omega_\alpha = \sum_\beta \int_M f_\gamma \omega_\beta \)
as well as
\( \sum_\alpha \int_M \omega_\alpha = \sum_\alpha \sum_\gamma \int f_\gamma \omega_\alpha = \sum_\gamma \sum_\alpha \int f_\gamma \omega_\alpha \)
\( = \sum_\gamma \int_M f_\gamma \omega_\beta = \sum_\beta \int_M \omega_\beta\)

Thus we can define uniquely

\( \int_M \omega \coloneqq \sum_\alpha \int_M \omega_\alpha \).

