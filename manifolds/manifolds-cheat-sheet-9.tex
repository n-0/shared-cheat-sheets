Define Riemannian manifolds and their metric

(This concept was used quite often in the last section without introducing it)

Let \( M \) be a smooth manifold. A Riemannian metric on M
is a smooth symmetric covariant 2-tensor field on M, that is positive definite at each point.
More conretely its some \( g \in \Gamma(T^{(0,2)}(M)) \) with \( g(u,v) = g(v,u) \)
and \( g_p(u,v) > 0 \) for all non-zero \( u,v \in T_pM \).
The pair \( (M,g) \) is a Riemannian manifold.

Note that \( g \) is not necessarily a metric by the definition from Topology/Analysis
but it is under most conditions, which hold in most situations of interest.

In local coordinates we may write \( g = g_{i,j} dx^i \otimes dx^j \), which is in this 
context shortened to \( g_{ij}dx^idx^j \). This is the reason why its common to call
a \( m \times m \) matrix also a Riemman metric as it encodes the \( g_{ij} \).

A standard example is of course \( (\mathbb{R}^n, \bar{g}) \) with
\( \bar{g} = \delta_{ij}dx^idx^j \) and clearly \( \bar{g}_p(v,w) = \langle v, w \rangle \)
for \( v, w \in T_p \mathbb{R}^n \simeq \mathbb{R}^n \).

In fact we have that all smooth manifolds \( M \) admit a Riemannian metric,
by the following.

In each chart \( (U_{\alpha}, \phi_\alpha) \) there is a Riemannian metric
\( g_\alpha = \phi^\ast_{\alpha}\bar{g} \). As usual we can create a partition of unity
\( \{\psi_\alpha\}_{\alpha} \) subordinate to \( \{U_\alpha\}_{\alpha} \) s.t. we can
define
\( g = \sum_{\alpha} \psi_\alpha g_{\alpha} \)
which turns out to be a Riemannian metric on all of \( M \).

This enables various geometric notions for manifolds such as
The length/norm
\( |v|_g = \sqrt{g_p(v, v)} \)
Angles: The unique \( \theta \in [0, \pi] \) satisfying
\( \cos \theta = \frac{g_p(v, w)}{|v|_g|w|_g}\).
Orthogonality:
\( v \perp w \iff g_p(v,w) = 0 \).

And also the concept of distance, which is a little bit more tricky:
Given points \( p_1,p_2 \in M \) and a smooth curve segment \( \gamma : [a, b] \to M \)
with \( \gamma(a) = p_1,\ \gamma(b) = p_2 \) the length of \( \gamma \) is defined as 

\( L_g(\gamma) = \int_{a}^{b} |\gamma'(t)|_g dt \)
and the distance as 
\( d(p_1, p_2) \coloneqq \inf_{\gamma} L_g(\gamma) \).

Define the musical isomorphisms in the context of Riemannian manifolds

(This played a marginal role in the final lecture, so its not ncessary for the exam)

A Riemannian metric introduces an isomorphism \( \hat{g}: TM \to T*M \) via
\( \hat{g}(v)(w) = g_p(v,w)  = g_ij V^iW^j \).
It is customary to write for a vector field \( X \in \mathfrak{X}(M) \)
\( \hat{g}(X) = X_j dx^j \), where \( X_j = g_{ij}X^i \),
thus one says \( \hat{g} \) lowers the index of \( X \) which is also written
as \( X^\flat \).

Since \( \hat{g} \) is an isomorphism it has an inverse \( \hat{g}^{-1} : T^\ast M \to TM \)
whose entries in the matrix representation we denote by \( g^{ij} \).
Then 
\( \hat{g}^{-1}(\omega) = \omega^i \frac{\partial}{\partial x^i} \), where \( \omega^i = g^{ij} \omega_j \)
and one says that \( \hat{g}^{-1} \) raises the index of \( \omega \), also termed \( \omega^{\sharp} \).

A helpful menmonic is to note that \( \omega^\sharp \) refers to a (sharp) tangent vector,
while \( X^\flat \) is a (flat) covector.

One can encode a great deal of information with this construction quite compactly
for example \( \text{grad} f = (df)^\sharp \).


Define the orientation of a manifold

Given an \( n \)-dimensional vector space \( V \) and ordered bases \( \{E_j\}_{j \in [n]}, \{\tilde{E}_j\}_{j \in [n]} \),
we say that the bases are consistently orientated if the transition matrix \( B^j_i \) defined by
\( E_i = B^j_i \tilde{E}_j \) has positive determinant.

As it turns out this defines an equivalence relation on the set of all bases for \( V \), which has precisely 2 equivalence classes.
After picking one of the bases as the standard basis \( \{E_j\}_{j \in [n]} \), we can consistenly say that another basis 
\(\{\tilde{E}_j\}_{j \in [n]}\) is positively orientated or negatively depending on if its in the same equivalence class.

In the euclidean case it is custom to choose the standard basis as the columns of the identity matrix in the ascending order. 

We can choose for each \( p \in M \) an orientation of \( T_p M\) called a pointwise orientation
A local frame \( \{\sigma_j\}_{j \in [n]} \in \bigtimes^n \Gamma(TM) \) can then referred to as positively/negatively orientated 
with respect to this pointwise orientation. If each point is in the domain of an orientated local frame then the pointwise orienation is called continuous.
Finally an orientation for \( M \) is a continuous pointwise orientation.

Interestingly we can somewhat easier define an orientation then via a nonvanishing \( n \)-form \( \omega \) on \( M \)
and conversely each such \( \omega \) induces an orientation on \( M \).

To see this simply note that \( \sigma_1(p), \sigma_2(p), \dots, \sigma_n(p) \) defines a dual-frame
\( \epsilon^i(p) \) for each \( p \) and we can define \( \omega = f \epsilon^1 \wedge \epsilon^2 \dots \wedge \epsilon^n \)
for some \( f \in C^\infty(M) \), which needs to be non-zero and always positive or negative by connectednes and this can actually be influenced
by replacing \( \epsilon^1 \) with \( - \epsilon^1 \), if necessary. Note that a partition of unity argument is required to make this work
on all of \( M \), since we worked only on the local representation of the frames.

We call \( \omega \) an orientation form of \( M \), which has a lot of connections 
to the theory of integration and Frobenius theorem.

Observe that all of this is consistent with the initial definition in the lecture, which we call
an orientation determined by a coordinate atlas. Namely each chart determines an orientation of \( T_pM \)
via the standard orientation of \( \mathbb{R}^n \) and to be consistent two bases for the same vector space
(the preimages of the standard basis via different charts on their intersection), must have a positive determinant
on their transition matrix \( B^j_i \sim \psi \circ \phi^{-1} \).

Explain how submanifolds (and with that boundaries) inherit an orientation

Call a vector field along \( S \), a section \( \sigma \in \Gamma(TM)|_S,\ \sigma : S \to TM \)
s.t. \( \sigma(p) \in T_p M \) for all \( p \in S \).

One then has the theorem:
Suppose \( M \) is an oriented smooth n-manifold with or without
boundary, \( S \) is an immersed hypersurface with or without boundary in \( M \); and \( \sigma \) is
a vector field along \( S \) that is nowhere tangent to \( S \). Then \( S \) has a unique orientation
such that for each \( p \in S, (E_1, \dots, E_{n-1} \) is an oriented basis for \( T_pS \) if and only if
\( (\sigma(p), E_1, \dots, E_{n-1}) \) is an oriented basis for \( T_p M \). If \( \omega \) is an orientation form for \( M \),
then \( \iota^\ast_S(\sigma \lrcorner\omega) \) is an orientation form for \( S \) with respect to this orientation, where
\( \iota_S : S \to M \) is the inclusion.

This creates the following hack to determine if a submanifold is orientable,
suppose \( S \subset M \)  is the level set of some \( f \in C^\infty(M) \). Then \( S \) orientable.
Proof:
Choose any Riemmanian metric on \( M \) and let \( \sigma = \text{grad} f|_S \). Then \( \sigma \)
is nowhere tangent to \( S \) and we're done.

Further any boundary \( \partial M \) is a hypersurface and there always exists at least one outward pointing vector field,
which is then nowhere tangent to \( \partial M \), so should \( M \) be orientable any orientation form \( \omega \) on \( M \) 
can be made one for \( \partial M \) by \( \iota^\ast_{\partial M}(\sigma \lrcorner \omega) \).

Orientated Riemmanian manifolds enjoy a canonical choice for their orientation form, elaborate 

It can be shown that there exists an orientation form \( \omega_g \in \Omega^n(M) \) called 
the Riemmanian volume form satisfying for each local orientated orthonormal frame \( (E_i) \)
\( \omega_g(E_1, \dots, E_n) = 1 \)
and can expressed locally as
\( \omega_g = \sqrt{\text{det}(g_{ij})} dx^1 \wedge \dots \wedge dx^n \).

This is useful for integration  as seen in Definition 7.1.25 and Lemma 7.1.16 in the lecture.


Define the integral of an \( m \)-form

Let M be a manifold, \(\epsilon > 0\) and \(x: U \to W_\epsilon\) be a chart around \(p \in U\) with
\[W_\epsilon := \{\xi \in \mathbb{R}^m : |\xi^i| < \epsilon, i = 1, \dots, m\} \]
and \(x(p) = 0\). Then we call \(x: U \to W_\epsilon\) a cubic coordinate system around \(p\) with radius \(\epsilon\).

Let \( \omega  \) be an \( m \)-form with \( \text{supp } \omega \subset U \) for some pos. orientated cubic coord. sys. \( x : U \to W_\varepsilon \),
s.t.
\( \omega = f dx^1 \wedge \dots \wedge x^m \)
we define the local integral of \( \omega \) as
\( \int_M \omega \coloneqq \int_{[\varepsilon, \varepsilon]} \dots \int_{[\varepsilon, \varepsilon]} f(x) dx^1 \dots dx^m \),
this is actually independent of particular choices of the charts since we know that for \( \phi = \tilde{x} \circ x^{-1} \)
\( \omega = f dx^1 \wedge \dots \wedge x^m = f(\phi(\tilde{x})) |\text{det}(D\phi)| d\tilde{x}^1 \wedge \dots \wedge d\tilde{x}^m \)
\( = \tilde{f} \wedge \dots \wedge d\tilde{x}^m\)
so by change of variables theorem
\( \int_{x(U \cap \tilde{U}} f(x) dx = \int_{\tilde{x}(U \cap \tilde{U}} f(\phi(\tilde{x})) |\text{det}(D\phi)| d\tilde{x} = \int_{\tilde{x}(U \cap \tilde{U}} \tilde{f} d\tilde{x} \)

We can globalize this notion for an \( \omega \) with compact \( \text{supp } \omega \). Let \( (f_\alpha, \{U_\alpha\}_{\alpha}) \) be a partition of unity 
with respect to some locally finite covering of \( M \), thus \( U_\alpha \cap \text{supp } \neq \emptyset \) but also always only finitely many \( \alpha \)
are involved then note that
\( \omega = \sum_\alpha f_\alpha \omega \)
so set \( \omega_\alpha = f_\alpha \omega \).

This is actually independent of the choice for the partition, which can be seen that for \( (U_\alpha, \omega_\alpha) \), \( (U_\beta, \omega_\beta) \)
we can define a new one \((U_\gamma, \omega_\gamma) \) with \( K = \bigcup_{\alpha, \beta} (\text{supp } \omega_\alpha \cap U_\alpha) \cup (\text{supp } \omega_\beta \cap U_\beta) \) and \( U_\gamma \) being a partition of unity of \( K \).
And we have
\( \int_M f_\gamma \omega = \sum_\alpha \int_M f_\gamma \omega_\alpha = \sum_\beta \int_M f_\gamma \omega_\beta \)
as well as
\( \sum_\alpha \int_M \omega_\alpha = \sum_\alpha \sum_\gamma \int f_\gamma \omega_\alpha = \sum_\gamma \sum_\alpha \int f_\gamma \omega_\alpha \)
\( = \sum_\gamma \int_M f_\gamma \omega_\beta = \sum_\beta \int_M \omega_\beta\)

Thus we can define uniquely

\( \int_M \omega \coloneqq \sum_\alpha \int_M \omega_\alpha \).


State and prove Stokes Theorem

Let \( M \) be a compact orientable \( m \)-manifold with boundary \( \partial M \).
Then for each \( \omega \in \Omega^{m-1}(M) \)
\( \int_M d\omega = \int_{\partial M} i^\ast \omega \),
where \( i : \partial M \to M \) is the canonical inclusion.

Since the integral is a sum of compactly supported smooth forms in a positively
orientated cubic coordinate system it suffices to show this for a single arbitrary summand.

There are 2 cases interior points and one the boundary:

Let \( x: U \to \Omega \) be a cubic coordinate system around an interior point \( p \in M \), 
that is \(\Omega = W_\epsilon \) with \(\epsilon > 0 \), \(U \cap \partial M = \emptyset\). 
The \((m-1)\)-form \(\omega\) can be written in these coordinates in the form

\(\omega = \sum_{j=1}^m (-1)^{j-1} a_j dx^1 \wedge \dots \wedge dx^{j-1} \wedge dx^{j+1} \wedge \dots \wedge dx^m \)

with suitable functions \(a_1, \dots, a_m\) and then holds

\(d\omega = \left( \sum_{j=1}^m \frac{\partial a_j}{\partial x^j} \right) dx^1 \wedge \dots \wedge dx^m. \)

Since \(\text{supp} \, \omega \subset U\), it follows in this coordinate system from the fundamental theorem of differential and integral calculus

\(\int_M d\omega = \int_M \left( \sum_{j=1}^m \frac{\partial a_j}{\partial x^j} \right) dx^1 \dots dx^m \)
\(= \sum_{j=1}^m \int_{[- \epsilon, \epsilon]^{m-1}} (a_j(x^1, \dots, x^{j-1}, \epsilon, x^{j+1}, \dots, x^m) - a_j(x^1, \dots, x^{j-1}, -\epsilon, x^{j+1}, \dots, x^m)) dx^1 \dots dx^{j-1} dx^{j+1} \dots dx^m. \)

Since the support of \(\omega\) in this chart is contained in \(W_\epsilon\), it holds

\(a_j(x^1, \dots, x^{j-1}, \epsilon, x^{j+1}, \dots, x^m) = a_j(x^1, \dots, x^{j-1}, -\epsilon, x^{j+1}, \dots, x^m) = 0.\)

Thus,

\(\int_M d\omega = 0.\)

2. 
\(x: U \to \Omega\) is a cubic coordinate system around a boundary point \(p \in \partial M\), 
that is \(\Omega = Q_\epsilon\) with \(\epsilon > 0\), \(U \cap \partial M \neq \emptyset\). 
Here, now holds
\(\int_{M} d\omega = \int_0^\epsilon \left( \int_{[-\epsilon, \epsilon]^{m-1}} \left( \sum_{j=1}^m \frac{\partial a_j}{\partial x^j} \right) dx^1 \dotsm dx^{m-1} \right) dx^m \)
\(= \int_{[-\epsilon, \epsilon]^{m-1}} \left( \int_0^\epsilon \frac{\partial a_m}{\partial x^m} dx^m \right) dx^1 \dotsm dx^{m-1}, \)

because for the same reason as in the first part, here also
\(\int_{-\epsilon}^\epsilon \frac{\partial a_j}{\partial x^j} dx^j = 0\)
for \(j = 1, \dots, m-1\).

Because of \(a_m(x^1, \dots, x^{m-1}, \epsilon) = 0\), then overall
\(\int_M d\omega = -\int_{[-\epsilon, \epsilon]^{m-1}} a_m(x^1, \dots, x^{m-1}, 0) dx^1 \dotsm dx^{m-1}.\)
Now \((x^1, \dots, x^{m-1})\) is a local coordinate system for \(\partial M \cap U\) and since \(x^m(q) = 0\) for \(q \in U \cap \partial M\), there holds
\(i^* \omega = (-1)^{m-1} a_m (x^1, \dots, x^{m-1}, 0) dx^1 \wedge \dots \wedge dx^{m-1}.\)

By construction of the compatible orientation on \(\partial M\), \((x^1, \dots, x^{m-1})\) is positively oriented 
if \(m\) is even and otherwise negatively oriented. 
The definition of the integral thus implies:
• If \(m\) is even:
\(\int_{\partial M} i^* \omega = (-1)^{m-1} \int_{[-\epsilon, \epsilon]^{m-1}} a_m(x^1, \dots, x^{m-1}, 0) dx^1 \dotsm dx^{m-1}\)
• If \(m\) is odd:
\(\int_{\partial M} i^* \omega = (-1)^{m} \int_{[-\epsilon, \epsilon]^{m-1}} a_m(x^1, \dots, x^{m-1}, 0) dx^1 \dotsm dx^{m-1}\)
Either way
\( \int_{\partial M} i^\ast \omega = - \int_{[-\varepsilon, \varepsilon]^{m-1}} a_m(x^1, \dots, x^{m-1}, 0) dx^1 \dots dx^{m-1} = \int_M d\omega \)
and we're done.

State some immediate consequences from Stokes theorem

If \(M\) is a compact oriented manifold of dimension \(m\) with \(\partial M = \emptyset\), then
\(\int_{M} \sigma = 0\)
holds for all exact \(m\)-forms \(\sigma\).

Proof: By assumption, \(\sigma = d\omega\) with an \((m-1)\)-form \(\omega\). Stokes' theorem implies
\(\int_{M} \sigma = \int_{M} d\omega = \int_{\partial M} \omega = 0,\)
because the integral over the empty set vanishes.

Let \(M \subset \mathbb{R}^2\) be a compact 2-dimensional submanifold with boundary. For smooth functions \(f, g: M \rightarrow \mathbb{R}\), then 

\(\int_M \left( \frac{\partial g}{\partial x} - \frac{\partial f}{\partial y} \right) dx \, dy = \int_{\partial M} (f \, dx + g \, dy).\)

Proof: 
Consider the 1-form \(\omega = f \, dx + g \, dy\). 
Because of
\(d\omega = \left( \frac{\partial g}{\partial x} - \frac{\partial f}{\partial y} \right) dx \wedge dy,\)
the claim follows from Stokes' theorem.

Let \(\Omega \subset \mathbb{C}\) be a bounded domain with a smooth boundary. We set \(z = x + iy\). Then for every holomorphic function \(f\) on \(\Omega\)
\(\int_{\partial \Omega} f(z) dz = 0.\)
\textbf{Proof:} 
Because \(\Omega\) is a bounded domain with smooth boundary, \(M := \Omega\) is a compact differentiable submanifold with boundary. 
For a function \(f: \Omega \rightarrow \mathbb{C}\) we set 

\(f(z) = u(x,y) + i v(x, y)\)

We consider the complex 1-form

\(\omega = f(z)dz = (u + iv)(dx + idy) = u dx - v dy + i (v dx + u dy) = \omega_1 + i \omega_2\)
with 
\(\omega_1 = u dx - v dy, \quad \omega_2 = v dx + u dy.\)

Then it is
\(\int_{\partial \Omega} f(z) dz = \int_{\partial \Omega} \omega_1 + i \int_{\partial \Omega} \omega_2 = \int_{\Omega} d\omega.\)

Now, however, holds
\(d\omega = d\omega_1 + i d\omega_2 \)
and 
\(d\omega_1 = (-v_x + u_y) dx \wedge dy, \quad d\omega_2 = (u_x - v_y) dx \wedge dy.\)
Therefore, \(\Omega\) is closed exactly when the Cauchy-Riemann differential equations
\(u_x = v_y, \quad u_y = -v_x\)
are satisfied, so exactly when \(f\) is holomorphic, i.e., when \(f_{\bar{z}} = 0\). In this case, then
\(\int_{\partial \Omega} i^* (f(z)dz) = 0.\)



Proof the general divergence theorem and state some special cases

Let \( \Omega \) be a volume form on an orientable smooth m-manifold \( M \),
then for \( X \in \mathfrak{X}(M) \) we call 
\( (\text{div } X)\Omega \coloneqq \mathcal{L}_{X}\Omega \)
the divergence of \( X \) with respect to \( \Omega \).

The divergence theorem states:
Let \(M\) be a compact, oriented differentiable manifold with boundary.
Then for every volume form \(\Omega\) and every smooth vector field \(X \in \mathfrak{X}(M)\)

\(\int_{M} (\operatorname{div} X) \Omega = \int_{\partial M} i^*(X \lrcorner \Omega)\)

Proof: This follows immediately from Stokes' theorem, when we use the Cartan formula
\(\mathcal{L}_X \Omega = d(X \lrcorner \Omega) + X \lrcorner d\Omega = d(X \lrcorner \Omega)\),
and since \( \Omega \) is an \( m \)-form, \( d\Omega = 0 \).

This takes on concrete shape for example in \( \mathbb{R}^{3} \) as

Let \(\Omega \subset \mathbb{R}^3\) be a bounded domain with a smooth boundary. 
If \(\partial \Omega\) is equipped with the compatible orientation, then for every smooth vector field \(V \in \mathfrak{X}(\Omega)\)

\(\int_{\Omega} \operatorname{div} V \, dxdydz = \int_{\partial \Omega} \langle i^*V, \nu \rangle d\mu.\)

Here, \(d\mu\) is the induced area element on \(\partial \Omega\) and \(\nu\) is the outward-pointing unit normal vector.
In particular if \( F = (F^x, F^y, F^z) : U \to \partial \Omega \) we have \( \nu = \frac{F_{\zeta^1} \times F_{\zeta^2}}{\|F_{\zeta^1} \times F_{\zeta^2}\|} \)
where \( (\zeta^1, \zeta^2) \) are the local coordinates for \( \partial \Omega \), we can even express this with respect to the Riemannian metric
since \( \|F_{\zeta^1} \times F_{\zeta^2}\| = \sqrt{\text{det}g_{ij}} \) so that the RHS reads 
\( \int_U \langle F^\ast V, \nu \rangle \sqrt{\text{det}g_{ij}} d\zeta^{1} \wedge d\zeta^{2} \)

This is also often expressed as
\(\int_{\Omega} \operatorname{div} V \, dxdydz = \int_{\partial \Omega} i^*V \cdot d\vec{\nu}.\)
with the substitution \( \nu d\mu = d\vec{\nu} \)

In the special case that \( V = V^x \frac{\partial}{\partial x} + V^y \frac{\partial}{\partial y}\) consists only of two components,
we can use a simpler normal vector induced by a curve \( c : I \to \partial \Omega \) along the boundary, so
\( \nu \coloneqq \frac{1}{\|\dot{c}\|}(\dot{c}^y, -\dot{c}^x)^T\) and we have on the RHS \( \int_{I} \langle c^\ast V, \nu \rangle \|\dot{c}\|dt \).

Even more explicitely this states
Let \(\Omega \subset \mathbb{R}^3\) be a bounded domain with a smooth boundary. If \(f, g, h: \mathbb{R}^3 \to \mathbb{R}\) are smooth functions, then

\(\int_{\Omega} (f_x + g_y + h_z) dx dy dz = \int_{\partial \Omega} i^* (f dy \wedge dz + g dz \wedge dx + h dx \wedge dy).\)

One can give for this an even more elementary proof as 
\(\omega = f dy \wedge dz + g dz \wedge dx + h dx \wedge dy\)
it holds
\(d\omega = (f_x + g_y + h_z) dx \wedge dy \wedge dz.\)
Stokes theorem does the rest.


Define the Laplace operator and derive its expression with respect to the
Riemannian metric.

Let \( (M, g) \) be a Riemannian manifold and \( f \in C^2(M) \),
define the Laplace operator as \( \Delta : C^{k+2}(M) \to C^k(M) \)
with \( \Delta f \coloneqq \text{div} \nabla f \) 
and the definition of \( \nabla f \) that we got from the musical isomorphisms.
If \( \Delta f = 0 \) we call \( f \) harmonic.

We know that
\( \mathcal{L}_X \Omega = (\text{div } X)\Omega \)
and in particular
\( L_X \Omega (\frac{\partial}{\partial x^1}, \dots, \frac{\partial}{\partial x^m}) \)
\( = X(\Omega (\frac{\partial}{\partial x^1}, \dots, \frac{\partial}{\partial x^m}))\)
\( - \sum_{i=1}^m \Omega(\frac{\partial}{\partial x^1}, \dots, [X, \frac{\partial}{\partial x^i}], \dots \frac{\partial}{\partial x^m} \)

Further \( [X, \frac{\partial}{\partial x^i}] = - \sum_{j=1}^m \frac{\partial X^j}{\partial x^i}\frac{\partial}{\partial x^j} \)
and \( \Omega (\frac{\partial}{\partial x^1}, \dots, \frac{\partial}{\partial x^m}) = \sqrt{\text{det}g_{ij}} \)
thus if we set \( G = \sqrt{\text{det}g_{ij}} \) the divergence of \( X \) with respect to \( \Omega \) is
\( (\text{div} X)\sqrt{G} = \sum^m_{i=1} X^i \frac{\partial \sqrt{G}}{\partial x^i} + \sum^m_{i=1}\frac{\partial X^i}{\partial x^i}\sqrt{G}\)
which can also be written as 
\( (\text{div} X) = \frac{1}{2}\sum^m_{i=1} X^i \frac{\partial \log{G}}{\partial x^i} + \sum^m_{i=1}\frac{\partial X^i}{\partial x^i} \)

Now if we set \( X = \nabla f\) then it follows 

\( \Delta f = \sum^m_{i,j=1}g^{ij} \frac{\partial f}{\partial x^i \partial x^j} + (\sum^m_{i,j=1} \frac{\partial g^{ij}}{\partial x^i} + \frac{1}{2}g^{ij} \frac{\partial \log{G}}{\partial x^i}) \frac{\partial f}{\partial x^j} \).

State and prove the general divergence theorem with respect to a Riemannian metric

Let \(M, g\) be an oriented Riemannian manifold with boundary, \(d \mu\), \(d \sigma = \nu \lrcorner d\mu\) 
the induced volume forms on \(M\) and \(\partial M\) respectively, and \(\nu\) the outward-pointing unit normal vector field. 
Then for every smooth vector field \(X \in \mathfrak{X}(M)\)

\(\int_M \operatorname{div} X \, d\mu = \int_{\partial M} g(X, \nu) \, d\sigma.\)

Proof: Along the boundary holds 

\(X = g(X, \nu) \nu + X^\top,\)

where \(X^\top\) denotes the projection of X onto the tangent space of \(\partial M\).
Since
\(i^*(X^\top \lrcorner d\mu) = 0,\)
it follows
\(i^*(X \lrcorner d\mu) = g(X, \nu) \nu \lrcorner d\mu = g(X, \nu) d\sigma\)
and the theorem follows directly from the usual divergence theorem.


Prove that on closed orientable smooth manifolds every harmonic function is constant
and a certain integral identity for the Laplace operator

Preliminary notion: A compact manifold with \( \partial M = \emptyset \) is called closed.

For harmonic functions it holds

\( \text{div}(f \nabla f) = \|\nabla f\|^2 + f \Delta f = \|\nabla f\|^2 \)
and also \( \|\nabla f\|^2 = g(\nabla f, \nabla f) \).
The divergence theorem implies due to \( \partial M = \emptyset \)
\( \int_M \|\nabla f\|^2 d\mu = 0 \) but then \( \nabla f = 0 \), yielding the statement.

Let \( M \) be closed then for every \( f, h \in C^2(M) \)
\( \int_M f \cdot \Delta h d\mu = \int_M \Delta f \cdot h d\mu \).
Proof: Follows from \( \text{div}(f \nabla h - h \nabla f) = f\Delta h - h\Delta f \) and inserting the term into the divergence theorem.
